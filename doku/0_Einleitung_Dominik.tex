\chapter*{Einleitung}
\thispagestyle{empty}

\begin{center}
\emph{{\small Dominik Grodt}}
\end{center}

\bigskip

Unter \textit{Smoothed Particle Hydrodynamics} (SPH) wird eine Methode verstanden, hydrodynamische Gleichungen numerisch zu lösen und somit das Verhalten von beliebigen Flüssigkeiten zu simulieren. Die Flüssigkeit wird dabei mithilfe von Partikeln diskretisiert um die Anzahl an Akteuren und deren Berechnungen auf ein endliches Maß zu reduzieren.\\
Abhängig von den umliegenden Elementen, der Zeit und dem eigenen Zustand wird daraufhin die Zustandsveränderung berechnet, wobei die daran beteiligten Formeln reale - oder gewünschte - physikalische Kräfte nachahmen sollen. Durch das Verhalten der einzelnen Partikel entsteht auf diese Weise auf der Makroebene das gewünschte Gesamtverhalten der Flüssigkeit. \\
Gegenstand dieses Projektes ist neben der möglichst korrekten Simulation von Wasser mittels SPH auch eine möglichst ansprechende Visualisierung desselben. Dazu wird die zur Berechnung nötige Diskretisierung aufgehoben und eine auf den Partikeln aufbauende, aber kontinuierliche Wasseroberfläche generiert. Mithilfe von gängigen Methoden aus der Computergrafik wird diese daraufhin in einem weiteren Schritt verfeinert um den Eindruck von Wasser zu verstärken.