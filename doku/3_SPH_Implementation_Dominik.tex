\section{Implementation und Performance}


\begin{center}
\emph{{\small Dominik Grodt}}
\end{center}

\bigskip

Da der Fokus der Veranstaltung auf der Nebenläufigkeit der zu implementierenden Algorithmen und auf OpenCL liegt, werden im Folgenden die technischen Aspekte der Implementation erörtert. Dabei wird zuerst die zugrundeliegende Datenstruktur erläutert und darauf folgend die Umsetzung der oben dargestellten Berechnungen mittels OpenCL.

\subsection{Datenstruktur und Neighboring Search}
Im allgemeinen Fall müssen zur Berechnung des neuen Zustands eines Partikels alle anderen simulierten Partikel einbezogen werden. Dazu müssen allerdings $\mathcal O(n^2)$ Berechnungen durchgeführt werden, was, vor allem durch die Tatsache, dass Partikelsysteme dieser Art mit steigender Partikelzahl genauere Ergebnisse liefern, schnell zu Performanceproblemen führen kann.\\
Aus diesem Grunde musste eine Möglichkeit gefunden werden, die Zahl der Interaktionsberechnungen zu reduzieren, ohne die Genauigkeit zu vermindern. Von verschiedenen diskutierten und implementierten Ansätzen hat sich der folgende durchgesetzt, wobei die weiteren, nicht weiterverfolgten, Ansätze ebenfalls im weiteren Verlauf kurz skizziert werden.
\subsubsection{Speicherung und Sortierung der Partikelindices in einem vierdimensionalen Array}
Hierbei wird der zur Verfügung stehende Raum mithilfe eines dreidimensionalen Grids diskretisiert und die vierte Dimension dazu genutzt, die Indices aller Partikel innerhalb einer Zelle dieses Grids zu speichern. Dazu wird bei der Programminitialisierung ein - eindimensionales, aber im Kernel als vierdimensional addressiertes - Array angelegt, welches es einem Work Item ermöglicht, die Indices der Partikel in den umliegenden Zellen abzurufen und somit auf die Attribute dieser Partikel mithilfe der entsprechenden Buffer zuzugreifen.\\
In Abbildung~\ref{fig:datenstruktur_grid} wird die Datenstruktur für den dreidimensionalen Fall dargestellt, wobei die dritte Dimension alle in einer Zelle vorhandenen Partikel enthält.\\
\begin{figure}
  \centering
    % Graphic for TeX using PGF
% Title: D:\workspace\uos_pa_sph\doku\images\datenstruktur_grid.dia
% Creator: Dia v0.97.2
% CreationDate: Wed Sep 11 14:07:54 2013
% For: Nikki
% \usepackage{tikz}
% The following commands are not supported in PSTricks at present
% We define them conditionally, so when they are implemented,
% this pgf file will use them.
\ifx\du\undefined
  \newlength{\du}
\fi
\setlength{\du}{15\unitlength}
\begin{tikzpicture}
\pgftransformxscale{1.000000}
\pgftransformyscale{-1.000000}
\definecolor{dialinecolor}{rgb}{0.000000, 0.000000, 0.000000}
\pgfsetstrokecolor{dialinecolor}
\definecolor{dialinecolor}{rgb}{1.000000, 1.000000, 1.000000}
\pgfsetfillcolor{dialinecolor}
\pgfsetmiterjoin
\pgfsetdash{}{0pt}
\definecolor{dialinecolor}{rgb}{1.000000, 1.000000, 1.000000}
\pgfsetfillcolor{dialinecolor}
\fill (46.029684\du,24.094863\du)--(46.029684\du,34.863608\du)--(57.997986\du,34.863608\du)--(57.997986\du,24.094863\du)--cycle;
\pgfsetlinewidth{0.100000\du}
\definecolor{dialinecolor}{rgb}{0.000000, 0.000000, 0.000000}
\pgfsetstrokecolor{dialinecolor}
\draw (46.029684\du,26.248612\du)--(57.997986\du,26.248612\du);
\definecolor{dialinecolor}{rgb}{0.000000, 0.000000, 0.000000}
\pgfsetstrokecolor{dialinecolor}
\draw (46.029684\du,28.402361\du)--(57.997986\du,28.402361\du);
\definecolor{dialinecolor}{rgb}{0.000000, 0.000000, 0.000000}
\pgfsetstrokecolor{dialinecolor}
\draw (46.029684\du,30.556110\du)--(57.997986\du,30.556110\du);
\definecolor{dialinecolor}{rgb}{0.000000, 0.000000, 0.000000}
\pgfsetstrokecolor{dialinecolor}
\draw (46.029684\du,32.709859\du)--(57.997986\du,32.709859\du);
\definecolor{dialinecolor}{rgb}{0.000000, 0.000000, 0.000000}
\pgfsetstrokecolor{dialinecolor}
\draw (48.423345\du,24.094863\du)--(48.423345\du,34.863608\du);
\definecolor{dialinecolor}{rgb}{0.000000, 0.000000, 0.000000}
\pgfsetstrokecolor{dialinecolor}
\draw (50.817005\du,24.094863\du)--(50.817005\du,34.863608\du);
\definecolor{dialinecolor}{rgb}{0.000000, 0.000000, 0.000000}
\pgfsetstrokecolor{dialinecolor}
\draw (53.210666\du,24.094863\du)--(53.210666\du,34.863608\du);
\definecolor{dialinecolor}{rgb}{0.000000, 0.000000, 0.000000}
\pgfsetstrokecolor{dialinecolor}
\draw (55.604326\du,24.094863\du)--(55.604326\du,34.863608\du);
\pgfsetlinewidth{0.100000\du}
\definecolor{dialinecolor}{rgb}{0.000000, 0.000000, 0.000000}
\pgfsetstrokecolor{dialinecolor}
\draw (46.029684\du,24.094863\du)--(46.029684\du,34.863608\du)--(57.997986\du,34.863608\du)--(57.997986\du,24.094863\du)--cycle;
\pgfsetlinewidth{0.100000\du}
\pgfsetdash{}{0pt}
\pgfsetdash{}{0pt}
\pgfsetbuttcap
\pgfsetmiterjoin
\pgfsetlinewidth{0.100000\du}
\pgfsetbuttcap
\pgfsetmiterjoin
\pgfsetdash{}{0pt}
\definecolor{dialinecolor}{rgb}{1.000000, 1.000000, 1.000000}
\pgfsetfillcolor{dialinecolor}
\pgfpathellipse{\pgfpoint{47.890360\du}{24.755982\du}}{\pgfpoint{0.279442\du}{0\du}}{\pgfpoint{0\du}{0.279442\du}}
\pgfusepath{fill}
\definecolor{dialinecolor}{rgb}{0.000000, 0.000000, 0.000000}
\pgfsetstrokecolor{dialinecolor}
\pgfpathellipse{\pgfpoint{47.890360\du}{24.755982\du}}{\pgfpoint{0.279442\du}{0\du}}{\pgfpoint{0\du}{0.279442\du}}
\pgfusepath{stroke}
\pgfsetbuttcap
\pgfsetmiterjoin
\pgfsetdash{}{0pt}
\definecolor{dialinecolor}{rgb}{0.000000, 0.000000, 0.000000}
\pgfsetstrokecolor{dialinecolor}
\pgfpathellipse{\pgfpoint{47.890360\du}{24.755982\du}}{\pgfpoint{0.279442\du}{0\du}}{\pgfpoint{0\du}{0.279442\du}}
\pgfusepath{stroke}
\pgfsetlinewidth{0.100000\du}
\pgfsetdash{}{0pt}
\pgfsetdash{}{0pt}
\pgfsetbuttcap
\pgfsetmiterjoin
\pgfsetlinewidth{0.100000\du}
\pgfsetbuttcap
\pgfsetmiterjoin
\pgfsetdash{}{0pt}
\definecolor{dialinecolor}{rgb}{1.000000, 1.000000, 1.000000}
\pgfsetfillcolor{dialinecolor}
\pgfpathellipse{\pgfpoint{46.773519\du}{26.768892\du}}{\pgfpoint{0.279442\du}{0\du}}{\pgfpoint{0\du}{0.279442\du}}
\pgfusepath{fill}
\definecolor{dialinecolor}{rgb}{0.000000, 0.000000, 0.000000}
\pgfsetstrokecolor{dialinecolor}
\pgfpathellipse{\pgfpoint{46.773519\du}{26.768892\du}}{\pgfpoint{0.279442\du}{0\du}}{\pgfpoint{0\du}{0.279442\du}}
\pgfusepath{stroke}
\pgfsetbuttcap
\pgfsetmiterjoin
\pgfsetdash{}{0pt}
\definecolor{dialinecolor}{rgb}{0.000000, 0.000000, 0.000000}
\pgfsetstrokecolor{dialinecolor}
\pgfpathellipse{\pgfpoint{46.773519\du}{26.768892\du}}{\pgfpoint{0.279442\du}{0\du}}{\pgfpoint{0\du}{0.279442\du}}
\pgfusepath{stroke}
\pgfsetlinewidth{0.100000\du}
\pgfsetdash{}{0pt}
\pgfsetdash{}{0pt}
\pgfsetbuttcap
\pgfsetmiterjoin
\pgfsetlinewidth{0.100000\du}
\pgfsetbuttcap
\pgfsetmiterjoin
\pgfsetdash{}{0pt}
\definecolor{dialinecolor}{rgb}{1.000000, 1.000000, 1.000000}
\pgfsetfillcolor{dialinecolor}
\pgfpathellipse{\pgfpoint{47.160648\du}{28.895379\du}}{\pgfpoint{0.279442\du}{0\du}}{\pgfpoint{0\du}{0.279442\du}}
\pgfusepath{fill}
\definecolor{dialinecolor}{rgb}{0.000000, 0.000000, 0.000000}
\pgfsetstrokecolor{dialinecolor}
\pgfpathellipse{\pgfpoint{47.160648\du}{28.895379\du}}{\pgfpoint{0.279442\du}{0\du}}{\pgfpoint{0\du}{0.279442\du}}
\pgfusepath{stroke}
\pgfsetbuttcap
\pgfsetmiterjoin
\pgfsetdash{}{0pt}
\definecolor{dialinecolor}{rgb}{0.000000, 0.000000, 0.000000}
\pgfsetstrokecolor{dialinecolor}
\pgfpathellipse{\pgfpoint{47.160648\du}{28.895379\du}}{\pgfpoint{0.279442\du}{0\du}}{\pgfpoint{0\du}{0.279442\du}}
\pgfusepath{stroke}
\pgfsetlinewidth{0.100000\du}
\pgfsetdash{}{0pt}
\pgfsetdash{}{0pt}
\pgfsetbuttcap
\pgfsetmiterjoin
\pgfsetlinewidth{0.100000\du}
\pgfsetbuttcap
\pgfsetmiterjoin
\pgfsetdash{}{0pt}
\definecolor{dialinecolor}{rgb}{1.000000, 1.000000, 1.000000}
\pgfsetfillcolor{dialinecolor}
\pgfpathellipse{\pgfpoint{49.919628\du}{27.614034\du}}{\pgfpoint{0.279442\du}{0\du}}{\pgfpoint{0\du}{0.279442\du}}
\pgfusepath{fill}
\definecolor{dialinecolor}{rgb}{0.000000, 0.000000, 0.000000}
\pgfsetstrokecolor{dialinecolor}
\pgfpathellipse{\pgfpoint{49.919628\du}{27.614034\du}}{\pgfpoint{0.279442\du}{0\du}}{\pgfpoint{0\du}{0.279442\du}}
\pgfusepath{stroke}
\pgfsetbuttcap
\pgfsetmiterjoin
\pgfsetdash{}{0pt}
\definecolor{dialinecolor}{rgb}{0.000000, 0.000000, 0.000000}
\pgfsetstrokecolor{dialinecolor}
\pgfpathellipse{\pgfpoint{49.919628\du}{27.614034\du}}{\pgfpoint{0.279442\du}{0\du}}{\pgfpoint{0\du}{0.279442\du}}
\pgfusepath{stroke}
\pgfsetlinewidth{0.100000\du}
\pgfsetdash{}{0pt}
\pgfsetdash{}{0pt}
\pgfsetbuttcap
\pgfsetmiterjoin
\pgfsetlinewidth{0.100000\du}
\pgfsetbuttcap
\pgfsetmiterjoin
\pgfsetdash{}{0pt}
\definecolor{dialinecolor}{rgb}{1.000000, 1.000000, 1.000000}
\pgfsetfillcolor{dialinecolor}
\pgfpathellipse{\pgfpoint{48.970888\du}{26.741630\du}}{\pgfpoint{0.279442\du}{0\du}}{\pgfpoint{0\du}{0.279442\du}}
\pgfusepath{fill}
\definecolor{dialinecolor}{rgb}{0.000000, 0.000000, 0.000000}
\pgfsetstrokecolor{dialinecolor}
\pgfpathellipse{\pgfpoint{48.970888\du}{26.741630\du}}{\pgfpoint{0.279442\du}{0\du}}{\pgfpoint{0\du}{0.279442\du}}
\pgfusepath{stroke}
\pgfsetbuttcap
\pgfsetmiterjoin
\pgfsetdash{}{0pt}
\definecolor{dialinecolor}{rgb}{0.000000, 0.000000, 0.000000}
\pgfsetstrokecolor{dialinecolor}
\pgfpathellipse{\pgfpoint{48.970888\du}{26.741630\du}}{\pgfpoint{0.279442\du}{0\du}}{\pgfpoint{0\du}{0.279442\du}}
\pgfusepath{stroke}
\pgfsetlinewidth{0.100000\du}
\pgfsetdash{}{0pt}
\pgfsetdash{}{0pt}
\pgfsetbuttcap
\pgfsetmiterjoin
\pgfsetlinewidth{0.100000\du}
\pgfsetbuttcap
\pgfsetmiterjoin
\pgfsetdash{}{0pt}
\definecolor{dialinecolor}{rgb}{1.000000, 1.000000, 1.000000}
\pgfsetfillcolor{dialinecolor}
\pgfpathellipse{\pgfpoint{49.194442\du}{29.795046\du}}{\pgfpoint{0.279442\du}{0\du}}{\pgfpoint{0\du}{0.279442\du}}
\pgfusepath{fill}
\definecolor{dialinecolor}{rgb}{0.000000, 0.000000, 0.000000}
\pgfsetstrokecolor{dialinecolor}
\pgfpathellipse{\pgfpoint{49.194442\du}{29.795046\du}}{\pgfpoint{0.279442\du}{0\du}}{\pgfpoint{0\du}{0.279442\du}}
\pgfusepath{stroke}
\pgfsetbuttcap
\pgfsetmiterjoin
\pgfsetdash{}{0pt}
\definecolor{dialinecolor}{rgb}{0.000000, 0.000000, 0.000000}
\pgfsetstrokecolor{dialinecolor}
\pgfpathellipse{\pgfpoint{49.194442\du}{29.795046\du}}{\pgfpoint{0.279442\du}{0\du}}{\pgfpoint{0\du}{0.279442\du}}
\pgfusepath{stroke}
\pgfsetlinewidth{0.100000\du}
\pgfsetdash{}{0pt}
\pgfsetdash{}{0pt}
\pgfsetbuttcap
\pgfsetmiterjoin
\pgfsetlinewidth{0.100000\du}
\pgfsetbuttcap
\pgfsetmiterjoin
\pgfsetdash{}{0pt}
\definecolor{dialinecolor}{rgb}{1.000000, 1.000000, 1.000000}
\pgfsetfillcolor{dialinecolor}
\pgfpathellipse{\pgfpoint{47.782237\du}{31.267229\du}}{\pgfpoint{0.279442\du}{0\du}}{\pgfpoint{0\du}{0.279442\du}}
\pgfusepath{fill}
\definecolor{dialinecolor}{rgb}{0.000000, 0.000000, 0.000000}
\pgfsetstrokecolor{dialinecolor}
\pgfpathellipse{\pgfpoint{47.782237\du}{31.267229\du}}{\pgfpoint{0.279442\du}{0\du}}{\pgfpoint{0\du}{0.279442\du}}
\pgfusepath{stroke}
\pgfsetbuttcap
\pgfsetmiterjoin
\pgfsetdash{}{0pt}
\definecolor{dialinecolor}{rgb}{0.000000, 0.000000, 0.000000}
\pgfsetstrokecolor{dialinecolor}
\pgfpathellipse{\pgfpoint{47.782237\du}{31.267229\du}}{\pgfpoint{0.279442\du}{0\du}}{\pgfpoint{0\du}{0.279442\du}}
\pgfusepath{stroke}
\pgfsetlinewidth{0.100000\du}
\pgfsetdash{}{0pt}
\pgfsetdash{}{0pt}
\pgfsetbuttcap
\pgfsetmiterjoin
\pgfsetlinewidth{0.100000\du}
\pgfsetbuttcap
\pgfsetmiterjoin
\pgfsetdash{}{0pt}
\definecolor{dialinecolor}{rgb}{1.000000, 1.000000, 1.000000}
\pgfsetfillcolor{dialinecolor}
\pgfpathellipse{\pgfpoint{49.396185\du}{31.839745\du}}{\pgfpoint{0.279442\du}{0\du}}{\pgfpoint{0\du}{0.279442\du}}
\pgfusepath{fill}
\definecolor{dialinecolor}{rgb}{0.000000, 0.000000, 0.000000}
\pgfsetstrokecolor{dialinecolor}
\pgfpathellipse{\pgfpoint{49.396185\du}{31.839745\du}}{\pgfpoint{0.279442\du}{0\du}}{\pgfpoint{0\du}{0.279442\du}}
\pgfusepath{stroke}
\pgfsetbuttcap
\pgfsetmiterjoin
\pgfsetdash{}{0pt}
\definecolor{dialinecolor}{rgb}{0.000000, 0.000000, 0.000000}
\pgfsetstrokecolor{dialinecolor}
\pgfpathellipse{\pgfpoint{49.396185\du}{31.839745\du}}{\pgfpoint{0.279442\du}{0\du}}{\pgfpoint{0\du}{0.279442\du}}
\pgfusepath{stroke}
\pgfsetlinewidth{0.100000\du}
\pgfsetdash{}{0pt}
\pgfsetdash{}{0pt}
\pgfsetbuttcap
\pgfsetmiterjoin
\pgfsetlinewidth{0.100000\du}
\pgfsetbuttcap
\pgfsetmiterjoin
\pgfsetdash{}{0pt}
\definecolor{dialinecolor}{rgb}{1.000000, 1.000000, 1.000000}
\pgfsetfillcolor{dialinecolor}
\pgfpathellipse{\pgfpoint{48.038506\du}{32.902988\du}}{\pgfpoint{0.279442\du}{0\du}}{\pgfpoint{0\du}{0.279442\du}}
\pgfusepath{fill}
\definecolor{dialinecolor}{rgb}{0.000000, 0.000000, 0.000000}
\pgfsetstrokecolor{dialinecolor}
\pgfpathellipse{\pgfpoint{48.038506\du}{32.902988\du}}{\pgfpoint{0.279442\du}{0\du}}{\pgfpoint{0\du}{0.279442\du}}
\pgfusepath{stroke}
\pgfsetbuttcap
\pgfsetmiterjoin
\pgfsetdash{}{0pt}
\definecolor{dialinecolor}{rgb}{0.000000, 0.000000, 0.000000}
\pgfsetstrokecolor{dialinecolor}
\pgfpathellipse{\pgfpoint{48.038506\du}{32.902988\du}}{\pgfpoint{0.279442\du}{0\du}}{\pgfpoint{0\du}{0.279442\du}}
\pgfusepath{stroke}
\pgfsetlinewidth{0.100000\du}
\pgfsetdash{}{0pt}
\pgfsetdash{}{0pt}
\pgfsetbuttcap
\pgfsetmiterjoin
\pgfsetlinewidth{0.100000\du}
\pgfsetbuttcap
\pgfsetmiterjoin
\pgfsetdash{}{0pt}
\definecolor{dialinecolor}{rgb}{1.000000, 1.000000, 1.000000}
\pgfsetfillcolor{dialinecolor}
\pgfpathellipse{\pgfpoint{48.834575\du}{34.593272\du}}{\pgfpoint{0.279442\du}{0\du}}{\pgfpoint{0\du}{0.279442\du}}
\pgfusepath{fill}
\definecolor{dialinecolor}{rgb}{0.000000, 0.000000, 0.000000}
\pgfsetstrokecolor{dialinecolor}
\pgfpathellipse{\pgfpoint{48.834575\du}{34.593272\du}}{\pgfpoint{0.279442\du}{0\du}}{\pgfpoint{0\du}{0.279442\du}}
\pgfusepath{stroke}
\pgfsetbuttcap
\pgfsetmiterjoin
\pgfsetdash{}{0pt}
\definecolor{dialinecolor}{rgb}{0.000000, 0.000000, 0.000000}
\pgfsetstrokecolor{dialinecolor}
\pgfpathellipse{\pgfpoint{48.834575\du}{34.593272\du}}{\pgfpoint{0.279442\du}{0\du}}{\pgfpoint{0\du}{0.279442\du}}
\pgfusepath{stroke}
\pgfsetlinewidth{0.100000\du}
\pgfsetdash{}{0pt}
\pgfsetdash{}{0pt}
\pgfsetbuttcap
\pgfsetmiterjoin
\pgfsetlinewidth{0.100000\du}
\pgfsetbuttcap
\pgfsetmiterjoin
\pgfsetdash{}{0pt}
\definecolor{dialinecolor}{rgb}{1.000000, 1.000000, 1.000000}
\pgfsetfillcolor{dialinecolor}
\pgfpathellipse{\pgfpoint{46.582680\du}{33.202877\du}}{\pgfpoint{0.279442\du}{0\du}}{\pgfpoint{0\du}{0.279442\du}}
\pgfusepath{fill}
\definecolor{dialinecolor}{rgb}{0.000000, 0.000000, 0.000000}
\pgfsetstrokecolor{dialinecolor}
\pgfpathellipse{\pgfpoint{46.582680\du}{33.202877\du}}{\pgfpoint{0.279442\du}{0\du}}{\pgfpoint{0\du}{0.279442\du}}
\pgfusepath{stroke}
\pgfsetbuttcap
\pgfsetmiterjoin
\pgfsetdash{}{0pt}
\definecolor{dialinecolor}{rgb}{0.000000, 0.000000, 0.000000}
\pgfsetstrokecolor{dialinecolor}
\pgfpathellipse{\pgfpoint{46.582680\du}{33.202877\du}}{\pgfpoint{0.279442\du}{0\du}}{\pgfpoint{0\du}{0.279442\du}}
\pgfusepath{stroke}
\pgfsetlinewidth{0.100000\du}
\pgfsetdash{}{0pt}
\pgfsetdash{}{0pt}
\pgfsetbuttcap
\pgfsetmiterjoin
\pgfsetlinewidth{0.100000\du}
\pgfsetbuttcap
\pgfsetmiterjoin
\pgfsetdash{}{0pt}
\definecolor{dialinecolor}{rgb}{1.000000, 1.000000, 1.000000}
\pgfsetfillcolor{dialinecolor}
\pgfpathellipse{\pgfpoint{52.395076\du}{25.869225\du}}{\pgfpoint{0.279442\du}{0\du}}{\pgfpoint{0\du}{0.279442\du}}
\pgfusepath{fill}
\definecolor{dialinecolor}{rgb}{0.000000, 0.000000, 0.000000}
\pgfsetstrokecolor{dialinecolor}
\pgfpathellipse{\pgfpoint{52.395076\du}{25.869225\du}}{\pgfpoint{0.279442\du}{0\du}}{\pgfpoint{0\du}{0.279442\du}}
\pgfusepath{stroke}
\pgfsetbuttcap
\pgfsetmiterjoin
\pgfsetdash{}{0pt}
\definecolor{dialinecolor}{rgb}{0.000000, 0.000000, 0.000000}
\pgfsetstrokecolor{dialinecolor}
\pgfpathellipse{\pgfpoint{52.395076\du}{25.869225\du}}{\pgfpoint{0.279442\du}{0\du}}{\pgfpoint{0\du}{0.279442\du}}
\pgfusepath{stroke}
\pgfsetlinewidth{0.100000\du}
\pgfsetdash{}{0pt}
\pgfsetdash{}{0pt}
\pgfsetbuttcap
\pgfsetmiterjoin
\pgfsetlinewidth{0.100000\du}
\pgfsetbuttcap
\pgfsetmiterjoin
\pgfsetdash{}{0pt}
\definecolor{dialinecolor}{rgb}{1.000000, 1.000000, 1.000000}
\pgfsetfillcolor{dialinecolor}
\pgfpathellipse{\pgfpoint{54.417965\du}{27.941186\du}}{\pgfpoint{0.279442\du}{0\du}}{\pgfpoint{0\du}{0.279442\du}}
\pgfusepath{fill}
\definecolor{dialinecolor}{rgb}{0.000000, 0.000000, 0.000000}
\pgfsetstrokecolor{dialinecolor}
\pgfpathellipse{\pgfpoint{54.417965\du}{27.941186\du}}{\pgfpoint{0.279442\du}{0\du}}{\pgfpoint{0\du}{0.279442\du}}
\pgfusepath{stroke}
\pgfsetbuttcap
\pgfsetmiterjoin
\pgfsetdash{}{0pt}
\definecolor{dialinecolor}{rgb}{0.000000, 0.000000, 0.000000}
\pgfsetstrokecolor{dialinecolor}
\pgfpathellipse{\pgfpoint{54.417965\du}{27.941186\du}}{\pgfpoint{0.279442\du}{0\du}}{\pgfpoint{0\du}{0.279442\du}}
\pgfusepath{stroke}
\pgfsetlinewidth{0.100000\du}
\pgfsetdash{}{0pt}
\pgfsetdash{}{0pt}
\pgfsetbuttcap
\pgfsetmiterjoin
\pgfsetlinewidth{0.100000\du}
\pgfsetbuttcap
\pgfsetmiterjoin
\pgfsetdash{}{0pt}
\definecolor{dialinecolor}{rgb}{1.000000, 1.000000, 1.000000}
\pgfsetfillcolor{dialinecolor}
\pgfpathellipse{\pgfpoint{52.733133\du}{28.050237\du}}{\pgfpoint{0.279442\du}{0\du}}{\pgfpoint{0\du}{0.279442\du}}
\pgfusepath{fill}
\definecolor{dialinecolor}{rgb}{0.000000, 0.000000, 0.000000}
\pgfsetstrokecolor{dialinecolor}
\pgfpathellipse{\pgfpoint{52.733133\du}{28.050237\du}}{\pgfpoint{0.279442\du}{0\du}}{\pgfpoint{0\du}{0.279442\du}}
\pgfusepath{stroke}
\pgfsetbuttcap
\pgfsetmiterjoin
\pgfsetdash{}{0pt}
\definecolor{dialinecolor}{rgb}{0.000000, 0.000000, 0.000000}
\pgfsetstrokecolor{dialinecolor}
\pgfpathellipse{\pgfpoint{52.733133\du}{28.050237\du}}{\pgfpoint{0.279442\du}{0\du}}{\pgfpoint{0\du}{0.279442\du}}
\pgfusepath{stroke}
\pgfsetlinewidth{0.100000\du}
\pgfsetdash{}{0pt}
\pgfsetdash{}{0pt}
\pgfsetbuttcap
\pgfsetmiterjoin
\pgfsetlinewidth{0.100000\du}
\pgfsetbuttcap
\pgfsetmiterjoin
\pgfsetdash{}{0pt}
\definecolor{dialinecolor}{rgb}{1.000000, 1.000000, 1.000000}
\pgfsetfillcolor{dialinecolor}
\pgfpathellipse{\pgfpoint{55.034101\du}{29.031692\du}}{\pgfpoint{0.279442\du}{0\du}}{\pgfpoint{0\du}{0.279442\du}}
\pgfusepath{fill}
\definecolor{dialinecolor}{rgb}{0.000000, 0.000000, 0.000000}
\pgfsetstrokecolor{dialinecolor}
\pgfpathellipse{\pgfpoint{55.034101\du}{29.031692\du}}{\pgfpoint{0.279442\du}{0\du}}{\pgfpoint{0\du}{0.279442\du}}
\pgfusepath{stroke}
\pgfsetbuttcap
\pgfsetmiterjoin
\pgfsetdash{}{0pt}
\definecolor{dialinecolor}{rgb}{0.000000, 0.000000, 0.000000}
\pgfsetstrokecolor{dialinecolor}
\pgfpathellipse{\pgfpoint{55.034101\du}{29.031692\du}}{\pgfpoint{0.279442\du}{0\du}}{\pgfpoint{0\du}{0.279442\du}}
\pgfusepath{stroke}
\pgfsetlinewidth{0.100000\du}
\pgfsetdash{}{0pt}
\pgfsetdash{}{0pt}
\pgfsetbuttcap
\pgfsetmiterjoin
\pgfsetlinewidth{0.100000\du}
\pgfsetbuttcap
\pgfsetmiterjoin
\pgfsetdash{}{0pt}
\definecolor{dialinecolor}{rgb}{0.000000, 0.000000, 0.000000}
\pgfsetfillcolor{dialinecolor}
\pgfpathellipse{\pgfpoint{52.558652\du}{29.004429\du}}{\pgfpoint{0.279442\du}{0\du}}{\pgfpoint{0\du}{0.279442\du}}
\pgfusepath{fill}
\definecolor{dialinecolor}{rgb}{0.000000, 0.000000, 0.000000}
\pgfsetstrokecolor{dialinecolor}
\pgfpathellipse{\pgfpoint{52.558652\du}{29.004429\du}}{\pgfpoint{0.279442\du}{0\du}}{\pgfpoint{0\du}{0.279442\du}}
\pgfusepath{stroke}
\pgfsetbuttcap
\pgfsetmiterjoin
\pgfsetdash{}{0pt}
\definecolor{dialinecolor}{rgb}{0.000000, 0.000000, 0.000000}
\pgfsetstrokecolor{dialinecolor}
\pgfpathellipse{\pgfpoint{52.558652\du}{29.004429\du}}{\pgfpoint{0.279442\du}{0\du}}{\pgfpoint{0\du}{0.279442\du}}
\pgfusepath{stroke}
\pgfsetlinewidth{0.100000\du}
\pgfsetdash{}{0pt}
\pgfsetdash{}{0pt}
\pgfsetbuttcap
\pgfsetmiterjoin
\pgfsetlinewidth{0.100000\du}
\pgfsetbuttcap
\pgfsetmiterjoin
\pgfsetdash{}{0pt}
\definecolor{dialinecolor}{rgb}{1.000000, 1.000000, 1.000000}
\pgfsetfillcolor{dialinecolor}
\pgfpathellipse{\pgfpoint{51.528124\du}{29.549682\du}}{\pgfpoint{0.279442\du}{0\du}}{\pgfpoint{0\du}{0.279442\du}}
\pgfusepath{fill}
\definecolor{dialinecolor}{rgb}{0.000000, 0.000000, 0.000000}
\pgfsetstrokecolor{dialinecolor}
\pgfpathellipse{\pgfpoint{51.528124\du}{29.549682\du}}{\pgfpoint{0.279442\du}{0\du}}{\pgfpoint{0\du}{0.279442\du}}
\pgfusepath{stroke}
\pgfsetbuttcap
\pgfsetmiterjoin
\pgfsetdash{}{0pt}
\definecolor{dialinecolor}{rgb}{0.000000, 0.000000, 0.000000}
\pgfsetstrokecolor{dialinecolor}
\pgfpathellipse{\pgfpoint{51.528124\du}{29.549682\du}}{\pgfpoint{0.279442\du}{0\du}}{\pgfpoint{0\du}{0.279442\du}}
\pgfusepath{stroke}
\pgfsetlinewidth{0.100000\du}
\pgfsetdash{}{0pt}
\pgfsetdash{}{0pt}
\pgfsetbuttcap
\pgfsetmiterjoin
\pgfsetlinewidth{0.100000\du}
\pgfsetbuttcap
\pgfsetmiterjoin
\pgfsetdash{}{0pt}
\definecolor{dialinecolor}{rgb}{1.000000, 1.000000, 1.000000}
\pgfsetfillcolor{dialinecolor}
\pgfpathellipse{\pgfpoint{51.288213\du}{24.642406\du}}{\pgfpoint{0.279442\du}{0\du}}{\pgfpoint{0\du}{0.279442\du}}
\pgfusepath{fill}
\definecolor{dialinecolor}{rgb}{0.000000, 0.000000, 0.000000}
\pgfsetstrokecolor{dialinecolor}
\pgfpathellipse{\pgfpoint{51.288213\du}{24.642406\du}}{\pgfpoint{0.279442\du}{0\du}}{\pgfpoint{0\du}{0.279442\du}}
\pgfusepath{stroke}
\pgfsetbuttcap
\pgfsetmiterjoin
\pgfsetdash{}{0pt}
\definecolor{dialinecolor}{rgb}{0.000000, 0.000000, 0.000000}
\pgfsetstrokecolor{dialinecolor}
\pgfpathellipse{\pgfpoint{51.288213\du}{24.642406\du}}{\pgfpoint{0.279442\du}{0\du}}{\pgfpoint{0\du}{0.279442\du}}
\pgfusepath{stroke}
\pgfsetlinewidth{0.100000\du}
\pgfsetdash{}{0pt}
\pgfsetdash{}{0pt}
\pgfsetbuttcap
\pgfsetmiterjoin
\pgfsetlinewidth{0.100000\du}
\pgfsetbuttcap
\pgfsetmiterjoin
\pgfsetdash{}{0pt}
\definecolor{dialinecolor}{rgb}{1.000000, 1.000000, 1.000000}
\pgfsetfillcolor{dialinecolor}
\pgfpathellipse{\pgfpoint{52.771301\du}{30.994603\du}}{\pgfpoint{0.279442\du}{0\du}}{\pgfpoint{0\du}{0.279442\du}}
\pgfusepath{fill}
\definecolor{dialinecolor}{rgb}{0.000000, 0.000000, 0.000000}
\pgfsetstrokecolor{dialinecolor}
\pgfpathellipse{\pgfpoint{52.771301\du}{30.994603\du}}{\pgfpoint{0.279442\du}{0\du}}{\pgfpoint{0\du}{0.279442\du}}
\pgfusepath{stroke}
\pgfsetbuttcap
\pgfsetmiterjoin
\pgfsetdash{}{0pt}
\definecolor{dialinecolor}{rgb}{0.000000, 0.000000, 0.000000}
\pgfsetstrokecolor{dialinecolor}
\pgfpathellipse{\pgfpoint{52.771301\du}{30.994603\du}}{\pgfpoint{0.279442\du}{0\du}}{\pgfpoint{0\du}{0.279442\du}}
\pgfusepath{stroke}
\pgfsetlinewidth{0.100000\du}
\pgfsetdash{}{0pt}
\pgfsetdash{}{0pt}
\pgfsetbuttcap
\pgfsetmiterjoin
\pgfsetlinewidth{0.100000\du}
\pgfsetbuttcap
\pgfsetmiterjoin
\pgfsetdash{}{0pt}
\definecolor{dialinecolor}{rgb}{1.000000, 1.000000, 1.000000}
\pgfsetfillcolor{dialinecolor}
\pgfpathellipse{\pgfpoint{52.149713\du}{33.339190\du}}{\pgfpoint{0.279442\du}{0\du}}{\pgfpoint{0\du}{0.279442\du}}
\pgfusepath{fill}
\definecolor{dialinecolor}{rgb}{0.000000, 0.000000, 0.000000}
\pgfsetstrokecolor{dialinecolor}
\pgfpathellipse{\pgfpoint{52.149713\du}{33.339190\du}}{\pgfpoint{0.279442\du}{0\du}}{\pgfpoint{0\du}{0.279442\du}}
\pgfusepath{stroke}
\pgfsetbuttcap
\pgfsetmiterjoin
\pgfsetdash{}{0pt}
\definecolor{dialinecolor}{rgb}{0.000000, 0.000000, 0.000000}
\pgfsetstrokecolor{dialinecolor}
\pgfpathellipse{\pgfpoint{52.149713\du}{33.339190\du}}{\pgfpoint{0.279442\du}{0\du}}{\pgfpoint{0\du}{0.279442\du}}
\pgfusepath{stroke}
\pgfsetlinewidth{0.100000\du}
\pgfsetdash{}{0pt}
\pgfsetdash{}{0pt}
\pgfsetbuttcap
\pgfsetmiterjoin
\pgfsetlinewidth{0.100000\du}
\pgfsetbuttcap
\pgfsetmiterjoin
\pgfsetdash{}{0pt}
\definecolor{dialinecolor}{rgb}{1.000000, 1.000000, 1.000000}
\pgfsetfillcolor{dialinecolor}
\pgfpathellipse{\pgfpoint{56.244562\du}{30.122198\du}}{\pgfpoint{0.279442\du}{0\du}}{\pgfpoint{0\du}{0.279442\du}}
\pgfusepath{fill}
\definecolor{dialinecolor}{rgb}{0.000000, 0.000000, 0.000000}
\pgfsetstrokecolor{dialinecolor}
\pgfpathellipse{\pgfpoint{56.244562\du}{30.122198\du}}{\pgfpoint{0.279442\du}{0\du}}{\pgfpoint{0\du}{0.279442\du}}
\pgfusepath{stroke}
\pgfsetbuttcap
\pgfsetmiterjoin
\pgfsetdash{}{0pt}
\definecolor{dialinecolor}{rgb}{0.000000, 0.000000, 0.000000}
\pgfsetstrokecolor{dialinecolor}
\pgfpathellipse{\pgfpoint{56.244562\du}{30.122198\du}}{\pgfpoint{0.279442\du}{0\du}}{\pgfpoint{0\du}{0.279442\du}}
\pgfusepath{stroke}
\pgfsetlinewidth{0.100000\du}
\pgfsetdash{}{0pt}
\pgfsetdash{}{0pt}
\pgfsetbuttcap
\pgfsetmiterjoin
\pgfsetlinewidth{0.100000\du}
\pgfsetbuttcap
\pgfsetmiterjoin
\pgfsetdash{}{0pt}
\definecolor{dialinecolor}{rgb}{1.000000, 1.000000, 1.000000}
\pgfsetfillcolor{dialinecolor}
\pgfpathellipse{\pgfpoint{56.986106\du}{27.995711\du}}{\pgfpoint{0.279442\du}{0\du}}{\pgfpoint{0\du}{0.279442\du}}
\pgfusepath{fill}
\definecolor{dialinecolor}{rgb}{0.000000, 0.000000, 0.000000}
\pgfsetstrokecolor{dialinecolor}
\pgfpathellipse{\pgfpoint{56.986106\du}{27.995711\du}}{\pgfpoint{0.279442\du}{0\du}}{\pgfpoint{0\du}{0.279442\du}}
\pgfusepath{stroke}
\pgfsetbuttcap
\pgfsetmiterjoin
\pgfsetdash{}{0pt}
\definecolor{dialinecolor}{rgb}{0.000000, 0.000000, 0.000000}
\pgfsetstrokecolor{dialinecolor}
\pgfpathellipse{\pgfpoint{56.986106\du}{27.995711\du}}{\pgfpoint{0.279442\du}{0\du}}{\pgfpoint{0\du}{0.279442\du}}
\pgfusepath{stroke}
\pgfsetlinewidth{0.100000\du}
\pgfsetdash{}{0pt}
\pgfsetdash{}{0pt}
\pgfsetbuttcap
\pgfsetmiterjoin
\pgfsetlinewidth{0.100000\du}
\pgfsetbuttcap
\pgfsetmiterjoin
\pgfsetdash{}{0pt}
\definecolor{dialinecolor}{rgb}{1.000000, 1.000000, 1.000000}
\pgfsetfillcolor{dialinecolor}
\pgfpathellipse{\pgfpoint{56.609882\du}{25.732912\du}}{\pgfpoint{0.279442\du}{0\du}}{\pgfpoint{0\du}{0.279442\du}}
\pgfusepath{fill}
\definecolor{dialinecolor}{rgb}{0.000000, 0.000000, 0.000000}
\pgfsetstrokecolor{dialinecolor}
\pgfpathellipse{\pgfpoint{56.609882\du}{25.732912\du}}{\pgfpoint{0.279442\du}{0\du}}{\pgfpoint{0\du}{0.279442\du}}
\pgfusepath{stroke}
\pgfsetbuttcap
\pgfsetmiterjoin
\pgfsetdash{}{0pt}
\definecolor{dialinecolor}{rgb}{0.000000, 0.000000, 0.000000}
\pgfsetstrokecolor{dialinecolor}
\pgfpathellipse{\pgfpoint{56.609882\du}{25.732912\du}}{\pgfpoint{0.279442\du}{0\du}}{\pgfpoint{0\du}{0.279442\du}}
\pgfusepath{stroke}
\pgfsetlinewidth{0.100000\du}
\pgfsetdash{}{0pt}
\pgfsetdash{}{0pt}
\pgfsetbuttcap
\pgfsetmiterjoin
\pgfsetlinewidth{0.100000\du}
\pgfsetbuttcap
\pgfsetmiterjoin
\pgfsetdash{}{0pt}
\definecolor{dialinecolor}{rgb}{1.000000, 1.000000, 1.000000}
\pgfsetfillcolor{dialinecolor}
\pgfpathellipse{\pgfpoint{55.224939\du}{25.787437\du}}{\pgfpoint{0.279442\du}{0\du}}{\pgfpoint{0\du}{0.279442\du}}
\pgfusepath{fill}
\definecolor{dialinecolor}{rgb}{0.000000, 0.000000, 0.000000}
\pgfsetstrokecolor{dialinecolor}
\pgfpathellipse{\pgfpoint{55.224939\du}{25.787437\du}}{\pgfpoint{0.279442\du}{0\du}}{\pgfpoint{0\du}{0.279442\du}}
\pgfusepath{stroke}
\pgfsetbuttcap
\pgfsetmiterjoin
\pgfsetdash{}{0pt}
\definecolor{dialinecolor}{rgb}{0.000000, 0.000000, 0.000000}
\pgfsetstrokecolor{dialinecolor}
\pgfpathellipse{\pgfpoint{55.224939\du}{25.787437\du}}{\pgfpoint{0.279442\du}{0\du}}{\pgfpoint{0\du}{0.279442\du}}
\pgfusepath{stroke}
\pgfsetlinewidth{0.100000\du}
\pgfsetdash{}{0pt}
\pgfsetdash{}{0pt}
\pgfsetbuttcap
\pgfsetmiterjoin
\pgfsetlinewidth{0.100000\du}
\pgfsetbuttcap
\pgfsetmiterjoin
\pgfsetdash{}{0pt}
\definecolor{dialinecolor}{rgb}{1.000000, 1.000000, 1.000000}
\pgfsetfillcolor{dialinecolor}
\pgfpathellipse{\pgfpoint{56.266372\du}{24.833244\du}}{\pgfpoint{0.279442\du}{0\du}}{\pgfpoint{0\du}{0.279442\du}}
\pgfusepath{fill}
\definecolor{dialinecolor}{rgb}{0.000000, 0.000000, 0.000000}
\pgfsetstrokecolor{dialinecolor}
\pgfpathellipse{\pgfpoint{56.266372\du}{24.833244\du}}{\pgfpoint{0.279442\du}{0\du}}{\pgfpoint{0\du}{0.279442\du}}
\pgfusepath{stroke}
\pgfsetbuttcap
\pgfsetmiterjoin
\pgfsetdash{}{0pt}
\definecolor{dialinecolor}{rgb}{0.000000, 0.000000, 0.000000}
\pgfsetstrokecolor{dialinecolor}
\pgfpathellipse{\pgfpoint{56.266372\du}{24.833244\du}}{\pgfpoint{0.279442\du}{0\du}}{\pgfpoint{0\du}{0.279442\du}}
\pgfusepath{stroke}
\pgfsetlinewidth{0.100000\du}
\pgfsetdash{}{0pt}
\pgfsetdash{}{0pt}
\pgfsetbuttcap
\pgfsetmiterjoin
\pgfsetlinewidth{0.100000\du}
\pgfsetbuttcap
\pgfsetmiterjoin
\pgfsetdash{}{0pt}
\definecolor{dialinecolor}{rgb}{1.000000, 1.000000, 1.000000}
\pgfsetfillcolor{dialinecolor}
\pgfpathellipse{\pgfpoint{57.525907\du}{24.996820\du}}{\pgfpoint{0.279442\du}{0\du}}{\pgfpoint{0\du}{0.279442\du}}
\pgfusepath{fill}
\definecolor{dialinecolor}{rgb}{0.000000, 0.000000, 0.000000}
\pgfsetstrokecolor{dialinecolor}
\pgfpathellipse{\pgfpoint{57.525907\du}{24.996820\du}}{\pgfpoint{0.279442\du}{0\du}}{\pgfpoint{0\du}{0.279442\du}}
\pgfusepath{stroke}
\pgfsetbuttcap
\pgfsetmiterjoin
\pgfsetdash{}{0pt}
\definecolor{dialinecolor}{rgb}{0.000000, 0.000000, 0.000000}
\pgfsetstrokecolor{dialinecolor}
\pgfpathellipse{\pgfpoint{57.525907\du}{24.996820\du}}{\pgfpoint{0.279442\du}{0\du}}{\pgfpoint{0\du}{0.279442\du}}
\pgfusepath{stroke}
\pgfsetlinewidth{0.300000\du}
\pgfsetdash{{1.000000\du}{1.000000\du}}{0\du}
\pgfsetdash{{1.000000\du}{1.000000\du}}{0\du}
\pgfsetmiterjoin
\definecolor{dialinecolor}{rgb}{0.000000, 0.000000, 0.000000}
\pgfsetstrokecolor{dialinecolor}
\draw (48.428797\du,26.234814\du)--(48.428797\du,32.709859\du)--(55.626136\du,32.709859\du)--(55.626136\du,26.234814\du)--cycle;
% setfont left to latex
\definecolor{dialinecolor}{rgb}{0.000000, 0.000000, 0.000000}
\pgfsetstrokecolor{dialinecolor}
\node at (52.027467\du,29.712336\du){};
\end{tikzpicture}

  \caption{Schematische Datenstruktur}
  \label{fig:datenstruktur_grid}
\end{figure}
Bei der Erstellung und Nutzung des Arrays gibt es zwei Herangehensweisen, den vorhandenen Platz zu nutzen. Zum Einen ist es möglich, die Länge des Arrays exakt der Anzahl an Partikeln gleichzusetzen, wodurch nur der unbedingt benötigte Speicherplatz benutzt wird. Dies hätte allerdings zur Folge, dass in jedem Zeitschritt die Elemente des Arrays neu sortiert werden müssten, da die Anzahl an Partikeln innerhalb einer Zelle schwanken kann. Da sich gezeigt hat, dass bei der vorhandenen Grafikkarte die zur Verfügung stehende Leistung eine größere Rolle spielt als der Speicherplatz, wurde ein statischer Ansatz gewählt, bei dem beim Programmstart ein überdimensioniertes Array angelegt wird. Dazu wird jeder Zelle Platz für eine bestimmte Anzahl an Partikeln eingeräumt, unabhängig davon ob dieser Platz benötigt wird. Auf diese Weise entfällt der Sortiervorgang in jedem Zeitschritt, da der Speicherplatz für die einzelnen Partikel bereits vorhanden ist, und der Aufwand, das Grid stets im aktuellen Zustand zu behalten, reduziert sich auf eine atomic function, um die Indices ohne Seiteneffekte zu speichern. Abbildung~\ref{fig:datenstruktur_eintraege} zeigt ein Beispiel für die vierte Dimension dieses Grids, wobei der erste Eintrag die Anzahl an validen Folgeeinträgen speichert, um die Zahl der Zugriffe zu minimieren.
\begin{figure}
  \centering
    % Graphic for TeX using PGF
% Title: D:\workspace\uos_pa_sph\doku\images\datenstruktur_eintraege.dia
% Creator: Dia v0.97.2
% CreationDate: Wed Sep 11 14:25:16 2013
% For: Nikki
% \usepackage{tikz}
% The following commands are not supported in PSTricks at present
% We define them conditionally, so when they are implemented,
% this pgf file will use them.
\ifx\du\undefined
  \newlength{\du}
\fi
\setlength{\du}{15\unitlength}
\begin{tikzpicture}
\pgftransformxscale{1.000000}
\pgftransformyscale{-1.000000}
\definecolor{dialinecolor}{rgb}{0.000000, 0.000000, 0.000000}
\pgfsetstrokecolor{dialinecolor}
\definecolor{dialinecolor}{rgb}{1.000000, 1.000000, 1.000000}
\pgfsetfillcolor{dialinecolor}
\definecolor{dialinecolor}{rgb}{1.000000, 1.000000, 1.000000}
\pgfsetfillcolor{dialinecolor}
\fill (46.050000\du,22.150000\du)--(46.050000\du,24.150000\du)--(48.050000\du,24.150000\du)--(48.050000\du,22.150000\du)--cycle;
\pgfsetlinewidth{0.200000\du}
\pgfsetdash{}{0pt}
\pgfsetdash{}{0pt}
\pgfsetmiterjoin
\definecolor{dialinecolor}{rgb}{0.000000, 0.000000, 0.000000}
\pgfsetstrokecolor{dialinecolor}
\draw (46.050000\du,22.150000\du)--(46.050000\du,24.150000\du)--(48.050000\du,24.150000\du)--(48.050000\du,22.150000\du)--cycle;
% setfont left to latex
\definecolor{dialinecolor}{rgb}{0.000000, 0.000000, 0.000000}
\pgfsetstrokecolor{dialinecolor}
\node at (47.050000\du,23.337500\du){3};
\definecolor{dialinecolor}{rgb}{1.000000, 1.000000, 1.000000}
\pgfsetfillcolor{dialinecolor}
\fill (48.110000\du,22.200000\du)--(48.110000\du,24.100000\du)--(50.110000\du,24.100000\du)--(50.110000\du,22.200000\du)--cycle;
\pgfsetlinewidth{0.100000\du}
\pgfsetdash{}{0pt}
\pgfsetdash{}{0pt}
\pgfsetmiterjoin
\definecolor{dialinecolor}{rgb}{0.000000, 0.000000, 0.000000}
\pgfsetstrokecolor{dialinecolor}
\draw (48.110000\du,22.200000\du)--(48.110000\du,24.100000\du)--(50.110000\du,24.100000\du)--(50.110000\du,22.200000\du)--cycle;
% setfont left to latex
\definecolor{dialinecolor}{rgb}{0.000000, 0.000000, 0.000000}
\pgfsetstrokecolor{dialinecolor}
\node at (49.110000\du,23.337500\du){47};
\definecolor{dialinecolor}{rgb}{1.000000, 1.000000, 1.000000}
\pgfsetfillcolor{dialinecolor}
\fill (50.161077\du,22.198863\du)--(50.161077\du,24.098863\du)--(52.161077\du,24.098863\du)--(52.161077\du,22.198863\du)--cycle;
\pgfsetlinewidth{0.100000\du}
\pgfsetdash{}{0pt}
\pgfsetdash{}{0pt}
\pgfsetmiterjoin
\definecolor{dialinecolor}{rgb}{0.000000, 0.000000, 0.000000}
\pgfsetstrokecolor{dialinecolor}
\draw (50.161077\du,22.198863\du)--(50.161077\du,24.098863\du)--(52.161077\du,24.098863\du)--(52.161077\du,22.198863\du)--cycle;
% setfont left to latex
\definecolor{dialinecolor}{rgb}{0.000000, 0.000000, 0.000000}
\pgfsetstrokecolor{dialinecolor}
\node at (51.161077\du,23.336363\du){53};
\definecolor{dialinecolor}{rgb}{1.000000, 1.000000, 1.000000}
\pgfsetfillcolor{dialinecolor}
\fill (52.212855\du,22.198863\du)--(52.212855\du,24.098863\du)--(54.212855\du,24.098863\du)--(54.212855\du,22.198863\du)--cycle;
\pgfsetlinewidth{0.100000\du}
\pgfsetdash{}{0pt}
\pgfsetdash{}{0pt}
\pgfsetmiterjoin
\definecolor{dialinecolor}{rgb}{0.000000, 0.000000, 0.000000}
\pgfsetstrokecolor{dialinecolor}
\draw (52.212855\du,22.198863\du)--(52.212855\du,24.098863\du)--(54.212855\du,24.098863\du)--(54.212855\du,22.198863\du)--cycle;
% setfont left to latex
\definecolor{dialinecolor}{rgb}{0.000000, 0.000000, 0.000000}
\pgfsetstrokecolor{dialinecolor}
\node at (53.212855\du,23.336363\du){26};
\definecolor{dialinecolor}{rgb}{1.000000, 1.000000, 1.000000}
\pgfsetfillcolor{dialinecolor}
\fill (54.264633\du,22.198863\du)--(54.264633\du,24.098863\du)--(56.264633\du,24.098863\du)--(56.264633\du,22.198863\du)--cycle;
\pgfsetlinewidth{0.100000\du}
\pgfsetdash{}{0pt}
\pgfsetdash{}{0pt}
\pgfsetmiterjoin
\definecolor{dialinecolor}{rgb}{0.000000, 0.000000, 0.000000}
\pgfsetstrokecolor{dialinecolor}
\draw (54.264633\du,22.198863\du)--(54.264633\du,24.098863\du)--(56.264633\du,24.098863\du)--(56.264633\du,22.198863\du)--cycle;
% setfont left to latex
\definecolor{dialinecolor}{rgb}{0.000000, 0.000000, 0.000000}
\pgfsetstrokecolor{dialinecolor}
\node at (55.264633\du,23.336363\du){62};
\definecolor{dialinecolor}{rgb}{1.000000, 1.000000, 1.000000}
\pgfsetfillcolor{dialinecolor}
\fill (56.326923\du,22.198863\du)--(56.326923\du,24.098863\du)--(58.326923\du,24.098863\du)--(58.326923\du,22.198863\du)--cycle;
\pgfsetlinewidth{0.100000\du}
\pgfsetdash{}{0pt}
\pgfsetdash{}{0pt}
\pgfsetmiterjoin
\definecolor{dialinecolor}{rgb}{0.000000, 0.000000, 0.000000}
\pgfsetstrokecolor{dialinecolor}
\draw (56.326923\du,22.198863\du)--(56.326923\du,24.098863\du)--(58.326923\du,24.098863\du)--(58.326923\du,22.198863\du)--cycle;
% setfont left to latex
\definecolor{dialinecolor}{rgb}{0.000000, 0.000000, 0.000000}
\pgfsetstrokecolor{dialinecolor}
\node at (57.326923\du,23.336363\du){...};
\end{tikzpicture}

  \caption{Beispielhafte vierte Dimension der Datenstruktur}
  \label{fig:datenstruktur_eintraege}
\end{figure}

\begin{minipage}{\linewidth}
\begin{lstlisting}[caption=Iteration über benachbarte Partikel, label=lst:grid_for]
int4 gridPos = convert_int4((BUFFER_SIZE_SIDE - 1) * (body_Pos[id] + (float4)1) / 2);
for (int l = max(gridPos.x - OFFSET, 0); l <= min(gridPos.x + OFFSET, BUFFER_SIZE_SIDE - 1) ; l++) {
	for (int j = max(gridPos.y - OFFSET, 0); j <= min(gridPos.y + OFFSET, BUFFER_SIZE_SIDE - 1) ; j++) {
		for (int k = max(gridPos.z - OFFSET, 0); k <= min(gridPos.z + OFFSET, BUFFER_SIZE_SIDE - 1) ; k++) {
			int cnt_ind = BUFFER_SIZE_DEPTH * (l + BUFFER_SIZE_SIDE * j + BUFFER_SIZE_SIDE * BUFFER_SIZE_SIDE * k);
			uint cnt = data[cnt_ind];			
			for (int o = 1; o <= cnt; o++) {
				int i = data[cnt_ind + o];
				...
			}
		}
	}
}
\end{lstlisting}
\end{minipage}

Mithilfe dieses Arrays lässt sich nun, zu sehen in Code~\ref{lst:grid_for}, über alle benachbarten Partikel iterieren. Die Konstante \texttt{BUFFER\_SIZE\_SIDE} gibt dabei die Anzahl an Zellen für jede der ersten drei Dimensionen an und \texttt{BUFFER\_SIZE\_DEPTH} der reservierte Platz für Partikel innerhalb einer Zelle, also die vierte Dimension, wobei mithilfe von \texttt{OFFSET} der gewünschte Radius um die betrachtete Zelle festgelegt wird.


Alle diese Konstanten können unabhängig von der Partikelzahl gewählt werden, mit Ausnahme von \texttt{BUFFER\_SIZE\_DEPTH}. Ist diese Konstante zu gering, könnte der verfügbare Speicherplatz bei zu vielen Partikeln innerhalb einer Zelle überlaufen. Sollte dieser Fall eintreten, werden alle nachfolgenden Partikel dieser Zelle nicht gespeichert, und werden bei den folgenden Nachbarschaftsberechnungen dementsprechend solange nicht berücksichtigt, bis sie wieder einen Platz finden.\\
Als \textit{guter} Wert für \texttt{BUFFER\_SIZE\_DEPTH} hat sich $\sqrt(n)$ bewiesen, wobei auch zu geringe Werte, die einen Überlauf induzieren, vor allem bei einer hohen Partikelanzahl erst spät zu Veränderungen in der Simulation führen, da nur einzelne Partikel davon betroffen sind.\\
Wenn man nun davon ausgeht, dass $\sqrt(n)$ ein guter Wert für \texttt{BUFFER\_SIZE\_DEPTH} sei, dann hat die Schleife, die über die benachbarten Partikel iteriert, eine Laufzeit von $\mathcal O(\sqrt(n))$ im worst-case. Dadurch verbessert sich die asymptotische Laufzeit des gesamten Algorithmus' auf $\mathcal O(n \cdot \sqrt(n))$.
\subsubsection{Speicherung der Partikelindices in einer dreidimensionalen Textur}
Dieser Ansatz wurde nur diskutiert, da die vorhergehende Herangehensweise bereits zufriedenstellende Ergebnisse lieferte.\\
Die Grundidee hierbei ist, für die Partikelindices statt einem vierdimensionalen Array eine dreidimensionale Textur zu verwenden, wobei die eigentlichen Indices - vorher abgelegt in der vierten Dimension - nun in den vier Farbkanälen der Textur codiert werden müssen. Da diese Kanäle nur begrenzten Platz zur Verfügung stellen, müsste die Textur tendenziell höher aufgelöst sein als das Grid beim vorigen Ansatz, um einen Überlauf zu vermeiden. Dadurch würde zweifellos ein erhöhter Aufwand entstehen, der aber, so die Idee, dennoch zu einer schnelleren Laufzeit führen könnte aufgrund des internen Cachings durch die Grafikkarte. Da sich die Partikel allerdings stets bewegen, müssten die gespeicherten Indices in jedem Zeitschritt modifiziert werden, was effizientem Caching entgegenwirkt. Die Frage, ob dieser Ansatz effizienter als der vorhergehende ist, lässt sich also auf die Frage, ob das Caching bei erhöhtem Lesezugriff und gleichbleibendem Schreibzugriff zu einer schnelleren Laufzeit führt als bei herkömmlichen globalen Speicher, abbilden. Da die Antwort unklar und somit der Gewinn durch diesen Ansatz in Frage gestellt wurde, wurde er verworfen.
\subsubsection{Zufallsbasierte Selektion einer Referenzmenge}
Bei diesem Ansatz sollte die Laufzeit verringert werden, indem die zur Kräfteberechnung herangezogenen Partikel einer zufällig ausgewählten und echten Teilmenge aller Partikel entstammen. Bei einer entsprechend gewählten Größe dieser Teilmenge konnte diese außerdem im lokalen Speicher abgelegt werden, um die Laufzeit weiter zu reduzieren.\\
Es hat sich allerdings gezeigt, dass die Genauigkeit der Simulation durch diesen Ansatz signifikant verschlechtert wurde, sodass er - trotz überragend schneller Laufzeit - wieder verworfen wurde.
\subsection{OpenCL und Parallelisierung}
Wie bei einem herkömmlichen n-Body-System können auch bei dieser Simulation die Zustandsberechnungen der einzelnen Partikel unabhängig von den jeweils anderen durchgeführt werden. Da die Berechnungen allerdings von den Zuständen der anderen Partikel abhängen, welche wiederum ebenfalls einen neuen Zustand errechnen, muss mithilfe von Synchronisation sichergestellt werden, dass keine ungewünschten Seiteneffekte durch teilweise modifizierte Daten entstehen.\\
Da sich die Partikel aufgrund ihrer stetigen Bewegung und damit wechselnden Abhängigkeiten nicht sinnvoll mithilfe von Work Groups aufteilen ließen, wurden die Teilschritte der Berechnung in einzelne Kernel ausgelagert, die auf diese Weise global synchronisiert werden können.\\
Der grundlegende Ablauf ist hierbei in Abbildung~\ref{fig:simulation_kernelablauf} zu erkennen. Für die Berechnung eines neuen Geschwindigkeitsvektors \texttt{V} werden sowohl Druck $\rho$  als auch Dichte \texttt{P} des Partikels benötigt, welche wiederum voneinander und von anderen Partikeln abhängen. Aus diesem Grunde wird zuerst $\rho$ für alle Partikel berechnet, damit, nach einem Synchronisationspunkt, \texttt{P} mithilfe von $\rho$ berechnet werden kann. \\
Der Synchronisationspunkt nach der Berechnung der neuen Geschwindigkeit hat keine technische Grundlage, da die neue Position des Partikels nur von den eigenen Attributen ahängig ist, wurde aber zwecks Profiling eingeführt.
\begin{figure}
  \centering
    % Graphic for TeX using PGF
% Title: D:\workspace\uos_pa_sph\doku\images\simulation_kernelablauf.dia
% Creator: Dia v0.97.2
% CreationDate: Fri Sep 13 15:07:09 2013
% For: Nikki
% \usepackage{tikz}
% The following commands are not supported in PSTricks at present
% We define them conditionally, so when they are implemented,
% this pgf file will use them.
\ifx\du\undefined
  \newlength{\du}
\fi
\setlength{\du}{15\unitlength}
\begin{tikzpicture}
\pgftransformxscale{0.600000}
\pgftransformyscale{-0.600000}
\definecolor{dialinecolor}{rgb}{0.000000, 0.000000, 0.000000}
\pgfsetstrokecolor{dialinecolor}
\definecolor{dialinecolor}{rgb}{1.000000, 1.000000, 1.000000}
\pgfsetfillcolor{dialinecolor}
\definecolor{dialinecolor}{rgb}{1.000000, 1.000000, 1.000000}
\pgfsetfillcolor{dialinecolor}
\fill (18.462586\du,22.197519\du)--(18.462586\du,31.589660\du)--(41.526766\du,31.589660\du)--(41.526766\du,22.197519\du)--cycle;
\pgfsetlinewidth{0.100000\du}
\pgfsetdash{}{0pt}
\pgfsetdash{}{0pt}
\pgfsetmiterjoin
\definecolor{dialinecolor}{rgb}{0.000000, 0.000000, 0.000000}
\pgfsetstrokecolor{dialinecolor}
\draw (18.462586\du,22.197519\du)--(18.462586\du,31.589660\du)--(41.526766\du,31.589660\du)--(41.526766\du,22.197519\du)--cycle;
% setfont left to latex
\definecolor{dialinecolor}{rgb}{0.000000, 0.000000, 0.000000}
\pgfsetstrokecolor{dialinecolor}
\node at (29.994676\du,27.133589\du){};
\pgfsetlinewidth{0.100000\du}
\pgfsetdash{}{0pt}
\pgfsetdash{}{0pt}
\pgfsetbuttcap
{
\definecolor{dialinecolor}{rgb}{0.000000, 0.000000, 0.000000}
\pgfsetfillcolor{dialinecolor}
% was here!!!
\pgfsetarrowsend{latex}
\definecolor{dialinecolor}{rgb}{0.000000, 0.000000, 0.000000}
\pgfsetstrokecolor{dialinecolor}
\draw (47.125778\du,24.028174\du)--(47.100000\du,26.300000\du);
}
\pgfsetlinewidth{0.100000\du}
\pgfsetdash{}{0pt}
\pgfsetdash{}{0pt}
\pgfsetbuttcap
{
\definecolor{dialinecolor}{rgb}{0.000000, 0.000000, 0.000000}
\pgfsetfillcolor{dialinecolor}
% was here!!!
\pgfsetarrowsend{latex}
\definecolor{dialinecolor}{rgb}{0.000000, 0.000000, 0.000000}
\pgfsetstrokecolor{dialinecolor}
\draw (51.043113\du,24.057252\du)--(47.100000\du,26.300000\du);
}
\pgfsetlinewidth{0.100000\du}
\pgfsetdash{}{0pt}
\pgfsetdash{}{0pt}
\pgfsetbuttcap
{
\definecolor{dialinecolor}{rgb}{0.000000, 0.000000, 0.000000}
\pgfsetfillcolor{dialinecolor}
% was here!!!
\pgfsetarrowsend{latex}
\definecolor{dialinecolor}{rgb}{0.000000, 0.000000, 0.000000}
\pgfsetstrokecolor{dialinecolor}
\draw (60.013113\du,24.057252\du)--(47.100000\du,26.300000\du);
}
\pgfsetlinewidth{0.100000\du}
\pgfsetdash{}{0pt}
\pgfsetdash{}{0pt}
\pgfsetbuttcap
{
\definecolor{dialinecolor}{rgb}{0.000000, 0.000000, 0.000000}
\pgfsetfillcolor{dialinecolor}
% was here!!!
\pgfsetarrowsend{latex}
\definecolor{dialinecolor}{rgb}{0.000000, 0.000000, 0.000000}
\pgfsetstrokecolor{dialinecolor}
\draw (51.043113\du,24.057252\du)--(51.000000\du,26.250000\du);
}
\pgfsetlinewidth{0.100000\du}
\pgfsetdash{}{0pt}
\pgfsetdash{}{0pt}
\pgfsetbuttcap
{
\definecolor{dialinecolor}{rgb}{0.000000, 0.000000, 0.000000}
\pgfsetfillcolor{dialinecolor}
% was here!!!
\pgfsetarrowsend{latex}
\definecolor{dialinecolor}{rgb}{0.000000, 0.000000, 0.000000}
\pgfsetstrokecolor{dialinecolor}
\draw (55.003113\du,24.107252\du)--(51.000000\du,26.250000\du);
}
\pgfsetlinewidth{0.100000\du}
\pgfsetdash{}{0pt}
\pgfsetdash{}{0pt}
\pgfsetbuttcap
{
\definecolor{dialinecolor}{rgb}{0.000000, 0.000000, 0.000000}
\pgfsetfillcolor{dialinecolor}
% was here!!!
\pgfsetarrowsend{latex}
\definecolor{dialinecolor}{rgb}{0.000000, 0.000000, 0.000000}
\pgfsetstrokecolor{dialinecolor}
\draw (55.003113\du,24.107252\du)--(55.000000\du,26.300000\du);
}
\pgfsetlinewidth{0.100000\du}
\pgfsetdash{}{0pt}
\pgfsetdash{}{0pt}
\pgfsetbuttcap
{
\definecolor{dialinecolor}{rgb}{0.000000, 0.000000, 0.000000}
\pgfsetfillcolor{dialinecolor}
% was here!!!
\pgfsetarrowsend{latex}
\definecolor{dialinecolor}{rgb}{0.000000, 0.000000, 0.000000}
\pgfsetstrokecolor{dialinecolor}
\draw (47.125778\du,24.028174\du)--(55.000000\du,26.350000\du);
}
\pgfsetlinewidth{0.100000\du}
\pgfsetdash{}{0pt}
\pgfsetdash{}{0pt}
\pgfsetbuttcap
{
\definecolor{dialinecolor}{rgb}{0.000000, 0.000000, 0.000000}
\pgfsetfillcolor{dialinecolor}
% was here!!!
\pgfsetarrowsend{latex}
\definecolor{dialinecolor}{rgb}{0.000000, 0.000000, 0.000000}
\pgfsetstrokecolor{dialinecolor}
\draw (60.013113\du,24.057252\du)--(59.950000\du,26.300000\du);
}
\pgfsetlinewidth{0.100000\du}
\pgfsetdash{}{0pt}
\pgfsetdash{}{0pt}
\pgfsetbuttcap
{
\definecolor{dialinecolor}{rgb}{0.000000, 0.000000, 0.000000}
\pgfsetfillcolor{dialinecolor}
% was here!!!
\pgfsetarrowsend{latex}
\definecolor{dialinecolor}{rgb}{0.000000, 0.000000, 0.000000}
\pgfsetstrokecolor{dialinecolor}
\draw (51.043113\du,24.057252\du)--(59.950000\du,26.350000\du);
}
\definecolor{dialinecolor}{rgb}{1.000000, 1.000000, 1.000000}
\pgfsetfillcolor{dialinecolor}
\pgfpathellipse{\pgfpoint{47.521160\du}{31.381493\du}}{\pgfpoint{1.419866\du}{0\du}}{\pgfpoint{0\du}{1.403171\du}}
\pgfusepath{fill}
\pgfsetlinewidth{0.100000\du}
\pgfsetdash{}{0pt}
\pgfsetdash{}{0pt}
\pgfsetmiterjoin
\definecolor{dialinecolor}{rgb}{0.000000, 0.000000, 0.000000}
\pgfsetstrokecolor{dialinecolor}
\pgfpathellipse{\pgfpoint{47.521160\du}{31.381493\du}}{\pgfpoint{1.419866\du}{0\du}}{\pgfpoint{0\du}{1.403171\du}}
\pgfusepath{stroke}
% setfont left to latex
\definecolor{dialinecolor}{rgb}{0.000000, 0.000000, 0.000000}
\pgfsetstrokecolor{dialinecolor}
\node at (47.521160\du,31.513993\du){Rho1};
\definecolor{dialinecolor}{rgb}{1.000000, 1.000000, 1.000000}
\pgfsetfillcolor{dialinecolor}
\pgfpathellipse{\pgfpoint{51.029866\du}{31.453171\du}}{\pgfpoint{1.419866\du}{0\du}}{\pgfpoint{0\du}{1.403171\du}}
\pgfusepath{fill}
\pgfsetlinewidth{0.100000\du}
\pgfsetdash{}{0pt}
\pgfsetdash{}{0pt}
\pgfsetmiterjoin
\definecolor{dialinecolor}{rgb}{0.000000, 0.000000, 0.000000}
\pgfsetstrokecolor{dialinecolor}
\pgfpathellipse{\pgfpoint{51.029866\du}{31.453171\du}}{\pgfpoint{1.419866\du}{0\du}}{\pgfpoint{0\du}{1.403171\du}}
\pgfusepath{stroke}
% setfont left to latex
\definecolor{dialinecolor}{rgb}{0.000000, 0.000000, 0.000000}
\pgfsetstrokecolor{dialinecolor}
\node at (51.029866\du,31.585671\du){Rho2};
\definecolor{dialinecolor}{rgb}{1.000000, 1.000000, 1.000000}
\pgfsetfillcolor{dialinecolor}
\pgfpathellipse{\pgfpoint{54.939866\du}{31.453171\du}}{\pgfpoint{1.419866\du}{0\du}}{\pgfpoint{0\du}{1.403171\du}}
\pgfusepath{fill}
\pgfsetlinewidth{0.100000\du}
\pgfsetdash{}{0pt}
\pgfsetdash{}{0pt}
\pgfsetmiterjoin
\definecolor{dialinecolor}{rgb}{0.000000, 0.000000, 0.000000}
\pgfsetstrokecolor{dialinecolor}
\pgfpathellipse{\pgfpoint{54.939866\du}{31.453171\du}}{\pgfpoint{1.419866\du}{0\du}}{\pgfpoint{0\du}{1.403171\du}}
\pgfusepath{stroke}
% setfont left to latex
\definecolor{dialinecolor}{rgb}{0.000000, 0.000000, 0.000000}
\pgfsetstrokecolor{dialinecolor}
\node at (54.939866\du,31.585671\du){Rho3};
\definecolor{dialinecolor}{rgb}{1.000000, 1.000000, 1.000000}
\pgfsetfillcolor{dialinecolor}
\pgfpathellipse{\pgfpoint{59.949866\du}{31.453171\du}}{\pgfpoint{1.419866\du}{0\du}}{\pgfpoint{0\du}{1.403171\du}}
\pgfusepath{fill}
\pgfsetlinewidth{0.100000\du}
\pgfsetdash{}{0pt}
\pgfsetdash{}{0pt}
\pgfsetmiterjoin
\definecolor{dialinecolor}{rgb}{0.000000, 0.000000, 0.000000}
\pgfsetstrokecolor{dialinecolor}
\pgfpathellipse{\pgfpoint{59.949866\du}{31.453171\du}}{\pgfpoint{1.419866\du}{0\du}}{\pgfpoint{0\du}{1.403171\du}}
\pgfusepath{stroke}
% setfont left to latex
\definecolor{dialinecolor}{rgb}{0.000000, 0.000000, 0.000000}
\pgfsetstrokecolor{dialinecolor}
\node at (59.949866\du,31.585671\du){Rhon};
\pgfsetlinewidth{0.100000\du}
\pgfsetdash{}{0pt}
\pgfsetdash{}{0pt}
\pgfsetbuttcap
{
\definecolor{dialinecolor}{rgb}{0.000000, 0.000000, 0.000000}
\pgfsetfillcolor{dialinecolor}
% was here!!!
\pgfsetarrowsend{latex}
\definecolor{dialinecolor}{rgb}{0.000000, 0.000000, 0.000000}
\pgfsetstrokecolor{dialinecolor}
\draw (47.500000\du,28.400000\du)--(47.521160\du,29.978322\du);
}
\pgfsetlinewidth{0.100000\du}
\pgfsetdash{}{0pt}
\pgfsetdash{}{0pt}
\pgfsetbuttcap
{
\definecolor{dialinecolor}{rgb}{0.000000, 0.000000, 0.000000}
\pgfsetfillcolor{dialinecolor}
% was here!!!
\pgfsetarrowsend{latex}
\definecolor{dialinecolor}{rgb}{0.000000, 0.000000, 0.000000}
\pgfsetstrokecolor{dialinecolor}
\draw (51.000000\du,28.350000\du)--(51.029866\du,30.050000\du);
}
\pgfsetlinewidth{0.100000\du}
\pgfsetdash{}{0pt}
\pgfsetdash{}{0pt}
\pgfsetbuttcap
{
\definecolor{dialinecolor}{rgb}{0.000000, 0.000000, 0.000000}
\pgfsetfillcolor{dialinecolor}
% was here!!!
\pgfsetarrowsend{latex}
\definecolor{dialinecolor}{rgb}{0.000000, 0.000000, 0.000000}
\pgfsetstrokecolor{dialinecolor}
\draw (55.000000\du,28.400000\du)--(54.939866\du,30.050000\du);
}
\pgfsetlinewidth{0.100000\du}
\pgfsetdash{}{0pt}
\pgfsetdash{}{0pt}
\pgfsetbuttcap
{
\definecolor{dialinecolor}{rgb}{0.000000, 0.000000, 0.000000}
\pgfsetfillcolor{dialinecolor}
% was here!!!
\pgfsetarrowsend{latex}
\definecolor{dialinecolor}{rgb}{0.000000, 0.000000, 0.000000}
\pgfsetstrokecolor{dialinecolor}
\draw (60.050000\du,28.350000\du)--(59.949866\du,30.050000\du);
}
\definecolor{dialinecolor}{rgb}{1.000000, 1.000000, 1.000000}
\pgfsetfillcolor{dialinecolor}
\fill (43.980402\du,26.300000\du)--(43.980402\du,28.500000\du)--(63.009714\du,28.500000\du)--(63.009714\du,26.300000\du)--cycle;
\pgfsetlinewidth{0.100000\du}
\pgfsetdash{}{0pt}
\pgfsetdash{}{0pt}
\pgfsetmiterjoin
\definecolor{dialinecolor}{rgb}{0.000000, 0.000000, 0.000000}
\pgfsetstrokecolor{dialinecolor}
\draw (43.980402\du,26.300000\du)--(43.980402\du,28.500000\du)--(63.009714\du,28.500000\du)--(63.009714\du,26.300000\du)--cycle;
% setfont left to latex
\definecolor{dialinecolor}{rgb}{0.000000, 0.000000, 0.000000}
\pgfsetstrokecolor{dialinecolor}
\node at (53.495058\du,27.640000\du){Berechne Dichte (Rho)};
\pgfsetlinewidth{0.100000\du}
\pgfsetdash{}{0pt}
\pgfsetdash{}{0pt}
\pgfsetbuttcap
{
\definecolor{dialinecolor}{rgb}{0.000000, 0.000000, 0.000000}
\pgfsetfillcolor{dialinecolor}
% was here!!!
\pgfsetarrowsend{latex}
\definecolor{dialinecolor}{rgb}{0.000000, 0.000000, 0.000000}
\pgfsetstrokecolor{dialinecolor}
\draw (47.235778\du,32.712100\du)--(47.210000\du,34.983925\du);
}
\pgfsetlinewidth{0.100000\du}
\pgfsetdash{}{0pt}
\pgfsetdash{}{0pt}
\pgfsetbuttcap
{
\definecolor{dialinecolor}{rgb}{0.000000, 0.000000, 0.000000}
\pgfsetfillcolor{dialinecolor}
% was here!!!
\pgfsetarrowsend{latex}
\definecolor{dialinecolor}{rgb}{0.000000, 0.000000, 0.000000}
\pgfsetstrokecolor{dialinecolor}
\draw (51.153113\du,32.741178\du)--(47.210000\du,34.983925\du);
}
\pgfsetlinewidth{0.100000\du}
\pgfsetdash{}{0pt}
\pgfsetdash{}{0pt}
\pgfsetbuttcap
{
\definecolor{dialinecolor}{rgb}{0.000000, 0.000000, 0.000000}
\pgfsetfillcolor{dialinecolor}
% was here!!!
\pgfsetarrowsend{latex}
\definecolor{dialinecolor}{rgb}{0.000000, 0.000000, 0.000000}
\pgfsetstrokecolor{dialinecolor}
\draw (60.123113\du,32.741178\du)--(47.210000\du,34.983925\du);
}
\pgfsetlinewidth{0.100000\du}
\pgfsetdash{}{0pt}
\pgfsetdash{}{0pt}
\pgfsetbuttcap
{
\definecolor{dialinecolor}{rgb}{0.000000, 0.000000, 0.000000}
\pgfsetfillcolor{dialinecolor}
% was here!!!
\pgfsetarrowsend{latex}
\definecolor{dialinecolor}{rgb}{0.000000, 0.000000, 0.000000}
\pgfsetstrokecolor{dialinecolor}
\draw (51.153113\du,32.741178\du)--(51.110000\du,34.933925\du);
}
\pgfsetlinewidth{0.100000\du}
\pgfsetdash{}{0pt}
\pgfsetdash{}{0pt}
\pgfsetbuttcap
{
\definecolor{dialinecolor}{rgb}{0.000000, 0.000000, 0.000000}
\pgfsetfillcolor{dialinecolor}
% was here!!!
\pgfsetarrowsend{latex}
\definecolor{dialinecolor}{rgb}{0.000000, 0.000000, 0.000000}
\pgfsetstrokecolor{dialinecolor}
\draw (55.113113\du,32.791178\du)--(51.110000\du,34.933925\du);
}
\pgfsetlinewidth{0.100000\du}
\pgfsetdash{}{0pt}
\pgfsetdash{}{0pt}
\pgfsetbuttcap
{
\definecolor{dialinecolor}{rgb}{0.000000, 0.000000, 0.000000}
\pgfsetfillcolor{dialinecolor}
% was here!!!
\pgfsetarrowsend{latex}
\definecolor{dialinecolor}{rgb}{0.000000, 0.000000, 0.000000}
\pgfsetstrokecolor{dialinecolor}
\draw (55.113113\du,32.791178\du)--(55.110000\du,34.983925\du);
}
\pgfsetlinewidth{0.100000\du}
\pgfsetdash{}{0pt}
\pgfsetdash{}{0pt}
\pgfsetbuttcap
{
\definecolor{dialinecolor}{rgb}{0.000000, 0.000000, 0.000000}
\pgfsetfillcolor{dialinecolor}
% was here!!!
\pgfsetarrowsend{latex}
\definecolor{dialinecolor}{rgb}{0.000000, 0.000000, 0.000000}
\pgfsetstrokecolor{dialinecolor}
\draw (47.235778\du,32.712100\du)--(55.110000\du,35.033925\du);
}
\pgfsetlinewidth{0.100000\du}
\pgfsetdash{}{0pt}
\pgfsetdash{}{0pt}
\pgfsetbuttcap
{
\definecolor{dialinecolor}{rgb}{0.000000, 0.000000, 0.000000}
\pgfsetfillcolor{dialinecolor}
% was here!!!
\pgfsetarrowsend{latex}
\definecolor{dialinecolor}{rgb}{0.000000, 0.000000, 0.000000}
\pgfsetstrokecolor{dialinecolor}
\draw (60.123113\du,32.741178\du)--(60.060000\du,34.983925\du);
}
\pgfsetlinewidth{0.100000\du}
\pgfsetdash{}{0pt}
\pgfsetdash{}{0pt}
\pgfsetbuttcap
{
\definecolor{dialinecolor}{rgb}{0.000000, 0.000000, 0.000000}
\pgfsetfillcolor{dialinecolor}
% was here!!!
\pgfsetarrowsend{latex}
\definecolor{dialinecolor}{rgb}{0.000000, 0.000000, 0.000000}
\pgfsetstrokecolor{dialinecolor}
\draw (51.153113\du,32.741178\du)--(60.060000\du,35.033925\du);
}
\definecolor{dialinecolor}{rgb}{1.000000, 1.000000, 1.000000}
\pgfsetfillcolor{dialinecolor}
\fill (43.980402\du,34.933925\du)--(43.980402\du,37.200000\du)--(63.009714\du,37.200000\du)--(63.009714\du,34.933925\du)--cycle;
\pgfsetlinewidth{0.100000\du}
\pgfsetdash{}{0pt}
\pgfsetdash{}{0pt}
\pgfsetmiterjoin
\definecolor{dialinecolor}{rgb}{0.000000, 0.000000, 0.000000}
\pgfsetstrokecolor{dialinecolor}
\draw (43.980402\du,34.933925\du)--(43.980402\du,37.200000\du)--(63.009714\du,37.200000\du)--(63.009714\du,34.933925\du)--cycle;
% setfont left to latex
\definecolor{dialinecolor}{rgb}{0.000000, 0.000000, 0.000000}
\pgfsetstrokecolor{dialinecolor}
\node at (53.495058\du,36.306963\du){Berechne Druck (P)};
\pgfsetlinewidth{0.300000\du}
\pgfsetdash{{1.000000\du}{1.000000\du}}{0\du}
\pgfsetdash{{1.000000\du}{1.000000\du}}{0\du}
\pgfsetbuttcap
{
\definecolor{dialinecolor}{rgb}{0.000000, 0.000000, 0.000000}
\pgfsetfillcolor{dialinecolor}
% was here!!!
\definecolor{dialinecolor}{rgb}{0.000000, 0.000000, 0.000000}
\pgfsetstrokecolor{dialinecolor}
\draw (44.300000\du,33.800000\du)--(63.250000\du,33.750000\du);
}
\definecolor{dialinecolor}{rgb}{1.000000, 1.000000, 1.000000}
\pgfsetfillcolor{dialinecolor}
\pgfpathellipse{\pgfpoint{47.579866\du}{40.184346\du}}{\pgfpoint{1.419866\du}{0\du}}{\pgfpoint{0\du}{1.403171\du}}
\pgfusepath{fill}
\pgfsetlinewidth{0.100000\du}
\pgfsetdash{}{0pt}
\pgfsetdash{}{0pt}
\pgfsetmiterjoin
\definecolor{dialinecolor}{rgb}{0.000000, 0.000000, 0.000000}
\pgfsetstrokecolor{dialinecolor}
\pgfpathellipse{\pgfpoint{47.579866\du}{40.184346\du}}{\pgfpoint{1.419866\du}{0\du}}{\pgfpoint{0\du}{1.403171\du}}
\pgfusepath{stroke}
% setfont left to latex
\definecolor{dialinecolor}{rgb}{0.000000, 0.000000, 0.000000}
\pgfsetstrokecolor{dialinecolor}
\node at (47.579866\du,40.316846\du){P1};
\definecolor{dialinecolor}{rgb}{1.000000, 1.000000, 1.000000}
\pgfsetfillcolor{dialinecolor}
\pgfpathellipse{\pgfpoint{51.088572\du}{40.256025\du}}{\pgfpoint{1.419866\du}{0\du}}{\pgfpoint{0\du}{1.403171\du}}
\pgfusepath{fill}
\pgfsetlinewidth{0.100000\du}
\pgfsetdash{}{0pt}
\pgfsetdash{}{0pt}
\pgfsetmiterjoin
\definecolor{dialinecolor}{rgb}{0.000000, 0.000000, 0.000000}
\pgfsetstrokecolor{dialinecolor}
\pgfpathellipse{\pgfpoint{51.088572\du}{40.256025\du}}{\pgfpoint{1.419866\du}{0\du}}{\pgfpoint{0\du}{1.403171\du}}
\pgfusepath{stroke}
% setfont left to latex
\definecolor{dialinecolor}{rgb}{0.000000, 0.000000, 0.000000}
\pgfsetstrokecolor{dialinecolor}
\node at (51.088572\du,40.388525\du){P2};
\definecolor{dialinecolor}{rgb}{1.000000, 1.000000, 1.000000}
\pgfsetfillcolor{dialinecolor}
\pgfpathellipse{\pgfpoint{54.998572\du}{40.256025\du}}{\pgfpoint{1.419866\du}{0\du}}{\pgfpoint{0\du}{1.403171\du}}
\pgfusepath{fill}
\pgfsetlinewidth{0.100000\du}
\pgfsetdash{}{0pt}
\pgfsetdash{}{0pt}
\pgfsetmiterjoin
\definecolor{dialinecolor}{rgb}{0.000000, 0.000000, 0.000000}
\pgfsetstrokecolor{dialinecolor}
\pgfpathellipse{\pgfpoint{54.998572\du}{40.256025\du}}{\pgfpoint{1.419866\du}{0\du}}{\pgfpoint{0\du}{1.403171\du}}
\pgfusepath{stroke}
% setfont left to latex
\definecolor{dialinecolor}{rgb}{0.000000, 0.000000, 0.000000}
\pgfsetstrokecolor{dialinecolor}
\node at (54.998572\du,40.388525\du){P3};
\definecolor{dialinecolor}{rgb}{1.000000, 1.000000, 1.000000}
\pgfsetfillcolor{dialinecolor}
\pgfpathellipse{\pgfpoint{60.008572\du}{40.256025\du}}{\pgfpoint{1.419866\du}{0\du}}{\pgfpoint{0\du}{1.403171\du}}
\pgfusepath{fill}
\pgfsetlinewidth{0.100000\du}
\pgfsetdash{}{0pt}
\pgfsetdash{}{0pt}
\pgfsetmiterjoin
\definecolor{dialinecolor}{rgb}{0.000000, 0.000000, 0.000000}
\pgfsetstrokecolor{dialinecolor}
\pgfpathellipse{\pgfpoint{60.008572\du}{40.256025\du}}{\pgfpoint{1.419866\du}{0\du}}{\pgfpoint{0\du}{1.403171\du}}
\pgfusepath{stroke}
% setfont left to latex
\definecolor{dialinecolor}{rgb}{0.000000, 0.000000, 0.000000}
\pgfsetstrokecolor{dialinecolor}
\node at (60.008572\du,40.388525\du){Pn};
\pgfsetlinewidth{0.100000\du}
\pgfsetdash{}{0pt}
\pgfsetdash{}{0pt}
\pgfsetbuttcap
{
\definecolor{dialinecolor}{rgb}{0.000000, 0.000000, 0.000000}
\pgfsetfillcolor{dialinecolor}
% was here!!!
\pgfsetarrowsend{latex}
\definecolor{dialinecolor}{rgb}{0.000000, 0.000000, 0.000000}
\pgfsetstrokecolor{dialinecolor}
\draw (47.558706\du,37.202854\du)--(47.579866\du,38.781175\du);
}
\pgfsetlinewidth{0.100000\du}
\pgfsetdash{}{0pt}
\pgfsetdash{}{0pt}
\pgfsetbuttcap
{
\definecolor{dialinecolor}{rgb}{0.000000, 0.000000, 0.000000}
\pgfsetfillcolor{dialinecolor}
% was here!!!
\pgfsetarrowsend{latex}
\definecolor{dialinecolor}{rgb}{0.000000, 0.000000, 0.000000}
\pgfsetstrokecolor{dialinecolor}
\draw (51.058706\du,37.152854\du)--(51.088572\du,38.852854\du);
}
\pgfsetlinewidth{0.100000\du}
\pgfsetdash{}{0pt}
\pgfsetdash{}{0pt}
\pgfsetbuttcap
{
\definecolor{dialinecolor}{rgb}{0.000000, 0.000000, 0.000000}
\pgfsetfillcolor{dialinecolor}
% was here!!!
\pgfsetarrowsend{latex}
\definecolor{dialinecolor}{rgb}{0.000000, 0.000000, 0.000000}
\pgfsetstrokecolor{dialinecolor}
\draw (55.058706\du,37.202854\du)--(54.998572\du,38.852854\du);
}
\pgfsetlinewidth{0.100000\du}
\pgfsetdash{}{0pt}
\pgfsetdash{}{0pt}
\pgfsetbuttcap
{
\definecolor{dialinecolor}{rgb}{0.000000, 0.000000, 0.000000}
\pgfsetfillcolor{dialinecolor}
% was here!!!
\pgfsetarrowsend{latex}
\definecolor{dialinecolor}{rgb}{0.000000, 0.000000, 0.000000}
\pgfsetstrokecolor{dialinecolor}
\draw (60.108706\du,37.152854\du)--(60.008572\du,38.852854\du);
}
\pgfsetlinewidth{0.100000\du}
\pgfsetdash{}{0pt}
\pgfsetdash{}{0pt}
\pgfsetbuttcap
{
\definecolor{dialinecolor}{rgb}{0.000000, 0.000000, 0.000000}
\pgfsetfillcolor{dialinecolor}
% was here!!!
\pgfsetarrowsend{latex}
\definecolor{dialinecolor}{rgb}{0.000000, 0.000000, 0.000000}
\pgfsetstrokecolor{dialinecolor}
\draw (46.986507\du,46.199532\du)--(47.117334\du,48.479078\du);
}
\pgfsetlinewidth{0.100000\du}
\pgfsetdash{}{0pt}
\pgfsetdash{}{0pt}
\pgfsetbuttcap
{
\definecolor{dialinecolor}{rgb}{0.000000, 0.000000, 0.000000}
\pgfsetfillcolor{dialinecolor}
% was here!!!
\pgfsetarrowsend{latex}
\definecolor{dialinecolor}{rgb}{0.000000, 0.000000, 0.000000}
\pgfsetstrokecolor{dialinecolor}
\draw (51.029866\du,46.306342\du)--(47.117334\du,48.479078\du);
}
\pgfsetlinewidth{0.100000\du}
\pgfsetdash{}{0pt}
\pgfsetdash{}{0pt}
\pgfsetbuttcap
{
\definecolor{dialinecolor}{rgb}{0.000000, 0.000000, 0.000000}
\pgfsetfillcolor{dialinecolor}
% was here!!!
\pgfsetarrowsend{latex}
\definecolor{dialinecolor}{rgb}{0.000000, 0.000000, 0.000000}
\pgfsetstrokecolor{dialinecolor}
\draw (59.949866\du,46.356342\du)--(47.117334\du,48.479078\du);
}
\pgfsetlinewidth{0.100000\du}
\pgfsetdash{}{0pt}
\pgfsetdash{}{0pt}
\pgfsetbuttcap
{
\definecolor{dialinecolor}{rgb}{0.000000, 0.000000, 0.000000}
\pgfsetfillcolor{dialinecolor}
% was here!!!
\pgfsetarrowsend{latex}
\definecolor{dialinecolor}{rgb}{0.000000, 0.000000, 0.000000}
\pgfsetstrokecolor{dialinecolor}
\draw (51.029866\du,46.306342\du)--(51.017334\du,48.429078\du);
}
\pgfsetlinewidth{0.100000\du}
\pgfsetdash{}{0pt}
\pgfsetdash{}{0pt}
\pgfsetbuttcap
{
\definecolor{dialinecolor}{rgb}{0.000000, 0.000000, 0.000000}
\pgfsetfillcolor{dialinecolor}
% was here!!!
\pgfsetarrowsend{latex}
\definecolor{dialinecolor}{rgb}{0.000000, 0.000000, 0.000000}
\pgfsetstrokecolor{dialinecolor}
\draw (54.889866\du,46.306342\du)--(51.017334\du,48.429078\du);
}
\pgfsetlinewidth{0.100000\du}
\pgfsetdash{}{0pt}
\pgfsetdash{}{0pt}
\pgfsetbuttcap
{
\definecolor{dialinecolor}{rgb}{0.000000, 0.000000, 0.000000}
\pgfsetfillcolor{dialinecolor}
% was here!!!
\pgfsetarrowsend{latex}
\definecolor{dialinecolor}{rgb}{0.000000, 0.000000, 0.000000}
\pgfsetstrokecolor{dialinecolor}
\draw (54.889866\du,46.306342\du)--(55.017334\du,48.479078\du);
}
\pgfsetlinewidth{0.100000\du}
\pgfsetdash{}{0pt}
\pgfsetdash{}{0pt}
\pgfsetbuttcap
{
\definecolor{dialinecolor}{rgb}{0.000000, 0.000000, 0.000000}
\pgfsetfillcolor{dialinecolor}
% was here!!!
\pgfsetarrowsend{latex}
\definecolor{dialinecolor}{rgb}{0.000000, 0.000000, 0.000000}
\pgfsetstrokecolor{dialinecolor}
\draw (46.986507\du,46.199532\du)--(55.017334\du,48.529078\du);
}
\pgfsetlinewidth{0.100000\du}
\pgfsetdash{}{0pt}
\pgfsetdash{}{0pt}
\pgfsetbuttcap
{
\definecolor{dialinecolor}{rgb}{0.000000, 0.000000, 0.000000}
\pgfsetfillcolor{dialinecolor}
% was here!!!
\pgfsetarrowsend{latex}
\definecolor{dialinecolor}{rgb}{0.000000, 0.000000, 0.000000}
\pgfsetstrokecolor{dialinecolor}
\draw (59.949866\du,46.356342\du)--(59.967334\du,48.479078\du);
}
\pgfsetlinewidth{0.100000\du}
\pgfsetdash{}{0pt}
\pgfsetdash{}{0pt}
\pgfsetbuttcap
{
\definecolor{dialinecolor}{rgb}{0.000000, 0.000000, 0.000000}
\pgfsetfillcolor{dialinecolor}
% was here!!!
\pgfsetarrowsend{latex}
\definecolor{dialinecolor}{rgb}{0.000000, 0.000000, 0.000000}
\pgfsetstrokecolor{dialinecolor}
\draw (51.029866\du,46.306342\du)--(59.967334\du,48.529078\du);
}
\definecolor{dialinecolor}{rgb}{1.000000, 1.000000, 1.000000}
\pgfsetfillcolor{dialinecolor}
\fill (43.980402\du,48.479078\du)--(43.980402\du,50.679078\du)--(63.009714\du,50.679078\du)--(63.009714\du,48.479078\du)--cycle;
\pgfsetlinewidth{0.100000\du}
\pgfsetdash{}{0pt}
\pgfsetdash{}{0pt}
\pgfsetmiterjoin
\definecolor{dialinecolor}{rgb}{0.000000, 0.000000, 0.000000}
\pgfsetstrokecolor{dialinecolor}
\draw (43.980402\du,48.479078\du)--(43.980402\du,50.679078\du)--(63.009714\du,50.679078\du)--(63.009714\du,48.479078\du)--cycle;
% setfont left to latex
\definecolor{dialinecolor}{rgb}{0.000000, 0.000000, 0.000000}
\pgfsetstrokecolor{dialinecolor}
\node at (53.495058\du,49.819078\du){Berechne Geschwindigkeit (V)};
\pgfsetlinewidth{0.300000\du}
\pgfsetdash{{1.000000\du}{1.000000\du}}{0\du}
\pgfsetdash{{1.000000\du}{1.000000\du}}{0\du}
\pgfsetbuttcap
{
\definecolor{dialinecolor}{rgb}{0.000000, 0.000000, 0.000000}
\pgfsetfillcolor{dialinecolor}
% was here!!!
\definecolor{dialinecolor}{rgb}{0.000000, 0.000000, 0.000000}
\pgfsetstrokecolor{dialinecolor}
\draw (44.410395\du,42.550395\du)--(63.360395\du,42.500395\du);
}
\definecolor{dialinecolor}{rgb}{1.000000, 1.000000, 1.000000}
\pgfsetfillcolor{dialinecolor}
\pgfpathellipse{\pgfpoint{47.529866\du}{22.653171\du}}{\pgfpoint{1.419866\du}{0\du}}{\pgfpoint{0\du}{1.403171\du}}
\pgfusepath{fill}
\pgfsetlinewidth{0.100000\du}
\pgfsetdash{}{0pt}
\pgfsetdash{}{0pt}
\pgfsetmiterjoin
\definecolor{dialinecolor}{rgb}{0.000000, 0.000000, 0.000000}
\pgfsetstrokecolor{dialinecolor}
\pgfpathellipse{\pgfpoint{47.529866\du}{22.653171\du}}{\pgfpoint{1.419866\du}{0\du}}{\pgfpoint{0\du}{1.403171\du}}
\pgfusepath{stroke}
% setfont left to latex
\definecolor{dialinecolor}{rgb}{0.000000, 0.000000, 0.000000}
\pgfsetstrokecolor{dialinecolor}
\node at (47.529866\du,22.785671\du){p1};
\definecolor{dialinecolor}{rgb}{1.000000, 1.000000, 1.000000}
\pgfsetfillcolor{dialinecolor}
\pgfpathellipse{\pgfpoint{51.029866\du}{22.653171\du}}{\pgfpoint{1.419866\du}{0\du}}{\pgfpoint{0\du}{1.403171\du}}
\pgfusepath{fill}
\pgfsetlinewidth{0.100000\du}
\pgfsetdash{}{0pt}
\pgfsetdash{}{0pt}
\pgfsetmiterjoin
\definecolor{dialinecolor}{rgb}{0.000000, 0.000000, 0.000000}
\pgfsetstrokecolor{dialinecolor}
\pgfpathellipse{\pgfpoint{51.029866\du}{22.653171\du}}{\pgfpoint{1.419866\du}{0\du}}{\pgfpoint{0\du}{1.403171\du}}
\pgfusepath{stroke}
% setfont left to latex
\definecolor{dialinecolor}{rgb}{0.000000, 0.000000, 0.000000}
\pgfsetstrokecolor{dialinecolor}
\node at (51.029866\du,22.785671\du){p2};
\definecolor{dialinecolor}{rgb}{1.000000, 1.000000, 1.000000}
\pgfsetfillcolor{dialinecolor}
\pgfpathellipse{\pgfpoint{54.889866\du}{22.653171\du}}{\pgfpoint{1.419866\du}{0\du}}{\pgfpoint{0\du}{1.403171\du}}
\pgfusepath{fill}
\pgfsetlinewidth{0.100000\du}
\pgfsetdash{}{0pt}
\pgfsetdash{}{0pt}
\pgfsetmiterjoin
\definecolor{dialinecolor}{rgb}{0.000000, 0.000000, 0.000000}
\pgfsetstrokecolor{dialinecolor}
\pgfpathellipse{\pgfpoint{54.889866\du}{22.653171\du}}{\pgfpoint{1.419866\du}{0\du}}{\pgfpoint{0\du}{1.403171\du}}
\pgfusepath{stroke}
% setfont left to latex
\definecolor{dialinecolor}{rgb}{0.000000, 0.000000, 0.000000}
\pgfsetstrokecolor{dialinecolor}
\node at (54.889866\du,22.785671\du){p3};
\definecolor{dialinecolor}{rgb}{1.000000, 1.000000, 1.000000}
\pgfsetfillcolor{dialinecolor}
\pgfpathellipse{\pgfpoint{59.949866\du}{22.703171\du}}{\pgfpoint{1.419866\du}{0\du}}{\pgfpoint{0\du}{1.403171\du}}
\pgfusepath{fill}
\pgfsetlinewidth{0.100000\du}
\pgfsetdash{}{0pt}
\pgfsetdash{}{0pt}
\pgfsetmiterjoin
\definecolor{dialinecolor}{rgb}{0.000000, 0.000000, 0.000000}
\pgfsetstrokecolor{dialinecolor}
\pgfpathellipse{\pgfpoint{59.949866\du}{22.703171\du}}{\pgfpoint{1.419866\du}{0\du}}{\pgfpoint{0\du}{1.403171\du}}
\pgfusepath{stroke}
% setfont left to latex
\definecolor{dialinecolor}{rgb}{0.000000, 0.000000, 0.000000}
\pgfsetstrokecolor{dialinecolor}
\node at (59.949866\du,22.835671\du){pn};
\definecolor{dialinecolor}{rgb}{1.000000, 1.000000, 1.000000}
\pgfsetfillcolor{dialinecolor}
\pgfpathellipse{\pgfpoint{47.529866\du}{44.903171\du}}{\pgfpoint{1.419866\du}{0\du}}{\pgfpoint{0\du}{1.403171\du}}
\pgfusepath{fill}
\pgfsetlinewidth{0.100000\du}
\pgfsetdash{}{0pt}
\pgfsetdash{}{0pt}
\pgfsetmiterjoin
\definecolor{dialinecolor}{rgb}{0.000000, 0.000000, 0.000000}
\pgfsetstrokecolor{dialinecolor}
\pgfpathellipse{\pgfpoint{47.529866\du}{44.903171\du}}{\pgfpoint{1.419866\du}{0\du}}{\pgfpoint{0\du}{1.403171\du}}
\pgfusepath{stroke}
% setfont left to latex
\definecolor{dialinecolor}{rgb}{0.000000, 0.000000, 0.000000}
\pgfsetstrokecolor{dialinecolor}
\node at (47.529866\du,45.035671\du){p1};
\definecolor{dialinecolor}{rgb}{1.000000, 1.000000, 1.000000}
\pgfsetfillcolor{dialinecolor}
\pgfpathellipse{\pgfpoint{51.029866\du}{44.903171\du}}{\pgfpoint{1.419866\du}{0\du}}{\pgfpoint{0\du}{1.403171\du}}
\pgfusepath{fill}
\pgfsetlinewidth{0.100000\du}
\pgfsetdash{}{0pt}
\pgfsetdash{}{0pt}
\pgfsetmiterjoin
\definecolor{dialinecolor}{rgb}{0.000000, 0.000000, 0.000000}
\pgfsetstrokecolor{dialinecolor}
\pgfpathellipse{\pgfpoint{51.029866\du}{44.903171\du}}{\pgfpoint{1.419866\du}{0\du}}{\pgfpoint{0\du}{1.403171\du}}
\pgfusepath{stroke}
% setfont left to latex
\definecolor{dialinecolor}{rgb}{0.000000, 0.000000, 0.000000}
\pgfsetstrokecolor{dialinecolor}
\node at (51.029866\du,45.035671\du){p2};
\definecolor{dialinecolor}{rgb}{1.000000, 1.000000, 1.000000}
\pgfsetfillcolor{dialinecolor}
\pgfpathellipse{\pgfpoint{54.889866\du}{44.903171\du}}{\pgfpoint{1.419866\du}{0\du}}{\pgfpoint{0\du}{1.403171\du}}
\pgfusepath{fill}
\pgfsetlinewidth{0.100000\du}
\pgfsetdash{}{0pt}
\pgfsetdash{}{0pt}
\pgfsetmiterjoin
\definecolor{dialinecolor}{rgb}{0.000000, 0.000000, 0.000000}
\pgfsetstrokecolor{dialinecolor}
\pgfpathellipse{\pgfpoint{54.889866\du}{44.903171\du}}{\pgfpoint{1.419866\du}{0\du}}{\pgfpoint{0\du}{1.403171\du}}
\pgfusepath{stroke}
% setfont left to latex
\definecolor{dialinecolor}{rgb}{0.000000, 0.000000, 0.000000}
\pgfsetstrokecolor{dialinecolor}
\node at (54.889866\du,45.035671\du){p3};
\definecolor{dialinecolor}{rgb}{1.000000, 1.000000, 1.000000}
\pgfsetfillcolor{dialinecolor}
\pgfpathellipse{\pgfpoint{59.949866\du}{44.953171\du}}{\pgfpoint{1.419866\du}{0\du}}{\pgfpoint{0\du}{1.403171\du}}
\pgfusepath{fill}
\pgfsetlinewidth{0.100000\du}
\pgfsetdash{}{0pt}
\pgfsetdash{}{0pt}
\pgfsetmiterjoin
\definecolor{dialinecolor}{rgb}{0.000000, 0.000000, 0.000000}
\pgfsetstrokecolor{dialinecolor}
\pgfpathellipse{\pgfpoint{59.949866\du}{44.953171\du}}{\pgfpoint{1.419866\du}{0\du}}{\pgfpoint{0\du}{1.403171\du}}
\pgfusepath{stroke}
% setfont left to latex
\definecolor{dialinecolor}{rgb}{0.000000, 0.000000, 0.000000}
\pgfsetstrokecolor{dialinecolor}
\node at (59.949866\du,45.085671\du){pn};
\definecolor{dialinecolor}{rgb}{1.000000, 1.000000, 1.000000}
\pgfsetfillcolor{dialinecolor}
\pgfpathellipse{\pgfpoint{47.529866\du}{53.684346\du}}{\pgfpoint{1.419866\du}{0\du}}{\pgfpoint{0\du}{1.403171\du}}
\pgfusepath{fill}
\pgfsetlinewidth{0.100000\du}
\pgfsetdash{}{0pt}
\pgfsetdash{}{0pt}
\pgfsetmiterjoin
\definecolor{dialinecolor}{rgb}{0.000000, 0.000000, 0.000000}
\pgfsetstrokecolor{dialinecolor}
\pgfpathellipse{\pgfpoint{47.529866\du}{53.684346\du}}{\pgfpoint{1.419866\du}{0\du}}{\pgfpoint{0\du}{1.403171\du}}
\pgfusepath{stroke}
% setfont left to latex
\definecolor{dialinecolor}{rgb}{0.000000, 0.000000, 0.000000}
\pgfsetstrokecolor{dialinecolor}
\node at (47.529866\du,53.816846\du){V1};
\definecolor{dialinecolor}{rgb}{1.000000, 1.000000, 1.000000}
\pgfsetfillcolor{dialinecolor}
\pgfpathellipse{\pgfpoint{51.038572\du}{53.756025\du}}{\pgfpoint{1.419866\du}{0\du}}{\pgfpoint{0\du}{1.403171\du}}
\pgfusepath{fill}
\pgfsetlinewidth{0.100000\du}
\pgfsetdash{}{0pt}
\pgfsetdash{}{0pt}
\pgfsetmiterjoin
\definecolor{dialinecolor}{rgb}{0.000000, 0.000000, 0.000000}
\pgfsetstrokecolor{dialinecolor}
\pgfpathellipse{\pgfpoint{51.038572\du}{53.756025\du}}{\pgfpoint{1.419866\du}{0\du}}{\pgfpoint{0\du}{1.403171\du}}
\pgfusepath{stroke}
% setfont left to latex
\definecolor{dialinecolor}{rgb}{0.000000, 0.000000, 0.000000}
\pgfsetstrokecolor{dialinecolor}
\node at (51.038572\du,53.888525\du){V2};
\definecolor{dialinecolor}{rgb}{1.000000, 1.000000, 1.000000}
\pgfsetfillcolor{dialinecolor}
\pgfpathellipse{\pgfpoint{54.948572\du}{53.756025\du}}{\pgfpoint{1.419866\du}{0\du}}{\pgfpoint{0\du}{1.403171\du}}
\pgfusepath{fill}
\pgfsetlinewidth{0.100000\du}
\pgfsetdash{}{0pt}
\pgfsetdash{}{0pt}
\pgfsetmiterjoin
\definecolor{dialinecolor}{rgb}{0.000000, 0.000000, 0.000000}
\pgfsetstrokecolor{dialinecolor}
\pgfpathellipse{\pgfpoint{54.948572\du}{53.756025\du}}{\pgfpoint{1.419866\du}{0\du}}{\pgfpoint{0\du}{1.403171\du}}
\pgfusepath{stroke}
% setfont left to latex
\definecolor{dialinecolor}{rgb}{0.000000, 0.000000, 0.000000}
\pgfsetstrokecolor{dialinecolor}
\node at (54.948572\du,53.888525\du){V3};
\definecolor{dialinecolor}{rgb}{1.000000, 1.000000, 1.000000}
\pgfsetfillcolor{dialinecolor}
\pgfpathellipse{\pgfpoint{59.958572\du}{53.756025\du}}{\pgfpoint{1.419866\du}{0\du}}{\pgfpoint{0\du}{1.403171\du}}
\pgfusepath{fill}
\pgfsetlinewidth{0.100000\du}
\pgfsetdash{}{0pt}
\pgfsetdash{}{0pt}
\pgfsetmiterjoin
\definecolor{dialinecolor}{rgb}{0.000000, 0.000000, 0.000000}
\pgfsetstrokecolor{dialinecolor}
\pgfpathellipse{\pgfpoint{59.958572\du}{53.756025\du}}{\pgfpoint{1.419866\du}{0\du}}{\pgfpoint{0\du}{1.403171\du}}
\pgfusepath{stroke}
% setfont left to latex
\definecolor{dialinecolor}{rgb}{0.000000, 0.000000, 0.000000}
\pgfsetstrokecolor{dialinecolor}
\node at (59.958572\du,53.888525\du){Vn};
\pgfsetlinewidth{0.100000\du}
\pgfsetdash{}{0pt}
\pgfsetdash{}{0pt}
\pgfsetbuttcap
{
\definecolor{dialinecolor}{rgb}{0.000000, 0.000000, 0.000000}
\pgfsetfillcolor{dialinecolor}
% was here!!!
\pgfsetarrowsend{latex}
\definecolor{dialinecolor}{rgb}{0.000000, 0.000000, 0.000000}
\pgfsetstrokecolor{dialinecolor}
\draw (47.508706\du,50.702854\du)--(47.529866\du,52.281175\du);
}
\pgfsetlinewidth{0.100000\du}
\pgfsetdash{}{0pt}
\pgfsetdash{}{0pt}
\pgfsetbuttcap
{
\definecolor{dialinecolor}{rgb}{0.000000, 0.000000, 0.000000}
\pgfsetfillcolor{dialinecolor}
% was here!!!
\pgfsetarrowsend{latex}
\definecolor{dialinecolor}{rgb}{0.000000, 0.000000, 0.000000}
\pgfsetstrokecolor{dialinecolor}
\draw (51.008706\du,50.652854\du)--(51.038572\du,52.352854\du);
}
\pgfsetlinewidth{0.100000\du}
\pgfsetdash{}{0pt}
\pgfsetdash{}{0pt}
\pgfsetbuttcap
{
\definecolor{dialinecolor}{rgb}{0.000000, 0.000000, 0.000000}
\pgfsetfillcolor{dialinecolor}
% was here!!!
\pgfsetarrowsend{latex}
\definecolor{dialinecolor}{rgb}{0.000000, 0.000000, 0.000000}
\pgfsetstrokecolor{dialinecolor}
\draw (55.008706\du,50.702854\du)--(54.948572\du,52.352854\du);
}
\pgfsetlinewidth{0.100000\du}
\pgfsetdash{}{0pt}
\pgfsetdash{}{0pt}
\pgfsetbuttcap
{
\definecolor{dialinecolor}{rgb}{0.000000, 0.000000, 0.000000}
\pgfsetfillcolor{dialinecolor}
% was here!!!
\pgfsetarrowsend{latex}
\definecolor{dialinecolor}{rgb}{0.000000, 0.000000, 0.000000}
\pgfsetstrokecolor{dialinecolor}
\draw (60.058706\du,50.652854\du)--(59.958572\du,52.352854\du);
}
\pgfsetlinewidth{0.100000\du}
\pgfsetdash{}{0pt}
\pgfsetdash{}{0pt}
\pgfsetbuttcap
{
\definecolor{dialinecolor}{rgb}{0.000000, 0.000000, 0.000000}
\pgfsetfillcolor{dialinecolor}
% was here!!!
\pgfsetarrowsend{latex}
\definecolor{dialinecolor}{rgb}{0.000000, 0.000000, 0.000000}
\pgfsetstrokecolor{dialinecolor}
\draw (47.146174\du,55.062100\du)--(47.120395\du,57.333925\du);
}
\pgfsetlinewidth{0.100000\du}
\pgfsetdash{}{0pt}
\pgfsetdash{}{0pt}
\pgfsetbuttcap
{
\definecolor{dialinecolor}{rgb}{0.000000, 0.000000, 0.000000}
\pgfsetfillcolor{dialinecolor}
% was here!!!
\pgfsetarrowsend{latex}
\definecolor{dialinecolor}{rgb}{0.000000, 0.000000, 0.000000}
\pgfsetstrokecolor{dialinecolor}
\draw (51.038572\du,55.159196\du)--(51.020395\du,57.283925\du);
}
\pgfsetlinewidth{0.100000\du}
\pgfsetdash{}{0pt}
\pgfsetdash{}{0pt}
\pgfsetbuttcap
{
\definecolor{dialinecolor}{rgb}{0.000000, 0.000000, 0.000000}
\pgfsetfillcolor{dialinecolor}
% was here!!!
\pgfsetarrowsend{latex}
\definecolor{dialinecolor}{rgb}{0.000000, 0.000000, 0.000000}
\pgfsetstrokecolor{dialinecolor}
\draw (54.948572\du,55.159196\du)--(55.020395\du,57.333925\du);
}
\pgfsetlinewidth{0.100000\du}
\pgfsetdash{}{0pt}
\pgfsetdash{}{0pt}
\pgfsetbuttcap
{
\definecolor{dialinecolor}{rgb}{0.000000, 0.000000, 0.000000}
\pgfsetfillcolor{dialinecolor}
% was here!!!
\pgfsetarrowsend{latex}
\definecolor{dialinecolor}{rgb}{0.000000, 0.000000, 0.000000}
\pgfsetstrokecolor{dialinecolor}
\draw (59.958572\du,55.159196\du)--(59.970395\du,57.333925\du);
}
\definecolor{dialinecolor}{rgb}{1.000000, 1.000000, 1.000000}
\pgfsetfillcolor{dialinecolor}
\fill (43.980402\du,57.283925\du)--(43.980402\du,59.550000\du)--(62.955189\du,59.550000\du)--(62.955189\du,57.283925\du)--cycle;
\pgfsetlinewidth{0.100000\du}
\pgfsetdash{}{0pt}
\pgfsetdash{}{0pt}
\pgfsetmiterjoin
\definecolor{dialinecolor}{rgb}{0.000000, 0.000000, 0.000000}
\pgfsetstrokecolor{dialinecolor}
\draw (43.980402\du,57.283925\du)--(43.980402\du,59.550000\du)--(62.955189\du,59.550000\du)--(62.955189\du,57.283925\du)--cycle;
% setfont left to latex
\definecolor{dialinecolor}{rgb}{0.000000, 0.000000, 0.000000}
\pgfsetstrokecolor{dialinecolor}
\node at (53.467796\du,58.656963\du){Berechne Position};
\pgfsetlinewidth{0.300000\du}
\pgfsetdash{{1.000000\du}{1.000000\du}}{0\du}
\pgfsetdash{{1.000000\du}{1.000000\du}}{0\du}
\pgfsetbuttcap
{
\definecolor{dialinecolor}{rgb}{0.000000, 0.000000, 0.000000}
\pgfsetfillcolor{dialinecolor}
% was here!!!
\definecolor{dialinecolor}{rgb}{0.000000, 0.000000, 0.000000}
\pgfsetstrokecolor{dialinecolor}
\draw (44.360395\du,56.150000\du)--(63.310395\du,56.100000\du);
}
\definecolor{dialinecolor}{rgb}{1.000000, 1.000000, 1.000000}
\pgfsetfillcolor{dialinecolor}
\pgfpathellipse{\pgfpoint{47.490261\du}{62.534346\du}}{\pgfpoint{1.419866\du}{0\du}}{\pgfpoint{0\du}{1.403171\du}}
\pgfusepath{fill}
\pgfsetlinewidth{0.100000\du}
\pgfsetdash{}{0pt}
\pgfsetdash{}{0pt}
\pgfsetmiterjoin
\definecolor{dialinecolor}{rgb}{0.000000, 0.000000, 0.000000}
\pgfsetstrokecolor{dialinecolor}
\pgfpathellipse{\pgfpoint{47.490261\du}{62.534346\du}}{\pgfpoint{1.419866\du}{0\du}}{\pgfpoint{0\du}{1.403171\du}}
\pgfusepath{stroke}
% setfont left to latex
\definecolor{dialinecolor}{rgb}{0.000000, 0.000000, 0.000000}
\pgfsetstrokecolor{dialinecolor}
\node at (47.490261\du,62.666846\du){P1};
\definecolor{dialinecolor}{rgb}{1.000000, 1.000000, 1.000000}
\pgfsetfillcolor{dialinecolor}
\pgfpathellipse{\pgfpoint{50.998967\du}{62.606025\du}}{\pgfpoint{1.419866\du}{0\du}}{\pgfpoint{0\du}{1.403171\du}}
\pgfusepath{fill}
\pgfsetlinewidth{0.100000\du}
\pgfsetdash{}{0pt}
\pgfsetdash{}{0pt}
\pgfsetmiterjoin
\definecolor{dialinecolor}{rgb}{0.000000, 0.000000, 0.000000}
\pgfsetstrokecolor{dialinecolor}
\pgfpathellipse{\pgfpoint{50.998967\du}{62.606025\du}}{\pgfpoint{1.419866\du}{0\du}}{\pgfpoint{0\du}{1.403171\du}}
\pgfusepath{stroke}
% setfont left to latex
\definecolor{dialinecolor}{rgb}{0.000000, 0.000000, 0.000000}
\pgfsetstrokecolor{dialinecolor}
\node at (50.998967\du,62.738525\du){P2};
\definecolor{dialinecolor}{rgb}{1.000000, 1.000000, 1.000000}
\pgfsetfillcolor{dialinecolor}
\pgfpathellipse{\pgfpoint{54.908967\du}{62.606025\du}}{\pgfpoint{1.419866\du}{0\du}}{\pgfpoint{0\du}{1.403171\du}}
\pgfusepath{fill}
\pgfsetlinewidth{0.100000\du}
\pgfsetdash{}{0pt}
\pgfsetdash{}{0pt}
\pgfsetmiterjoin
\definecolor{dialinecolor}{rgb}{0.000000, 0.000000, 0.000000}
\pgfsetstrokecolor{dialinecolor}
\pgfpathellipse{\pgfpoint{54.908967\du}{62.606025\du}}{\pgfpoint{1.419866\du}{0\du}}{\pgfpoint{0\du}{1.403171\du}}
\pgfusepath{stroke}
% setfont left to latex
\definecolor{dialinecolor}{rgb}{0.000000, 0.000000, 0.000000}
\pgfsetstrokecolor{dialinecolor}
\node at (54.908967\du,62.738525\du){P3};
\definecolor{dialinecolor}{rgb}{1.000000, 1.000000, 1.000000}
\pgfsetfillcolor{dialinecolor}
\pgfpathellipse{\pgfpoint{59.918967\du}{62.606025\du}}{\pgfpoint{1.419866\du}{0\du}}{\pgfpoint{0\du}{1.403171\du}}
\pgfusepath{fill}
\pgfsetlinewidth{0.100000\du}
\pgfsetdash{}{0pt}
\pgfsetdash{}{0pt}
\pgfsetmiterjoin
\definecolor{dialinecolor}{rgb}{0.000000, 0.000000, 0.000000}
\pgfsetstrokecolor{dialinecolor}
\pgfpathellipse{\pgfpoint{59.918967\du}{62.606025\du}}{\pgfpoint{1.419866\du}{0\du}}{\pgfpoint{0\du}{1.403171\du}}
\pgfusepath{stroke}
% setfont left to latex
\definecolor{dialinecolor}{rgb}{0.000000, 0.000000, 0.000000}
\pgfsetstrokecolor{dialinecolor}
\node at (59.918967\du,62.738525\du){Pn};
\pgfsetlinewidth{0.100000\du}
\pgfsetdash{}{0pt}
\pgfsetdash{}{0pt}
\pgfsetbuttcap
{
\definecolor{dialinecolor}{rgb}{0.000000, 0.000000, 0.000000}
\pgfsetfillcolor{dialinecolor}
% was here!!!
\pgfsetarrowsend{latex}
\definecolor{dialinecolor}{rgb}{0.000000, 0.000000, 0.000000}
\pgfsetstrokecolor{dialinecolor}
\draw (47.469101\du,59.552854\du)--(47.490261\du,61.131175\du);
}
\pgfsetlinewidth{0.100000\du}
\pgfsetdash{}{0pt}
\pgfsetdash{}{0pt}
\pgfsetbuttcap
{
\definecolor{dialinecolor}{rgb}{0.000000, 0.000000, 0.000000}
\pgfsetfillcolor{dialinecolor}
% was here!!!
\pgfsetarrowsend{latex}
\definecolor{dialinecolor}{rgb}{0.000000, 0.000000, 0.000000}
\pgfsetstrokecolor{dialinecolor}
\draw (50.969101\du,59.502854\du)--(50.998967\du,61.202854\du);
}
\pgfsetlinewidth{0.100000\du}
\pgfsetdash{}{0pt}
\pgfsetdash{}{0pt}
\pgfsetbuttcap
{
\definecolor{dialinecolor}{rgb}{0.000000, 0.000000, 0.000000}
\pgfsetfillcolor{dialinecolor}
% was here!!!
\pgfsetarrowsend{latex}
\definecolor{dialinecolor}{rgb}{0.000000, 0.000000, 0.000000}
\pgfsetstrokecolor{dialinecolor}
\draw (54.969101\du,59.552854\du)--(54.908967\du,61.202854\du);
}
\pgfsetlinewidth{0.100000\du}
\pgfsetdash{}{0pt}
\pgfsetdash{}{0pt}
\pgfsetbuttcap
{
\definecolor{dialinecolor}{rgb}{0.000000, 0.000000, 0.000000}
\pgfsetfillcolor{dialinecolor}
% was here!!!
\pgfsetarrowsend{latex}
\definecolor{dialinecolor}{rgb}{0.000000, 0.000000, 0.000000}
\pgfsetstrokecolor{dialinecolor}
\draw (60.019101\du,59.502854\du)--(59.918967\du,61.202854\du);
}
\definecolor{dialinecolor}{rgb}{1.000000, 1.000000, 1.000000}
\pgfsetfillcolor{dialinecolor}
\fill (20.349536\du,26.838567\du)--(20.349536\du,28.738567\du)--(23.939536\du,28.738567\du)--(23.939536\du,26.838567\du)--cycle;
\pgfsetlinewidth{0.100000\du}
\pgfsetdash{}{0pt}
\pgfsetdash{}{0pt}
\pgfsetmiterjoin
\definecolor{dialinecolor}{rgb}{0.000000, 0.000000, 0.000000}
\pgfsetstrokecolor{dialinecolor}
\draw (20.349536\du,26.838567\du)--(20.349536\du,28.738567\du)--(23.939536\du,28.738567\du)--(23.939536\du,26.838567\du)--cycle;
% setfont left to latex
\definecolor{dialinecolor}{rgb}{0.000000, 0.000000, 0.000000}
\pgfsetstrokecolor{dialinecolor}
\node at (22.144536\du,28.028567\du){};
\definecolor{dialinecolor}{rgb}{1.000000, 1.000000, 1.000000}
\pgfsetfillcolor{dialinecolor}
\pgfpathellipse{\pgfpoint{22.119402\du}{24.741738\du}}{\pgfpoint{1.419866\du}{0\du}}{\pgfpoint{0\du}{1.403171\du}}
\pgfusepath{fill}
\pgfsetlinewidth{0.100000\du}
\pgfsetdash{}{0pt}
\pgfsetdash{}{0pt}
\pgfsetmiterjoin
\definecolor{dialinecolor}{rgb}{0.000000, 0.000000, 0.000000}
\pgfsetstrokecolor{dialinecolor}
\pgfpathellipse{\pgfpoint{22.119402\du}{24.741738\du}}{\pgfpoint{1.419866\du}{0\du}}{\pgfpoint{0\du}{1.403171\du}}
\pgfusepath{stroke}
% setfont left to latex
\definecolor{dialinecolor}{rgb}{0.000000, 0.000000, 0.000000}
\pgfsetstrokecolor{dialinecolor}
\node at (22.119402\du,24.874238\du){};
\pgfsetlinewidth{0.300000\du}
\pgfsetdash{{1.000000\du}{1.000000\du}}{0\du}
\pgfsetdash{{1.000000\du}{1.000000\du}}{0\du}
\pgfsetbuttcap
{
\definecolor{dialinecolor}{rgb}{0.000000, 0.000000, 0.000000}
\pgfsetfillcolor{dialinecolor}
% was here!!!
\definecolor{dialinecolor}{rgb}{0.000000, 0.000000, 0.000000}
\pgfsetstrokecolor{dialinecolor}
\draw (20.449931\du,29.938962\du)--(23.739536\du,29.988567\du);
}
\pgfsetlinewidth{0.200000\du}
\pgfsetdash{}{0pt}
\pgfsetdash{}{0pt}
\pgfsetbuttcap
{
\definecolor{dialinecolor}{rgb}{0.000000, 0.000000, 0.000000}
\pgfsetfillcolor{dialinecolor}
% was here!!!
\pgfsetarrowsend{latex}
\definecolor{dialinecolor}{rgb}{0.000000, 0.000000, 0.000000}
\pgfsetstrokecolor{dialinecolor}
\draw (43.008922\du,19.988489\du)--(42.998948\du,64.850062\du);
}
\pgfsetlinewidth{0.200000\du}
\pgfsetdash{}{0pt}
\pgfsetdash{}{0pt}
\pgfsetbuttcap
{
\definecolor{dialinecolor}{rgb}{0.000000, 0.000000, 0.000000}
\pgfsetfillcolor{dialinecolor}
% was here!!!
\pgfsetarrowsend{latex}
\definecolor{dialinecolor}{rgb}{0.000000, 0.000000, 0.000000}
\pgfsetstrokecolor{dialinecolor}
\draw (42.983922\du,19.988489\du)--(62.483884\du,19.988489\du);
}
\pgfsetlinewidth{0.000000\du}
\pgfsetdash{}{0pt}
\pgfsetmiterjoin
\pgfsetroundcap
\definecolor{dialinecolor}{rgb}{0.000000, 0.000000, 0.000000}
\pgfsetfillcolor{dialinecolor}
\pgfpathmoveto{\pgfpoint{42.190061\du}{44.413822\du}}
\pgfpathlineto{\pgfpoint{42.086639\du}{44.413490\du}}
\pgfpathlineto{\pgfpoint{41.706825\du}{44.107605\du}}
\pgfpathlineto{\pgfpoint{41.706252\du}{44.378040\du}}
\pgfpathlineto{\pgfpoint{41.606668\du}{44.376982\du}}
\pgfpathlineto{\pgfpoint{41.609658\du}{43.951050\du}}
\pgfpathlineto{\pgfpoint{41.700841\du}{43.949720\du}}
\pgfpathlineto{\pgfpoint{42.091232\du}{44.269508\du}}
\pgfpathlineto{\pgfpoint{42.092439\du}{43.939322\du}}
\pgfpathlineto{\pgfpoint{42.192024\du}{43.940379\du}}
\pgfpathlineto{\pgfpoint{42.190061\du}{44.413822\du}}
\pgfusepath{fill}
\definecolor{dialinecolor}{rgb}{0.000000, 0.000000, 0.000000}
\pgfsetfillcolor{dialinecolor}
\pgfpathmoveto{\pgfpoint{42.059827\du}{43.619503\du}}
\pgfpathlineto{\pgfpoint{42.078626\du}{43.508615\du}}
\pgfpathcurveto{\pgfpoint{42.118248\du}{43.528955\du}}{\pgfpoint{42.146356\du}{43.551471\du}}{\pgfpoint{42.171350\du}{43.578550\du}}
\pgfpathcurveto{\pgfpoint{42.189393\du}{43.610919\du}}{\pgfpoint{42.200485\du}{43.648577\du}}{\pgfpoint{42.200788\du}{43.692249\du}}
\pgfpathcurveto{\pgfpoint{42.199216\du}{43.768078\du}}{\pgfpoint{42.177940\du}{43.823779\du}}{\pgfpoint{42.132394\du}{43.856239\du}}
\pgfpathcurveto{\pgfpoint{42.094525\du}{43.887248\du}}{\pgfpoint{42.045352\du}{43.900516\du}}{\pgfpoint{41.985601\du}{43.899881\du}}
\pgfpathcurveto{\pgfpoint{41.918174\du}{43.900698\du}}{\pgfpoint{41.867036\du}{43.882534\du}}{\pgfpoint{41.823787\du}{43.843003\du}}
\pgfpathcurveto{\pgfpoint{41.785827\du}{43.810422\du}}{\pgfpoint{41.764882\du}{43.762700\du}}{\pgfpoint{41.761678\du}{43.703675\du}}
\pgfpathcurveto{\pgfpoint{41.766151\du}{43.643199\du}}{\pgfpoint{41.785040\du}{43.595900\du}}{\pgfpoint{41.825297\du}{43.556489\du}}
\pgfpathcurveto{\pgfpoint{41.867005\du}{43.524755\du}}{\pgfpoint{41.930805\du}{43.504747\du}}{\pgfpoint{42.014311\du}{43.504867\du}}
\pgfpathlineto{\pgfpoint{42.015189\du}{43.782979\du}}
\pgfpathcurveto{\pgfpoint{42.051184\du}{43.784127\du}}{\pgfpoint{42.077327\du}{43.775211\du}}{\pgfpoint{42.093617\du}{43.756231\du}}
\pgfpathcurveto{\pgfpoint{42.115196\du}{43.744203\du}}{\pgfpoint{42.123085\du}{43.722835\du}}{\pgfpoint{42.117282\du}{43.692128\du}}
\pgfpathcurveto{\pgfpoint{42.122782\du}{43.679163\du}}{\pgfpoint{42.116042\du}{43.664535\du}}{\pgfpoint{42.105464\du}{43.650632\du}}
\pgfpathcurveto{\pgfpoint{42.095612\du}{43.640568\du}}{\pgfpoint{42.081921\du}{43.631229\du}}{\pgfpoint{42.059827\du}{43.619503\du}}
\pgfpathlineto{\pgfpoint{42.059827\du}{43.619503\du}}
\pgfusepath{fill}
\definecolor{dialinecolor}{rgb}{1.000000, 1.000000, 1.000000}
\pgfsetfillcolor{dialinecolor}
\pgfpathmoveto{\pgfpoint{41.943439\du}{43.613670\du}}
\pgfpathcurveto{\pgfpoint{41.912007\du}{43.615635\du}}{\pgfpoint{41.889703\du}{43.623825\du}}{\pgfpoint{41.872687\du}{43.638967\du}}
\pgfpathcurveto{\pgfpoint{41.856397\du}{43.657947\du}}{\pgfpoint{41.848509\du}{43.679315\du}}{\pgfpoint{41.848297\du}{43.699232\du}}
\pgfpathcurveto{\pgfpoint{41.849537\du}{43.726825\du}}{\pgfpoint{41.857002\du}{43.745292\du}}{\pgfpoint{41.871418\du}{43.758469\du}}
\pgfpathcurveto{\pgfpoint{41.894237\du}{43.774033\du}}{\pgfpoint{41.919443\du}{43.781196\du}}{\pgfpoint{41.947036\du}{43.779957\du}}
\pgfusepath{fill}
\definecolor{dialinecolor}{rgb}{0.000000, 0.000000, 0.000000}
\pgfsetfillcolor{dialinecolor}
\pgfpathmoveto{\pgfpoint{41.711114\du}{43.415076\du}}
\pgfpathlineto{\pgfpoint{41.607691\du}{43.414744\du}}
\pgfpathlineto{\pgfpoint{41.605847\du}{43.299806\du}}
\pgfpathlineto{\pgfpoint{41.709270\du}{43.300138\du}}
\pgfpathlineto{\pgfpoint{41.711114\du}{43.415076\du}}
\pgfusepath{fill}
\definecolor{dialinecolor}{rgb}{0.000000, 0.000000, 0.000000}
\pgfsetfillcolor{dialinecolor}
\pgfpathmoveto{\pgfpoint{42.191508\du}{43.411750\du}}
\pgfpathlineto{\pgfpoint{41.770140\du}{43.411872\du}}
\pgfpathlineto{\pgfpoint{41.769021\du}{43.300773\du}}
\pgfpathlineto{\pgfpoint{42.190389\du}{43.300650\du}}
\pgfpathlineto{\pgfpoint{42.191508\du}{43.411750\du}}
\pgfusepath{fill}
\definecolor{dialinecolor}{rgb}{0.000000, 0.000000, 0.000000}
\pgfsetfillcolor{dialinecolor}
\pgfpathmoveto{\pgfpoint{41.770742\du}{42.994342\du}}
\pgfpathlineto{\pgfpoint{41.858087\du}{42.993738\du}}
\pgfpathlineto{\pgfpoint{41.860354\du}{43.068842\du}}
\pgfpathlineto{\pgfpoint{42.027366\du}{43.069083\du}}
\pgfpathcurveto{\pgfpoint{42.064087\du}{43.074069\du}}{\pgfpoint{42.087842\du}{43.073555\du}}{\pgfpoint{42.090955\du}{43.068992\du}}
\pgfpathcurveto{\pgfpoint{42.098631\du}{43.067541\du}}{\pgfpoint{42.101744\du}{43.062977\du}}{\pgfpoint{42.104857\du}{43.058414\du}}
\pgfpathcurveto{\pgfpoint{42.107970\du}{43.053850\du}}{\pgfpoint{42.111083\du}{43.049286\du}}{\pgfpoint{42.108907\du}{43.037771\du}}
\pgfpathcurveto{\pgfpoint{42.115133\du}{43.028644\du}}{\pgfpoint{42.108393\du}{43.014016\du}}{\pgfpoint{42.096364\du}{42.992437\du}}
\pgfpathlineto{\pgfpoint{42.186096\du}{42.983430\du}}
\pgfpathcurveto{\pgfpoint{42.191899\du}{43.014137\du}}{\pgfpoint{42.197702\du}{43.044843\du}}{\pgfpoint{42.202780\du}{43.071711\du}}
\pgfpathcurveto{\pgfpoint{42.199455\du}{43.096192\du}}{\pgfpoint{42.195406\du}{43.116834\du}}{\pgfpoint{42.189905\du}{43.129800\du}}
\pgfpathcurveto{\pgfpoint{42.181292\du}{43.147329\du}}{\pgfpoint{42.178904\du}{43.155731\du}}{\pgfpoint{42.168114\du}{43.161746\du}}
\pgfpathcurveto{\pgfpoint{42.161889\du}{43.170873\du}}{\pgfpoint{42.143422\du}{43.178338\du}}{\pgfpoint{42.119667\du}{43.178852\du}}
\pgfpathcurveto{\pgfpoint{42.108878\du}{43.184866\du}}{\pgfpoint{42.085122\du}{43.185380\du}}{\pgfpoint{42.044563\du}{43.181119\du}}
\pgfpathlineto{\pgfpoint{41.857634\du}{43.180667\du}}
\pgfpathlineto{\pgfpoint{41.859387\du}{43.232015\du}}
\pgfpathlineto{\pgfpoint{41.768205\du}{43.233345\du}}
\pgfpathlineto{\pgfpoint{41.770290\du}{43.181271\du}}
\pgfpathlineto{\pgfpoint{41.686784\du}{43.181151\du}}
\pgfpathlineto{\pgfpoint{41.622076\du}{43.070142\du}}
\pgfpathlineto{\pgfpoint{41.769171\du}{43.070172\du}}
\pgfpathlineto{\pgfpoint{41.770742\du}{42.994342\du}}
\pgfusepath{fill}
\definecolor{dialinecolor}{rgb}{0.000000, 0.000000, 0.000000}
\pgfsetstrokecolor{dialinecolor}
\pgfpathmoveto{\pgfpoint{42.190061\du}{44.413822\du}}
\pgfpathlineto{\pgfpoint{42.086639\du}{44.413490\du}}
\pgfpathlineto{\pgfpoint{41.706825\du}{44.107605\du}}
\pgfpathlineto{\pgfpoint{41.706252\du}{44.378040\du}}
\pgfpathlineto{\pgfpoint{41.606668\du}{44.376982\du}}
\pgfpathlineto{\pgfpoint{41.609658\du}{43.951050\du}}
\pgfpathlineto{\pgfpoint{41.700841\du}{43.949720\du}}
\pgfpathlineto{\pgfpoint{42.091232\du}{44.269508\du}}
\pgfpathlineto{\pgfpoint{42.092439\du}{43.939322\du}}
\pgfpathlineto{\pgfpoint{42.192024\du}{43.940379\du}}
\pgfpathlineto{\pgfpoint{42.190061\du}{44.413822\du}}
\pgfusepath{stroke}
\definecolor{dialinecolor}{rgb}{0.000000, 0.000000, 0.000000}
\pgfsetstrokecolor{dialinecolor}
\pgfpathmoveto{\pgfpoint{42.059827\du}{43.619503\du}}
\pgfpathlineto{\pgfpoint{42.078626\du}{43.508615\du}}
\pgfpathcurveto{\pgfpoint{42.118248\du}{43.528955\du}}{\pgfpoint{42.146356\du}{43.551471\du}}{\pgfpoint{42.171350\du}{43.578550\du}}
\pgfpathcurveto{\pgfpoint{42.189393\du}{43.610919\du}}{\pgfpoint{42.200485\du}{43.648577\du}}{\pgfpoint{42.200788\du}{43.692249\du}}
\pgfpathcurveto{\pgfpoint{42.199216\du}{43.768078\du}}{\pgfpoint{42.177940\du}{43.823779\du}}{\pgfpoint{42.132394\du}{43.856239\du}}
\pgfpathcurveto{\pgfpoint{42.094525\du}{43.887248\du}}{\pgfpoint{42.045352\du}{43.900516\du}}{\pgfpoint{41.985601\du}{43.899881\du}}
\pgfpathcurveto{\pgfpoint{41.918174\du}{43.900698\du}}{\pgfpoint{41.867036\du}{43.882534\du}}{\pgfpoint{41.823787\du}{43.843003\du}}
\pgfpathcurveto{\pgfpoint{41.785827\du}{43.810422\du}}{\pgfpoint{41.764882\du}{43.762700\du}}{\pgfpoint{41.761678\du}{43.703675\du}}
\pgfpathcurveto{\pgfpoint{41.766151\du}{43.643199\du}}{\pgfpoint{41.785040\du}{43.595900\du}}{\pgfpoint{41.825297\du}{43.556489\du}}
\pgfpathcurveto{\pgfpoint{41.867005\du}{43.524755\du}}{\pgfpoint{41.930805\du}{43.504747\du}}{\pgfpoint{42.014311\du}{43.504867\du}}
\pgfpathlineto{\pgfpoint{42.015189\du}{43.782979\du}}
\pgfpathcurveto{\pgfpoint{42.051184\du}{43.784127\du}}{\pgfpoint{42.077327\du}{43.775211\du}}{\pgfpoint{42.093617\du}{43.756231\du}}
\pgfpathcurveto{\pgfpoint{42.115196\du}{43.744203\du}}{\pgfpoint{42.123085\du}{43.722835\du}}{\pgfpoint{42.117282\du}{43.692128\du}}
\pgfpathcurveto{\pgfpoint{42.122782\du}{43.679163\du}}{\pgfpoint{42.116042\du}{43.664535\du}}{\pgfpoint{42.105464\du}{43.650632\du}}
\pgfpathcurveto{\pgfpoint{42.095612\du}{43.640568\du}}{\pgfpoint{42.081921\du}{43.631229\du}}{\pgfpoint{42.059827\du}{43.619503\du}}
\pgfpathlineto{\pgfpoint{42.059827\du}{43.619503\du}}
\pgfusepath{stroke}
\definecolor{dialinecolor}{rgb}{0.000000, 0.000000, 0.000000}
\pgfsetstrokecolor{dialinecolor}
\pgfpathmoveto{\pgfpoint{41.943439\du}{43.613670\du}}
\pgfpathcurveto{\pgfpoint{41.912007\du}{43.615635\du}}{\pgfpoint{41.889703\du}{43.623825\du}}{\pgfpoint{41.872687\du}{43.638967\du}}
\pgfpathcurveto{\pgfpoint{41.856397\du}{43.657947\du}}{\pgfpoint{41.848509\du}{43.679315\du}}{\pgfpoint{41.848297\du}{43.699232\du}}
\pgfpathcurveto{\pgfpoint{41.849537\du}{43.726825\du}}{\pgfpoint{41.857002\du}{43.745292\du}}{\pgfpoint{41.871418\du}{43.758469\du}}
\pgfpathcurveto{\pgfpoint{41.894237\du}{43.774033\du}}{\pgfpoint{41.919443\du}{43.781196\du}}{\pgfpoint{41.947036\du}{43.779957\du}}
\pgfpathlineto{\pgfpoint{41.943439\du}{43.613670\du}}
\pgfusepath{stroke}
\definecolor{dialinecolor}{rgb}{0.000000, 0.000000, 0.000000}
\pgfsetstrokecolor{dialinecolor}
\pgfpathmoveto{\pgfpoint{41.711114\du}{43.415076\du}}
\pgfpathlineto{\pgfpoint{41.607691\du}{43.414744\du}}
\pgfpathlineto{\pgfpoint{41.605847\du}{43.299806\du}}
\pgfpathlineto{\pgfpoint{41.709270\du}{43.300138\du}}
\pgfpathlineto{\pgfpoint{41.711114\du}{43.415076\du}}
\pgfusepath{stroke}
\definecolor{dialinecolor}{rgb}{0.000000, 0.000000, 0.000000}
\pgfsetstrokecolor{dialinecolor}
\pgfpathmoveto{\pgfpoint{42.191508\du}{43.411750\du}}
\pgfpathlineto{\pgfpoint{41.770140\du}{43.411872\du}}
\pgfpathlineto{\pgfpoint{41.769021\du}{43.300773\du}}
\pgfpathlineto{\pgfpoint{42.190389\du}{43.300650\du}}
\pgfpathlineto{\pgfpoint{42.191508\du}{43.411750\du}}
\pgfusepath{stroke}
\definecolor{dialinecolor}{rgb}{0.000000, 0.000000, 0.000000}
\pgfsetstrokecolor{dialinecolor}
\pgfpathmoveto{\pgfpoint{41.770742\du}{42.994342\du}}
\pgfpathlineto{\pgfpoint{41.858087\du}{42.993738\du}}
\pgfpathlineto{\pgfpoint{41.860354\du}{43.068842\du}}
\pgfpathlineto{\pgfpoint{42.027366\du}{43.069083\du}}
\pgfpathcurveto{\pgfpoint{42.064087\du}{43.074069\du}}{\pgfpoint{42.087842\du}{43.073555\du}}{\pgfpoint{42.090955\du}{43.068992\du}}
\pgfpathcurveto{\pgfpoint{42.098631\du}{43.067541\du}}{\pgfpoint{42.101744\du}{43.062977\du}}{\pgfpoint{42.104857\du}{43.058414\du}}
\pgfpathcurveto{\pgfpoint{42.107970\du}{43.053850\du}}{\pgfpoint{42.111083\du}{43.049286\du}}{\pgfpoint{42.108907\du}{43.037771\du}}
\pgfpathcurveto{\pgfpoint{42.115133\du}{43.028644\du}}{\pgfpoint{42.108393\du}{43.014016\du}}{\pgfpoint{42.096364\du}{42.992437\du}}
\pgfpathlineto{\pgfpoint{42.186096\du}{42.983430\du}}
\pgfpathcurveto{\pgfpoint{42.191899\du}{43.014137\du}}{\pgfpoint{42.197702\du}{43.044843\du}}{\pgfpoint{42.202780\du}{43.071711\du}}
\pgfpathcurveto{\pgfpoint{42.199455\du}{43.096192\du}}{\pgfpoint{42.195406\du}{43.116834\du}}{\pgfpoint{42.189905\du}{43.129800\du}}
\pgfpathcurveto{\pgfpoint{42.181292\du}{43.147329\du}}{\pgfpoint{42.178904\du}{43.155731\du}}{\pgfpoint{42.168114\du}{43.161746\du}}
\pgfpathcurveto{\pgfpoint{42.161889\du}{43.170873\du}}{\pgfpoint{42.143422\du}{43.178338\du}}{\pgfpoint{42.119667\du}{43.178852\du}}
\pgfpathcurveto{\pgfpoint{42.108878\du}{43.184866\du}}{\pgfpoint{42.085122\du}{43.185380\du}}{\pgfpoint{42.044563\du}{43.181119\du}}
\pgfpathlineto{\pgfpoint{41.857634\du}{43.180667\du}}
\pgfpathlineto{\pgfpoint{41.859387\du}{43.232015\du}}
\pgfpathlineto{\pgfpoint{41.768205\du}{43.233345\du}}
\pgfpathlineto{\pgfpoint{41.770290\du}{43.181271\du}}
\pgfpathlineto{\pgfpoint{41.686784\du}{43.181151\du}}
\pgfpathlineto{\pgfpoint{41.622076\du}{43.070142\du}}
\pgfpathlineto{\pgfpoint{41.769171\du}{43.070172\du}}
\pgfpathlineto{\pgfpoint{41.770742\du}{42.994342\du}}
\pgfusepath{stroke}
\pgfsetlinewidth{0.000000\du}
\pgfsetdash{}{0pt}
\pgfsetmiterjoin
\pgfsetroundcap
\definecolor{dialinecolor}{rgb}{0.000000, 0.000000, 0.000000}
\pgfsetfillcolor{dialinecolor}
\pgfpathmoveto{\pgfpoint{50.671056\du}{19.478497\du}}
\pgfpathlineto{\pgfpoint{50.534337\du}{18.896465\du}}
\pgfpathlineto{\pgfpoint{50.655431\du}{18.896465\du}}
\pgfpathlineto{\pgfpoint{50.741369\du}{19.294903\du}}
\pgfpathlineto{\pgfpoint{50.846837\du}{18.896465\du}}
\pgfpathlineto{\pgfpoint{50.987462\du}{18.896465\du}}
\pgfpathlineto{\pgfpoint{51.089025\du}{19.302715\du}}
\pgfpathlineto{\pgfpoint{51.178869\du}{18.896465\du}}
\pgfpathlineto{\pgfpoint{51.299962\du}{18.896465\du}}
\pgfpathlineto{\pgfpoint{51.155431\du}{19.478497\du}}
\pgfpathlineto{\pgfpoint{51.030431\du}{19.478497\du}}
\pgfpathlineto{\pgfpoint{50.917150\du}{19.040997\du}}
\pgfpathlineto{\pgfpoint{50.799962\du}{19.478497\du}}
\pgfpathlineto{\pgfpoint{50.671056\du}{19.478497\du}}
\pgfusepath{fill}
\definecolor{dialinecolor}{rgb}{0.000000, 0.000000, 0.000000}
\pgfsetfillcolor{dialinecolor}
\pgfpathmoveto{\pgfpoint{51.331212\du}{19.259747\du}}
\pgfpathcurveto{\pgfpoint{51.327306\du}{19.228497\du}}{\pgfpoint{51.339025\du}{19.193340\du}}{\pgfpoint{51.358556\du}{19.154278\du}}
\pgfpathcurveto{\pgfpoint{51.374181\du}{19.123028\du}}{\pgfpoint{51.397619\du}{19.095684\du}}{\pgfpoint{51.436681\du}{19.076153\du}}
\pgfpathcurveto{\pgfpoint{51.467931\du}{19.056622\du}}{\pgfpoint{51.506994\du}{19.044903\du}}{\pgfpoint{51.549962\du}{19.044903\du}}
\pgfpathcurveto{\pgfpoint{51.612462\du}{19.044903\du}}{\pgfpoint{51.663244\du}{19.068340\du}}{\pgfpoint{51.706212\du}{19.107403\du}}
\pgfpathcurveto{\pgfpoint{51.745275\du}{19.154278\du}}{\pgfpoint{51.768712\du}{19.205059\du}}{\pgfpoint{51.768712\du}{19.267559\du}}
\pgfpathcurveto{\pgfpoint{51.768712\du}{19.330059\du}}{\pgfpoint{51.745275\du}{19.384747\du}}{\pgfpoint{51.706212\du}{19.423809\du}}
\pgfpathcurveto{\pgfpoint{51.659337\du}{19.466778\du}}{\pgfpoint{51.608556\du}{19.482403\du}}{\pgfpoint{51.549962\du}{19.486309\du}}
\pgfpathcurveto{\pgfpoint{51.506994\du}{19.482403\du}}{\pgfpoint{51.467931\du}{19.478497\du}}{\pgfpoint{51.436681\du}{19.462872\du}}
\pgfpathcurveto{\pgfpoint{51.397619\du}{19.447247\du}}{\pgfpoint{51.374181\du}{19.419903\du}}{\pgfpoint{51.358556\du}{19.384747\du}}
\pgfpathcurveto{\pgfpoint{51.339025\du}{19.353497\du}}{\pgfpoint{51.327306\du}{19.314434\du}}{\pgfpoint{51.331212\du}{19.259747\du}}
\pgfpathlineto{\pgfpoint{51.331212\du}{19.259747\du}}
\pgfusepath{fill}
\definecolor{dialinecolor}{rgb}{1.000000, 1.000000, 1.000000}
\pgfsetfillcolor{dialinecolor}
\pgfpathmoveto{\pgfpoint{51.448400\du}{19.267559\du}}
\pgfpathcurveto{\pgfpoint{51.444494\du}{19.310528\du}}{\pgfpoint{51.456212\du}{19.345684\du}}{\pgfpoint{51.475744\du}{19.365215\du}}
\pgfpathcurveto{\pgfpoint{51.495275\du}{19.388653\du}}{\pgfpoint{51.518712\du}{19.396465\du}}{\pgfpoint{51.549962\du}{19.396465\du}}
\pgfpathcurveto{\pgfpoint{51.577306\du}{19.396465\du}}{\pgfpoint{51.600744\du}{19.388653\du}}{\pgfpoint{51.620275\du}{19.365215\du}}
\pgfpathcurveto{\pgfpoint{51.639806\du}{19.345684\du}}{\pgfpoint{51.651525\du}{19.310528\du}}{\pgfpoint{51.651525\du}{19.267559\du}}
\pgfpathcurveto{\pgfpoint{51.651525\du}{19.228497\du}}{\pgfpoint{51.639806\du}{19.197247\du}}{\pgfpoint{51.620275\du}{19.173809\du}}
\pgfpathcurveto{\pgfpoint{51.600744\du}{19.150372\du}}{\pgfpoint{51.577306\du}{19.138653\du}}{\pgfpoint{51.549962\du}{19.138653\du}}
\pgfpathcurveto{\pgfpoint{51.518712\du}{19.138653\du}}{\pgfpoint{51.495275\du}{19.150372\du}}{\pgfpoint{51.475744\du}{19.173809\du}}
\pgfpathcurveto{\pgfpoint{51.456212\du}{19.197247\du}}{\pgfpoint{51.444494\du}{19.228497\du}}{\pgfpoint{51.448400\du}{19.267559\du}}
\pgfusepath{fill}
\definecolor{dialinecolor}{rgb}{0.000000, 0.000000, 0.000000}
\pgfsetfillcolor{dialinecolor}
\pgfpathmoveto{\pgfpoint{51.960119\du}{19.478497\du}}
\pgfpathlineto{\pgfpoint{51.850744\du}{19.478497\du}}
\pgfpathlineto{\pgfpoint{51.850744\du}{19.056622\du}}
\pgfpathlineto{\pgfpoint{51.952306\du}{19.056622\du}}
\pgfpathlineto{\pgfpoint{51.952306\du}{19.115215\du}}
\pgfpathcurveto{\pgfpoint{51.967931\du}{19.087872\du}}{\pgfpoint{51.983556\du}{19.072247\du}}{\pgfpoint{51.999181\du}{19.060528\du}}
\pgfpathcurveto{\pgfpoint{52.010900\du}{19.052715\du}}{\pgfpoint{52.026525\du}{19.044903\du}}{\pgfpoint{52.049962\du}{19.044903\du}}
\pgfpathcurveto{\pgfpoint{52.073400\du}{19.044903\du}}{\pgfpoint{52.100744\du}{19.052715\du}}{\pgfpoint{52.124181\du}{19.068340\du}}
\pgfpathlineto{\pgfpoint{52.089025\du}{19.165997\du}}
\pgfpathcurveto{\pgfpoint{52.069494\du}{19.158184\du}}{\pgfpoint{52.049962\du}{19.150372\du}}{\pgfpoint{52.034337\du}{19.146465\du}}
\pgfpathcurveto{\pgfpoint{52.014806\du}{19.150372\du}}{\pgfpoint{52.003087\du}{19.154278\du}}{\pgfpoint{51.995275\du}{19.162090\du}}
\pgfpathcurveto{\pgfpoint{51.983556\du}{19.169903\du}}{\pgfpoint{51.971837\du}{19.185528\du}}{\pgfpoint{51.967931\du}{19.208965\du}}
\pgfpathcurveto{\pgfpoint{51.960119\du}{19.232403\du}}{\pgfpoint{51.960119\du}{19.279278\du}}{\pgfpoint{51.960119\du}{19.349590\du}}
\pgfpathlineto{\pgfpoint{51.960119\du}{19.478497\du}}
\pgfusepath{fill}
\definecolor{dialinecolor}{rgb}{0.000000, 0.000000, 0.000000}
\pgfsetfillcolor{dialinecolor}
\pgfpathmoveto{\pgfpoint{52.167150\du}{19.478497\du}}
\pgfpathlineto{\pgfpoint{52.167150\du}{18.896465\du}}
\pgfpathlineto{\pgfpoint{52.280431\du}{18.896465\du}}
\pgfpathlineto{\pgfpoint{52.280431\du}{19.205059\du}}
\pgfpathlineto{\pgfpoint{52.409337\du}{19.056622\du}}
\pgfpathlineto{\pgfpoint{52.546056\du}{19.056622\du}}
\pgfpathlineto{\pgfpoint{52.401525\du}{19.208965\du}}
\pgfpathlineto{\pgfpoint{52.557775\du}{19.478497\du}}
\pgfpathlineto{\pgfpoint{52.436681\du}{19.478497\du}}
\pgfpathlineto{\pgfpoint{52.331212\du}{19.287090\du}}
\pgfpathlineto{\pgfpoint{52.280431\du}{19.341778\du}}
\pgfpathlineto{\pgfpoint{52.280431\du}{19.478497\du}}
\pgfpathlineto{\pgfpoint{52.167150\du}{19.478497\du}}
\pgfusepath{fill}
\definecolor{dialinecolor}{rgb}{0.000000, 0.000000, 0.000000}
\pgfsetfillcolor{dialinecolor}
\pgfpathmoveto{\pgfpoint{52.846837\du}{19.478497\du}}
\pgfpathlineto{\pgfpoint{52.846837\du}{18.896465\du}}
\pgfpathlineto{\pgfpoint{52.964025\du}{18.896465\du}}
\pgfpathlineto{\pgfpoint{52.964025\du}{19.478497\du}}
\pgfpathlineto{\pgfpoint{52.846837\du}{19.478497\du}}
\pgfusepath{fill}
\definecolor{dialinecolor}{rgb}{0.000000, 0.000000, 0.000000}
\pgfsetfillcolor{dialinecolor}
\pgfpathmoveto{\pgfpoint{53.272619\du}{19.056622\du}}
\pgfpathlineto{\pgfpoint{53.272619\du}{19.146465\du}}
\pgfpathlineto{\pgfpoint{53.194494\du}{19.146465\du}}
\pgfpathlineto{\pgfpoint{53.194494\du}{19.314434\du}}
\pgfpathcurveto{\pgfpoint{53.194494\du}{19.353497\du}}{\pgfpoint{53.194494\du}{19.376934\du}}{\pgfpoint{53.194494\du}{19.376934\du}}
\pgfpathcurveto{\pgfpoint{53.194494\du}{19.384747\du}}{\pgfpoint{53.198400\du}{19.388653\du}}{\pgfpoint{53.206212\du}{19.392559\du}}
\pgfpathcurveto{\pgfpoint{53.210119\du}{19.396465\du}}{\pgfpoint{53.214025\du}{19.396465\du}}{\pgfpoint{53.225744\du}{19.396465\du}}
\pgfpathcurveto{\pgfpoint{53.233556\du}{19.396465\du}}{\pgfpoint{53.249181\du}{19.392559\du}}{\pgfpoint{53.268712\du}{19.384747\du}}
\pgfpathlineto{\pgfpoint{53.280431\du}{19.470684\du}}
\pgfpathcurveto{\pgfpoint{53.253087\du}{19.482403\du}}{\pgfpoint{53.221837\du}{19.482403\du}}{\pgfpoint{53.190587\du}{19.486309\du}}
\pgfpathcurveto{\pgfpoint{53.167150\du}{19.482403\du}}{\pgfpoint{53.151525\du}{19.482403\du}}{\pgfpoint{53.135900\du}{19.478497\du}}
\pgfpathcurveto{\pgfpoint{53.120275\du}{19.470684\du}}{\pgfpoint{53.108556\du}{19.462872\du}}{\pgfpoint{53.100744\du}{19.451153\du}}
\pgfpathcurveto{\pgfpoint{53.092931\du}{19.443340\du}}{\pgfpoint{53.085119\du}{19.427715\du}}{\pgfpoint{53.085119\du}{19.408184\du}}
\pgfpathcurveto{\pgfpoint{53.077306\du}{19.396465\du}}{\pgfpoint{53.077306\du}{19.373028\du}}{\pgfpoint{53.081212\du}{19.330059\du}}
\pgfpathlineto{\pgfpoint{53.081212\du}{19.146465\du}}
\pgfpathlineto{\pgfpoint{53.030431\du}{19.146465\du}}
\pgfpathlineto{\pgfpoint{53.030431\du}{19.056622\du}}
\pgfpathlineto{\pgfpoint{53.081212\du}{19.056622\du}}
\pgfpathlineto{\pgfpoint{53.081212\du}{18.970684\du}}
\pgfpathlineto{\pgfpoint{53.194494\du}{18.908184\du}}
\pgfpathlineto{\pgfpoint{53.194494\du}{19.056622\du}}
\pgfpathlineto{\pgfpoint{53.272619\du}{19.056622\du}}
\pgfusepath{fill}
\definecolor{dialinecolor}{rgb}{0.000000, 0.000000, 0.000000}
\pgfsetfillcolor{dialinecolor}
\pgfpathmoveto{\pgfpoint{53.592931\du}{19.345684\du}}
\pgfpathlineto{\pgfpoint{53.702306\du}{19.361309\du}}
\pgfpathcurveto{\pgfpoint{53.686681\du}{19.404278\du}}{\pgfpoint{53.663244\du}{19.435528\du}}{\pgfpoint{53.635900\du}{19.455059\du}}
\pgfpathcurveto{\pgfpoint{53.600744\du}{19.474590\du}}{\pgfpoint{53.561681\du}{19.482403\du}}{\pgfpoint{53.518712\du}{19.486309\du}}
\pgfpathcurveto{\pgfpoint{53.440587\du}{19.482403\du}}{\pgfpoint{53.385900\du}{19.462872\du}}{\pgfpoint{53.354650\du}{19.415997\du}}
\pgfpathcurveto{\pgfpoint{53.323400\du}{19.380840\du}}{\pgfpoint{53.311681\du}{19.333965\du}}{\pgfpoint{53.315587\du}{19.271465\du}}
\pgfpathcurveto{\pgfpoint{53.311681\du}{19.205059\du}}{\pgfpoint{53.331212\du}{19.150372\du}}{\pgfpoint{53.370275\du}{19.107403\du}}
\pgfpathcurveto{\pgfpoint{53.405431\du}{19.068340\du}}{\pgfpoint{53.448400\du}{19.044903\du}}{\pgfpoint{53.506994\du}{19.044903\du}}
\pgfpathcurveto{\pgfpoint{53.565587\du}{19.044903\du}}{\pgfpoint{53.616369\du}{19.068340\du}}{\pgfpoint{53.655431\du}{19.107403\du}}
\pgfpathcurveto{\pgfpoint{53.690587\du}{19.154278\du}}{\pgfpoint{53.710119\du}{19.216778\du}}{\pgfpoint{53.710119\du}{19.298809\du}}
\pgfpathlineto{\pgfpoint{53.428869\du}{19.298809\du}}
\pgfpathcurveto{\pgfpoint{53.428869\du}{19.333965\du}}{\pgfpoint{53.436681\du}{19.361309\du}}{\pgfpoint{53.452306\du}{19.376934\du}}
\pgfpathcurveto{\pgfpoint{53.467931\du}{19.400372\du}}{\pgfpoint{53.487462\du}{19.408184\du}}{\pgfpoint{53.518712\du}{19.404278\du}}
\pgfpathcurveto{\pgfpoint{53.534337\du}{19.408184\du}}{\pgfpoint{53.549962\du}{19.404278\du}}{\pgfpoint{53.561681\du}{19.392559\du}}
\pgfpathcurveto{\pgfpoint{53.573400\du}{19.384747\du}}{\pgfpoint{53.585119\du}{19.369122\du}}{\pgfpoint{53.592931\du}{19.345684\du}}
\pgfpathlineto{\pgfpoint{53.592931\du}{19.345684\du}}
\pgfusepath{fill}
\definecolor{dialinecolor}{rgb}{1.000000, 1.000000, 1.000000}
\pgfsetfillcolor{dialinecolor}
\pgfpathmoveto{\pgfpoint{53.596837\du}{19.232403\du}}
\pgfpathcurveto{\pgfpoint{53.592931\du}{19.201153\du}}{\pgfpoint{53.585119\du}{19.177715\du}}{\pgfpoint{53.573400\du}{19.158184\du}}
\pgfpathcurveto{\pgfpoint{53.553869\du}{19.142559\du}}{\pgfpoint{53.534337\du}{19.134747\du}}{\pgfpoint{53.514806\du}{19.130840\du}}
\pgfpathcurveto{\pgfpoint{53.487462\du}{19.134747\du}}{\pgfpoint{53.467931\du}{19.142559\du}}{\pgfpoint{53.452306\du}{19.158184\du}}
\pgfpathcurveto{\pgfpoint{53.436681\du}{19.177715\du}}{\pgfpoint{53.428869\du}{19.201153\du}}{\pgfpoint{53.428869\du}{19.232403\du}}
\pgfusepath{fill}
\definecolor{dialinecolor}{rgb}{0.000000, 0.000000, 0.000000}
\pgfsetfillcolor{dialinecolor}
\pgfpathmoveto{\pgfpoint{53.792150\du}{19.056622\du}}
\pgfpathlineto{\pgfpoint{53.893712\du}{19.056622\du}}
\pgfpathlineto{\pgfpoint{53.893712\du}{19.115215\du}}
\pgfpathcurveto{\pgfpoint{53.928869\du}{19.068340\du}}{\pgfpoint{53.971837\du}{19.044903\du}}{\pgfpoint{54.026525\du}{19.044903\du}}
\pgfpathcurveto{\pgfpoint{54.049962\du}{19.044903\du}}{\pgfpoint{54.073400\du}{19.052715\du}}{\pgfpoint{54.096837\du}{19.064434\du}}
\pgfpathcurveto{\pgfpoint{54.116369\du}{19.076153\du}}{\pgfpoint{54.131994\du}{19.091778\du}}{\pgfpoint{54.143712\du}{19.115215\du}}
\pgfpathcurveto{\pgfpoint{54.163244\du}{19.091778\du}}{\pgfpoint{54.182775\du}{19.076153\du}}{\pgfpoint{54.206212\du}{19.064434\du}}
\pgfpathcurveto{\pgfpoint{54.221837\du}{19.052715\du}}{\pgfpoint{54.245275\du}{19.044903\du}}{\pgfpoint{54.272619\du}{19.044903\du}}
\pgfpathcurveto{\pgfpoint{54.299962\du}{19.044903\du}}{\pgfpoint{54.327306\du}{19.052715\du}}{\pgfpoint{54.350744\du}{19.064434\du}}
\pgfpathcurveto{\pgfpoint{54.374181\du}{19.080059\du}}{\pgfpoint{54.389806\du}{19.099590\du}}{\pgfpoint{54.401525\du}{19.123028\du}}
\pgfpathcurveto{\pgfpoint{54.405431\du}{19.142559\du}}{\pgfpoint{54.409337\du}{19.169903\du}}{\pgfpoint{54.413244\du}{19.208965\du}}
\pgfpathlineto{\pgfpoint{54.413244\du}{19.478497\du}}
\pgfpathlineto{\pgfpoint{54.299962\du}{19.478497\du}}
\pgfpathlineto{\pgfpoint{54.299962\du}{19.236309\du}}
\pgfpathcurveto{\pgfpoint{54.296056\du}{19.197247\du}}{\pgfpoint{54.292150\du}{19.169903\du}}{\pgfpoint{54.288244\du}{19.154278\du}}
\pgfpathcurveto{\pgfpoint{54.272619\du}{19.142559\du}}{\pgfpoint{54.256994\du}{19.134747\du}}{\pgfpoint{54.241369\du}{19.130840\du}}
\pgfpathcurveto{\pgfpoint{54.221837\du}{19.134747\du}}{\pgfpoint{54.206212\du}{19.138653\du}}{\pgfpoint{54.194494\du}{19.146465\du}}
\pgfpathcurveto{\pgfpoint{54.178869\du}{19.158184\du}}{\pgfpoint{54.171056\du}{19.173809\du}}{\pgfpoint{54.167150\du}{19.189434\du}}
\pgfpathcurveto{\pgfpoint{54.159337\du}{19.212872\du}}{\pgfpoint{54.155431\du}{19.240215\du}}{\pgfpoint{54.159337\du}{19.275372\du}}
\pgfpathlineto{\pgfpoint{54.159337\du}{19.478497\du}}
\pgfpathlineto{\pgfpoint{54.046056\du}{19.478497\du}}
\pgfpathlineto{\pgfpoint{54.046056\du}{19.248028\du}}
\pgfpathcurveto{\pgfpoint{54.046056\du}{19.208965\du}}{\pgfpoint{54.042150\du}{19.181622\du}}{\pgfpoint{54.038244\du}{19.169903\du}}
\pgfpathcurveto{\pgfpoint{54.034337\du}{19.158184\du}}{\pgfpoint{54.030431\du}{19.150372\du}}{\pgfpoint{54.022619\du}{19.142559\du}}
\pgfpathcurveto{\pgfpoint{54.014806\du}{19.138653\du}}{\pgfpoint{54.003087\du}{19.134747\du}}{\pgfpoint{53.987462\du}{19.130840\du}}
\pgfpathcurveto{\pgfpoint{53.971837\du}{19.134747\du}}{\pgfpoint{53.956212\du}{19.138653\du}}{\pgfpoint{53.940587\du}{19.146465\du}}
\pgfpathcurveto{\pgfpoint{53.924962\du}{19.158184\du}}{\pgfpoint{53.913244\du}{19.169903\du}}{\pgfpoint{53.909337\du}{19.185528\du}}
\pgfpathcurveto{\pgfpoint{53.901525\du}{19.205059\du}}{\pgfpoint{53.897619\du}{19.236309\du}}{\pgfpoint{53.901525\du}{19.271465\du}}
\pgfpathlineto{\pgfpoint{53.901525\du}{19.478497\du}}
\pgfpathlineto{\pgfpoint{53.792150\du}{19.478497\du}}
\pgfpathlineto{\pgfpoint{53.792150\du}{19.056622\du}}
\pgfusepath{fill}
\definecolor{dialinecolor}{rgb}{0.000000, 0.000000, 0.000000}
\pgfsetstrokecolor{dialinecolor}
\pgfpathmoveto{\pgfpoint{50.671056\du}{19.478497\du}}
\pgfpathlineto{\pgfpoint{50.534337\du}{18.896465\du}}
\pgfpathlineto{\pgfpoint{50.655431\du}{18.896465\du}}
\pgfpathlineto{\pgfpoint{50.741369\du}{19.294903\du}}
\pgfpathlineto{\pgfpoint{50.846837\du}{18.896465\du}}
\pgfpathlineto{\pgfpoint{50.987462\du}{18.896465\du}}
\pgfpathlineto{\pgfpoint{51.089025\du}{19.302715\du}}
\pgfpathlineto{\pgfpoint{51.178869\du}{18.896465\du}}
\pgfpathlineto{\pgfpoint{51.299962\du}{18.896465\du}}
\pgfpathlineto{\pgfpoint{51.155431\du}{19.478497\du}}
\pgfpathlineto{\pgfpoint{51.030431\du}{19.478497\du}}
\pgfpathlineto{\pgfpoint{50.917150\du}{19.040997\du}}
\pgfpathlineto{\pgfpoint{50.799962\du}{19.478497\du}}
\pgfpathlineto{\pgfpoint{50.671056\du}{19.478497\du}}
\pgfusepath{stroke}
\definecolor{dialinecolor}{rgb}{0.000000, 0.000000, 0.000000}
\pgfsetstrokecolor{dialinecolor}
\pgfpathmoveto{\pgfpoint{51.331212\du}{19.259747\du}}
\pgfpathcurveto{\pgfpoint{51.327306\du}{19.228497\du}}{\pgfpoint{51.339025\du}{19.193340\du}}{\pgfpoint{51.358556\du}{19.154278\du}}
\pgfpathcurveto{\pgfpoint{51.374181\du}{19.123028\du}}{\pgfpoint{51.397619\du}{19.095684\du}}{\pgfpoint{51.436681\du}{19.076153\du}}
\pgfpathcurveto{\pgfpoint{51.467931\du}{19.056622\du}}{\pgfpoint{51.506994\du}{19.044903\du}}{\pgfpoint{51.549962\du}{19.044903\du}}
\pgfpathcurveto{\pgfpoint{51.612462\du}{19.044903\du}}{\pgfpoint{51.663244\du}{19.068340\du}}{\pgfpoint{51.706212\du}{19.107403\du}}
\pgfpathcurveto{\pgfpoint{51.745275\du}{19.154278\du}}{\pgfpoint{51.768712\du}{19.205059\du}}{\pgfpoint{51.768712\du}{19.267559\du}}
\pgfpathcurveto{\pgfpoint{51.768712\du}{19.330059\du}}{\pgfpoint{51.745275\du}{19.384747\du}}{\pgfpoint{51.706212\du}{19.423809\du}}
\pgfpathcurveto{\pgfpoint{51.659337\du}{19.466778\du}}{\pgfpoint{51.608556\du}{19.482403\du}}{\pgfpoint{51.549962\du}{19.486309\du}}
\pgfpathcurveto{\pgfpoint{51.506994\du}{19.482403\du}}{\pgfpoint{51.467931\du}{19.478497\du}}{\pgfpoint{51.436681\du}{19.462872\du}}
\pgfpathcurveto{\pgfpoint{51.397619\du}{19.447247\du}}{\pgfpoint{51.374181\du}{19.419903\du}}{\pgfpoint{51.358556\du}{19.384747\du}}
\pgfpathcurveto{\pgfpoint{51.339025\du}{19.353497\du}}{\pgfpoint{51.327306\du}{19.314434\du}}{\pgfpoint{51.331212\du}{19.259747\du}}
\pgfpathlineto{\pgfpoint{51.331212\du}{19.259747\du}}
\pgfusepath{stroke}
\definecolor{dialinecolor}{rgb}{0.000000, 0.000000, 0.000000}
\pgfsetstrokecolor{dialinecolor}
\pgfpathmoveto{\pgfpoint{51.448400\du}{19.267559\du}}
\pgfpathcurveto{\pgfpoint{51.444494\du}{19.310528\du}}{\pgfpoint{51.456212\du}{19.345684\du}}{\pgfpoint{51.475744\du}{19.365215\du}}
\pgfpathcurveto{\pgfpoint{51.495275\du}{19.388653\du}}{\pgfpoint{51.518712\du}{19.396465\du}}{\pgfpoint{51.549962\du}{19.396465\du}}
\pgfpathcurveto{\pgfpoint{51.577306\du}{19.396465\du}}{\pgfpoint{51.600744\du}{19.388653\du}}{\pgfpoint{51.620275\du}{19.365215\du}}
\pgfpathcurveto{\pgfpoint{51.639806\du}{19.345684\du}}{\pgfpoint{51.651525\du}{19.310528\du}}{\pgfpoint{51.651525\du}{19.267559\du}}
\pgfpathcurveto{\pgfpoint{51.651525\du}{19.228497\du}}{\pgfpoint{51.639806\du}{19.197247\du}}{\pgfpoint{51.620275\du}{19.173809\du}}
\pgfpathcurveto{\pgfpoint{51.600744\du}{19.150372\du}}{\pgfpoint{51.577306\du}{19.138653\du}}{\pgfpoint{51.549962\du}{19.138653\du}}
\pgfpathcurveto{\pgfpoint{51.518712\du}{19.138653\du}}{\pgfpoint{51.495275\du}{19.150372\du}}{\pgfpoint{51.475744\du}{19.173809\du}}
\pgfpathcurveto{\pgfpoint{51.456212\du}{19.197247\du}}{\pgfpoint{51.444494\du}{19.228497\du}}{\pgfpoint{51.448400\du}{19.267559\du}}
\pgfpathlineto{\pgfpoint{51.448400\du}{19.267559\du}}
\pgfusepath{stroke}
\definecolor{dialinecolor}{rgb}{0.000000, 0.000000, 0.000000}
\pgfsetstrokecolor{dialinecolor}
\pgfpathmoveto{\pgfpoint{51.960119\du}{19.478497\du}}
\pgfpathlineto{\pgfpoint{51.850744\du}{19.478497\du}}
\pgfpathlineto{\pgfpoint{51.850744\du}{19.056622\du}}
\pgfpathlineto{\pgfpoint{51.952306\du}{19.056622\du}}
\pgfpathlineto{\pgfpoint{51.952306\du}{19.115215\du}}
\pgfpathcurveto{\pgfpoint{51.967931\du}{19.087872\du}}{\pgfpoint{51.983556\du}{19.072247\du}}{\pgfpoint{51.999181\du}{19.060528\du}}
\pgfpathcurveto{\pgfpoint{52.010900\du}{19.052715\du}}{\pgfpoint{52.026525\du}{19.044903\du}}{\pgfpoint{52.049962\du}{19.044903\du}}
\pgfpathcurveto{\pgfpoint{52.073400\du}{19.044903\du}}{\pgfpoint{52.100744\du}{19.052715\du}}{\pgfpoint{52.124181\du}{19.068340\du}}
\pgfpathlineto{\pgfpoint{52.089025\du}{19.165997\du}}
\pgfpathcurveto{\pgfpoint{52.069494\du}{19.158184\du}}{\pgfpoint{52.049962\du}{19.150372\du}}{\pgfpoint{52.034337\du}{19.146465\du}}
\pgfpathcurveto{\pgfpoint{52.014806\du}{19.150372\du}}{\pgfpoint{52.003087\du}{19.154278\du}}{\pgfpoint{51.995275\du}{19.162090\du}}
\pgfpathcurveto{\pgfpoint{51.983556\du}{19.169903\du}}{\pgfpoint{51.971837\du}{19.185528\du}}{\pgfpoint{51.967931\du}{19.208965\du}}
\pgfpathcurveto{\pgfpoint{51.960119\du}{19.232403\du}}{\pgfpoint{51.960119\du}{19.279278\du}}{\pgfpoint{51.960119\du}{19.349590\du}}
\pgfpathlineto{\pgfpoint{51.960119\du}{19.478497\du}}
\pgfusepath{stroke}
\definecolor{dialinecolor}{rgb}{0.000000, 0.000000, 0.000000}
\pgfsetstrokecolor{dialinecolor}
\pgfpathmoveto{\pgfpoint{52.167150\du}{19.478497\du}}
\pgfpathlineto{\pgfpoint{52.167150\du}{18.896465\du}}
\pgfpathlineto{\pgfpoint{52.280431\du}{18.896465\du}}
\pgfpathlineto{\pgfpoint{52.280431\du}{19.205059\du}}
\pgfpathlineto{\pgfpoint{52.409337\du}{19.056622\du}}
\pgfpathlineto{\pgfpoint{52.546056\du}{19.056622\du}}
\pgfpathlineto{\pgfpoint{52.401525\du}{19.208965\du}}
\pgfpathlineto{\pgfpoint{52.557775\du}{19.478497\du}}
\pgfpathlineto{\pgfpoint{52.436681\du}{19.478497\du}}
\pgfpathlineto{\pgfpoint{52.331212\du}{19.287090\du}}
\pgfpathlineto{\pgfpoint{52.280431\du}{19.341778\du}}
\pgfpathlineto{\pgfpoint{52.280431\du}{19.478497\du}}
\pgfpathlineto{\pgfpoint{52.167150\du}{19.478497\du}}
\pgfusepath{stroke}
\definecolor{dialinecolor}{rgb}{0.000000, 0.000000, 0.000000}
\pgfsetstrokecolor{dialinecolor}
\pgfpathmoveto{\pgfpoint{52.846837\du}{19.478497\du}}
\pgfpathlineto{\pgfpoint{52.846837\du}{18.896465\du}}
\pgfpathlineto{\pgfpoint{52.964025\du}{18.896465\du}}
\pgfpathlineto{\pgfpoint{52.964025\du}{19.478497\du}}
\pgfpathlineto{\pgfpoint{52.846837\du}{19.478497\du}}
\pgfusepath{stroke}
\definecolor{dialinecolor}{rgb}{0.000000, 0.000000, 0.000000}
\pgfsetstrokecolor{dialinecolor}
\pgfpathmoveto{\pgfpoint{53.272619\du}{19.056622\du}}
\pgfpathlineto{\pgfpoint{53.272619\du}{19.146465\du}}
\pgfpathlineto{\pgfpoint{53.194494\du}{19.146465\du}}
\pgfpathlineto{\pgfpoint{53.194494\du}{19.314434\du}}
\pgfpathcurveto{\pgfpoint{53.194494\du}{19.353497\du}}{\pgfpoint{53.194494\du}{19.376934\du}}{\pgfpoint{53.194494\du}{19.376934\du}}
\pgfpathcurveto{\pgfpoint{53.194494\du}{19.384747\du}}{\pgfpoint{53.198400\du}{19.388653\du}}{\pgfpoint{53.206212\du}{19.392559\du}}
\pgfpathcurveto{\pgfpoint{53.210119\du}{19.396465\du}}{\pgfpoint{53.214025\du}{19.396465\du}}{\pgfpoint{53.225744\du}{19.396465\du}}
\pgfpathcurveto{\pgfpoint{53.233556\du}{19.396465\du}}{\pgfpoint{53.249181\du}{19.392559\du}}{\pgfpoint{53.268712\du}{19.384747\du}}
\pgfpathlineto{\pgfpoint{53.280431\du}{19.470684\du}}
\pgfpathcurveto{\pgfpoint{53.253087\du}{19.482403\du}}{\pgfpoint{53.221837\du}{19.482403\du}}{\pgfpoint{53.190587\du}{19.486309\du}}
\pgfpathcurveto{\pgfpoint{53.167150\du}{19.482403\du}}{\pgfpoint{53.151525\du}{19.482403\du}}{\pgfpoint{53.135900\du}{19.478497\du}}
\pgfpathcurveto{\pgfpoint{53.120275\du}{19.470684\du}}{\pgfpoint{53.108556\du}{19.462872\du}}{\pgfpoint{53.100744\du}{19.451153\du}}
\pgfpathcurveto{\pgfpoint{53.092931\du}{19.443340\du}}{\pgfpoint{53.085119\du}{19.427715\du}}{\pgfpoint{53.085119\du}{19.408184\du}}
\pgfpathcurveto{\pgfpoint{53.077306\du}{19.396465\du}}{\pgfpoint{53.077306\du}{19.373028\du}}{\pgfpoint{53.081212\du}{19.330059\du}}
\pgfpathlineto{\pgfpoint{53.081212\du}{19.146465\du}}
\pgfpathlineto{\pgfpoint{53.030431\du}{19.146465\du}}
\pgfpathlineto{\pgfpoint{53.030431\du}{19.056622\du}}
\pgfpathlineto{\pgfpoint{53.081212\du}{19.056622\du}}
\pgfpathlineto{\pgfpoint{53.081212\du}{18.970684\du}}
\pgfpathlineto{\pgfpoint{53.194494\du}{18.908184\du}}
\pgfpathlineto{\pgfpoint{53.194494\du}{19.056622\du}}
\pgfpathlineto{\pgfpoint{53.272619\du}{19.056622\du}}
\pgfusepath{stroke}
\definecolor{dialinecolor}{rgb}{0.000000, 0.000000, 0.000000}
\pgfsetstrokecolor{dialinecolor}
\pgfpathmoveto{\pgfpoint{53.592931\du}{19.345684\du}}
\pgfpathlineto{\pgfpoint{53.702306\du}{19.361309\du}}
\pgfpathcurveto{\pgfpoint{53.686681\du}{19.404278\du}}{\pgfpoint{53.663244\du}{19.435528\du}}{\pgfpoint{53.635900\du}{19.455059\du}}
\pgfpathcurveto{\pgfpoint{53.600744\du}{19.474590\du}}{\pgfpoint{53.561681\du}{19.482403\du}}{\pgfpoint{53.518712\du}{19.486309\du}}
\pgfpathcurveto{\pgfpoint{53.440587\du}{19.482403\du}}{\pgfpoint{53.385900\du}{19.462872\du}}{\pgfpoint{53.354650\du}{19.415997\du}}
\pgfpathcurveto{\pgfpoint{53.323400\du}{19.380840\du}}{\pgfpoint{53.311681\du}{19.333965\du}}{\pgfpoint{53.315587\du}{19.271465\du}}
\pgfpathcurveto{\pgfpoint{53.311681\du}{19.205059\du}}{\pgfpoint{53.331212\du}{19.150372\du}}{\pgfpoint{53.370275\du}{19.107403\du}}
\pgfpathcurveto{\pgfpoint{53.405431\du}{19.068340\du}}{\pgfpoint{53.448400\du}{19.044903\du}}{\pgfpoint{53.506994\du}{19.044903\du}}
\pgfpathcurveto{\pgfpoint{53.565587\du}{19.044903\du}}{\pgfpoint{53.616369\du}{19.068340\du}}{\pgfpoint{53.655431\du}{19.107403\du}}
\pgfpathcurveto{\pgfpoint{53.690587\du}{19.154278\du}}{\pgfpoint{53.710119\du}{19.216778\du}}{\pgfpoint{53.710119\du}{19.298809\du}}
\pgfpathlineto{\pgfpoint{53.428869\du}{19.298809\du}}
\pgfpathcurveto{\pgfpoint{53.428869\du}{19.333965\du}}{\pgfpoint{53.436681\du}{19.361309\du}}{\pgfpoint{53.452306\du}{19.376934\du}}
\pgfpathcurveto{\pgfpoint{53.467931\du}{19.400372\du}}{\pgfpoint{53.487462\du}{19.408184\du}}{\pgfpoint{53.518712\du}{19.404278\du}}
\pgfpathcurveto{\pgfpoint{53.534337\du}{19.408184\du}}{\pgfpoint{53.549962\du}{19.404278\du}}{\pgfpoint{53.561681\du}{19.392559\du}}
\pgfpathcurveto{\pgfpoint{53.573400\du}{19.384747\du}}{\pgfpoint{53.585119\du}{19.369122\du}}{\pgfpoint{53.592931\du}{19.345684\du}}
\pgfpathlineto{\pgfpoint{53.592931\du}{19.345684\du}}
\pgfusepath{stroke}
\definecolor{dialinecolor}{rgb}{0.000000, 0.000000, 0.000000}
\pgfsetstrokecolor{dialinecolor}
\pgfpathmoveto{\pgfpoint{53.596837\du}{19.232403\du}}
\pgfpathcurveto{\pgfpoint{53.592931\du}{19.201153\du}}{\pgfpoint{53.585119\du}{19.177715\du}}{\pgfpoint{53.573400\du}{19.158184\du}}
\pgfpathcurveto{\pgfpoint{53.553869\du}{19.142559\du}}{\pgfpoint{53.534337\du}{19.134747\du}}{\pgfpoint{53.514806\du}{19.130840\du}}
\pgfpathcurveto{\pgfpoint{53.487462\du}{19.134747\du}}{\pgfpoint{53.467931\du}{19.142559\du}}{\pgfpoint{53.452306\du}{19.158184\du}}
\pgfpathcurveto{\pgfpoint{53.436681\du}{19.177715\du}}{\pgfpoint{53.428869\du}{19.201153\du}}{\pgfpoint{53.428869\du}{19.232403\du}}
\pgfpathlineto{\pgfpoint{53.596837\du}{19.232403\du}}
\pgfusepath{stroke}
\definecolor{dialinecolor}{rgb}{0.000000, 0.000000, 0.000000}
\pgfsetstrokecolor{dialinecolor}
\pgfpathmoveto{\pgfpoint{53.792150\du}{19.056622\du}}
\pgfpathlineto{\pgfpoint{53.893712\du}{19.056622\du}}
\pgfpathlineto{\pgfpoint{53.893712\du}{19.115215\du}}
\pgfpathcurveto{\pgfpoint{53.928869\du}{19.068340\du}}{\pgfpoint{53.971837\du}{19.044903\du}}{\pgfpoint{54.026525\du}{19.044903\du}}
\pgfpathcurveto{\pgfpoint{54.049962\du}{19.044903\du}}{\pgfpoint{54.073400\du}{19.052715\du}}{\pgfpoint{54.096837\du}{19.064434\du}}
\pgfpathcurveto{\pgfpoint{54.116369\du}{19.076153\du}}{\pgfpoint{54.131994\du}{19.091778\du}}{\pgfpoint{54.143712\du}{19.115215\du}}
\pgfpathcurveto{\pgfpoint{54.163244\du}{19.091778\du}}{\pgfpoint{54.182775\du}{19.076153\du}}{\pgfpoint{54.206212\du}{19.064434\du}}
\pgfpathcurveto{\pgfpoint{54.221837\du}{19.052715\du}}{\pgfpoint{54.245275\du}{19.044903\du}}{\pgfpoint{54.272619\du}{19.044903\du}}
\pgfpathcurveto{\pgfpoint{54.299962\du}{19.044903\du}}{\pgfpoint{54.327306\du}{19.052715\du}}{\pgfpoint{54.350744\du}{19.064434\du}}
\pgfpathcurveto{\pgfpoint{54.374181\du}{19.080059\du}}{\pgfpoint{54.389806\du}{19.099590\du}}{\pgfpoint{54.401525\du}{19.123028\du}}
\pgfpathcurveto{\pgfpoint{54.405431\du}{19.142559\du}}{\pgfpoint{54.409337\du}{19.169903\du}}{\pgfpoint{54.413244\du}{19.208965\du}}
\pgfpathlineto{\pgfpoint{54.413244\du}{19.478497\du}}
\pgfpathlineto{\pgfpoint{54.299962\du}{19.478497\du}}
\pgfpathlineto{\pgfpoint{54.299962\du}{19.236309\du}}
\pgfpathcurveto{\pgfpoint{54.296056\du}{19.197247\du}}{\pgfpoint{54.292150\du}{19.169903\du}}{\pgfpoint{54.288244\du}{19.154278\du}}
\pgfpathcurveto{\pgfpoint{54.272619\du}{19.142559\du}}{\pgfpoint{54.256994\du}{19.134747\du}}{\pgfpoint{54.241369\du}{19.130840\du}}
\pgfpathcurveto{\pgfpoint{54.221837\du}{19.134747\du}}{\pgfpoint{54.206212\du}{19.138653\du}}{\pgfpoint{54.194494\du}{19.146465\du}}
\pgfpathcurveto{\pgfpoint{54.178869\du}{19.158184\du}}{\pgfpoint{54.171056\du}{19.173809\du}}{\pgfpoint{54.167150\du}{19.189434\du}}
\pgfpathcurveto{\pgfpoint{54.159337\du}{19.212872\du}}{\pgfpoint{54.155431\du}{19.240215\du}}{\pgfpoint{54.159337\du}{19.275372\du}}
\pgfpathlineto{\pgfpoint{54.159337\du}{19.478497\du}}
\pgfpathlineto{\pgfpoint{54.046056\du}{19.478497\du}}
\pgfpathlineto{\pgfpoint{54.046056\du}{19.248028\du}}
\pgfpathcurveto{\pgfpoint{54.046056\du}{19.208965\du}}{\pgfpoint{54.042150\du}{19.181622\du}}{\pgfpoint{54.038244\du}{19.169903\du}}
\pgfpathcurveto{\pgfpoint{54.034337\du}{19.158184\du}}{\pgfpoint{54.030431\du}{19.150372\du}}{\pgfpoint{54.022619\du}{19.142559\du}}
\pgfpathcurveto{\pgfpoint{54.014806\du}{19.138653\du}}{\pgfpoint{54.003087\du}{19.134747\du}}{\pgfpoint{53.987462\du}{19.130840\du}}
\pgfpathcurveto{\pgfpoint{53.971837\du}{19.134747\du}}{\pgfpoint{53.956212\du}{19.138653\du}}{\pgfpoint{53.940587\du}{19.146465\du}}
\pgfpathcurveto{\pgfpoint{53.924962\du}{19.158184\du}}{\pgfpoint{53.913244\du}{19.169903\du}}{\pgfpoint{53.909337\du}{19.185528\du}}
\pgfpathcurveto{\pgfpoint{53.901525\du}{19.205059\du}}{\pgfpoint{53.897619\du}{19.236309\du}}{\pgfpoint{53.901525\du}{19.271465\du}}
\pgfpathlineto{\pgfpoint{53.901525\du}{19.478497\du}}
\pgfpathlineto{\pgfpoint{53.792150\du}{19.478497\du}}
\pgfpathlineto{\pgfpoint{53.792150\du}{19.056622\du}}
\pgfusepath{stroke}
% setfont left to latex
\definecolor{dialinecolor}{rgb}{0.000000, 0.000000, 0.000000}
\pgfsetstrokecolor{dialinecolor}
\node[anchor=west] at (24.569414\du,24.774004\du){Attribut / Partikel};
% setfont left to latex
\definecolor{dialinecolor}{rgb}{0.000000, 0.000000, 0.000000}
\pgfsetstrokecolor{dialinecolor}
\node[anchor=west] at (24.787515\du,27.936468\du){Kernel};
% setfont left to latex
\definecolor{dialinecolor}{rgb}{0.000000, 0.000000, 0.000000}
\pgfsetstrokecolor{dialinecolor}
\node[anchor=west] at (24.842040\du,30.117478\du){Globale Synchronisation};
\end{tikzpicture}

  \caption{Reihenfolge der Kernel mit Synchronisationspunkten}
  \label{fig:simulation_kernelablauf}
\end{figure}
