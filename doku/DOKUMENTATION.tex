\documentclass[a4paper,12pt,oneside]{amsbook}

\linespread{1.05}

\usepackage{amssymb}

\usepackage{amsfonts}

% Deutsche Sprache verwenden
\usepackage{german}

% Umlaute normal eingeben
\usepackage[utf8]{inputenc}

\usepackage{graphicx}

\usepackage{tikz}

\usepackage{listings}
\renewcommand{\lstlistingname}{Code}

\usepackage{color}

\definecolor{mygreen}{rgb}{0,0.6,0}
\definecolor{mygray}{rgb}{0.5,0.5,0.5}
\definecolor{mymauve}{rgb}{0.58,0,0.82}
\lstset{ %
  backgroundcolor=\color{white},   % choose the background color; you must add \usepackage{color} or \usepackage{xcolor}
  basicstyle=\footnotesize,        % the size of the fonts that are used for the code
  breakatwhitespace=false,         % sets if automatic breaks should only happen at whitespace
  breaklines=true,                 % sets automatic line breaking
  captionpos=b,                    % sets the caption-position to bottom
  commentstyle=\color{mygreen},    % comment style
  deletekeywords={...},            % if you want to delete keywords from the given language
  escapeinside={\%*}{*)},          % if you want to add LaTeX within your code
  extendedchars=true,              % lets you use non-ASCII characters; for 8-bits encodings only, does not work with UTF-8
  frame=single,                    % adds a frame around the code
  keepspaces=true,                 % keeps spaces in text, useful for keeping indentation of code (possibly needs columns=flexible)
  keywordstyle=\color{blue},       % keyword style
  morekeywords={*,...},            % if you want to add more keywords to the set
  numbers=left,                    % where to put the line-numbers; possible values are (none, left, right)
  numbersep=5pt,                   % how far the line-numbers are from the code
  numberstyle=\tiny\color{mygray}, % the style that is used for the line-numbers
  rulecolor=\color{black},         % if not set, the frame-color may be changed on line-breaks within not-black text (e.g. comments (green here))
  showspaces=false,                % show spaces everywhere adding particular underscores; it overrides 'showstringspaces'
  showstringspaces=false,          % underline spaces within strings only
  showtabs=false,                  % show tabs within strings adding particular underscores
  stepnumber=1,                    % the step between two line-numbers. If it's 1, each line will be numbered
  stringstyle=\color{mymauve},     % string literal style
  tabsize=2,                       % sets default tabsize to 2 spaces
  title=\lstname                   % show the filename of files included with \lstinputlisting; also try caption instead of title
}

% oben Ueberschriften
\pagestyle{headings}

\textwidth=15 cm
\textheight=22 cm
\topmargin=1.2 cm
\oddsidemargin=0.5 cm
\evensidemargin=0.5 cm
\footskip=40 pt
\parskip5pt


\begin{document}


\begin{titlepage}
\begin{center}
{\large Universität Osnabrück}


\vspace{9em}
{\LARGE \emph{Smoothed Particle Hydrodynamics}}

\vspace{2em}
{\large Tobias Graf\\ Dominik Grodt\\ Sascha Bachmann}

\vspace{9em}
{\large Projektdokumentation zur Veranstaltung\\ Parallele Algorithmen mit OpenCL\\ SoSe 2013}



\end{center}
\end{titlepage}

\tableofcontents

\chapter*{Einleitung}
\thispagestyle{empty}

\begin{center}
\emph{{\small Dominik Grodt}}
\end{center}

\bigskip

Unter \textit{Smoothed Particle Hydrodynamics} (SPH) wird eine Methode verstanden, hydrodynamische Gleichungen numerisch zu lösen und somit das Verhalten von beliebigen Flüssigkeiten zu simulieren. Die Flüssigkeit wird dabei mithilfe von Partikeln diskretisiert um die Anzahl an Akteuren und deren Berechnungen auf ein endliches Maß zu reduzieren.\\
Abhängig von den umliegenden Elementen, der Zeit und dem eigenen Zustand wird daraufhin die Zustandsveränderung berechnet, wobei die daran beteiligten Formeln reale - oder gewünschte - physikalische Kräfte nachahmen sollen. Durch das Verhalten der einzelnen Partikel entsteht auf diese Weise auf der Makroebene das gewünschte Gesamtverhalten der Flüssigkeit. \\
Gegenstand dieses Projektes ist neben der möglichst korrekten Simulation von Wasser mittels SPH auch eine möglichst ansprechende Visualisierung desselben. Dazu wird die zur Berechnung nötige Diskretisierung aufgehoben und eine auf den Partikeln aufbauende, aber kontinuierliche Wasseroberfläche generiert. Mithilfe von gängigen Methoden aus der Computergrafik wird diese daraufhin in einem weiteren Schritt verfeinert um den Eindruck von Wasser zu verstärken.
\chapter{Programmstruktur}
\thispagestyle{empty}

\begin{center}
\emph{{\small Dominik Grodt}}
\end{center}

\bigskip

Text...

\chapter{Der SPH Algorithmus}
\label{chap:sph}
\thispagestyle{empty}

\section{Physikalische Herleitung}

\begin{center}
\emph{{\small Sascha Bachmann}}
\end{center}

\bigskip

Text...

\section{Implementation und Performance}


\begin{center}
\emph{{\small Dominik Grodt}}
\end{center}

\bigskip

Text...

\subsection{Datenstruktur und Neighboring Search}
\subsection{OpenCL und Parallelisierung}


\pagebreak
\chapter{Visualizierung}
\label{chap:visualisierung}
\thispagestyle{empty}

\noindent Um die Visualisierung zu verbessern und nicht nur einzelne Punkte zu rendern haben wir uns für die erstellung eines Drahtgittermodells entschieden, welches aus unseren Datenstrukturen erstellt werden soll. Nach kurzer Recherche kamen wir somit auf Marching Cubes und in diesem Kapitel wird allgemein einmal der Algorithmus vorgestellt und speziell auf unsere Implementierung eingegangen.

\section{Marching Cubes}
\subsection{Allgemeine Beschreibung des Algorithmus}

\begin{center}
\emph{{\small Tobias Graf}}
\end{center}

\bigskip

\noindent Der Marching Cubes Algorithmus ziehlt darauf ab eine sogenannte Voxel - Datenmenge, ein Wurfelförmiges dreidimensionales Gitter aus Datenpunkten, in eine flächendeckende polygonale Oberfläche umzuwandeln und wird meistens für Medizische Anwendungen benutz. Da Computertomografen im allgemeinen bereits Voxel-Datenmengen liefern und für die Visualisierung der Ergebnisse Drahtgittermodelle benötigt werden, ist dies der größte Anwendungsbereich. 
\medskip
\noindent Hinweis: \textit{Für die allgemeine Erklärung wird nicht weiter auf die Erstellung von Voxel-Datenmengen eingegangen, im spezifischen Teil wird erläutert wie sie in unserem Projekt erstellt wird. }
\medskip
\noindent Zuerst wird das Datenmodell in einzelne Würfel unterteilt deren Eckpunkte die Voxel sind die die Dichte der Oberfläche an dem betreffendem Eckpunkt des Würfels darstellen. Abhängig davon ob eine gewisse Dichte erreicht wird muss in dem Würfel ein Drahtgittermodell erstellt werden welches die belegten Voxel verdeckt. Dazu werden an den Kanten des Würfels Vertices angelegt welche nach einem bestimmten Schema miteinander verknüpft werden. 
\begin{align*}
\noindent
\includegraphics[scale=0.9]{images/Marching_Cubes_Kubus}
\end{align*}
\medskip 
\noindent Da sich pro Würfel acht Vertices und zwei Zustände (Innen und Aussen) ergeben sich $2^8 = 256$ Konfigurationen wie das Drahtgittermodell aufgebaut sein muss. Da jedoch die Triangulation der $256$ Konfigurationen ziemlich Fehleranfällig ist kann durch zwei Symmetrien eines Würfels das Problem auf $14$ Fälle (nach \cite{MC}) bzw. $17$ Fälle (nach \cite{MCADD}) reduziert werden. Zum einen ist die Topologie der triangulierten Oberfläche unverändert wenn sie nach Innen oder Aussen zeigt und Fälle mit $0$ bis $4$ Vertices für die Darstellung interessant sind reduziert sich das Konfigurationsspektrum bereits auf $128$. Als zweites kommt die Symmetrie der Rotation hinzu wodurch die $14$ Vorlagen ausreichen um die meisten Fälle abzudecken. 
\begin{align*}
\noindent
\includegraphics[scale=0.9]{images/MarchingCubesConfig}
\end{align*}
\medskip 
\noindent Die einfachste Konfiguration $0$ tritt auf wenn alle Voxel - Werte über (oder unter entsprechend Symmetrie eins) dem Schwellwert für die Darstellung liegen und generiert keine Dreiecke des Drahtgittermodells. Die nächste Vorlage $1$ tritt auf wenn lediglich ein Voxel - Wert die Bedingung erfüllt und erstellt ein zu renderndes Dreieck. Alle anderen Konfigurationen erstellen mehrere Dreiecke. Permutationen dieser Vorlagen unter ausnutzung der Komplementären und Rotationssymmetry ergeben die $256$ möglichen Fälle der Darstellung.
\medskip
\noindent Diese Konfigurationen werden in einer Datentabelle abgelegt und müssen nach dem feststellen der Dichten des Würfels nur noch ausgelesen werden und als Vertex - und Index Buffers an die OpenGL Pipeline weitergereicht werden. 

\subsection{Details und Implementation}

\begin{center}
\emph{{\small Sascha Bachmann}}
\end{center}

\bigskip

Text...

\pagebreak
\section{Deferred Shading}

\begin{center}
\emph{{\small Tobias Graf}}
\end{center}

\bigskip

\subsubsection*{Einleitung} Um aus der entstandenen Drahtgitter - Oberfläche des Marching Cubes Algorithmus eine Näherung zu einer Wasseroberfläche zu erzielen verwenden wir Deferred Shading. In diesem Kapitel wird kurz erläutert was Deferred Shading generell ist und wie unsere Implementierung das ganze Nutzt um einen Transparenten Wassereffekt zu erreichen.

\subsection*{Deferred Shading} Ursprünglich ist Deferred Shading eine Technik der Computergrafik um die Lichtberechnung von der Geometrieberechnung zu trennen. Sie erlaubt durch das Reduzieren der Komplexität wesentlich mehr Lichtobjekte in einer Szene als bei klassischen Rendermethoden, in denen anhand der Tiefe, Normale und Farbwert eines Eckpunktes in Korrelation einer Lichtquelle die Dargestellte Farbe berechnet wurden. Beim Deferred Shading werden durch sogenannte Framebufferobjekte (FBO) Tiefenwerte, Normale und Farbe der Geometrien in eine Textur mit Bildschirmauflösung gerendert. Statt nun für jeden Geometrischen Eckpunkt den Farbwert zu berechnen wird dies nun für jeden Pixel angewandt wodurch der Aufwand von $O(m*n)$ auf $O(m+n)$ reduziert wird. Nachteilig dabei ist das zwar weniger Hardwareanforderungen benötigt werden um die gleiche Szene zu rendern, jedoch der Speicherverbrauch extrem ansteigt da die Texturen im Grafikkartenspeicher vorgehalten werden müssen.\\
Relativ schnell Entwickelten sich vielfältige Nutzungen der Technik zur Berechnung von Shadow Maps und einsetzen von Postprocessing Effekten.

\subsection*{Unser Ansatz} Angelehnt an den Ansatz des GDC Vortrags\cite{DSGDC} erfolgte zuerst die Implementation mehrerer Framebufferobjekte die als einzelne Rendertargets benutzt werden um verschiedene Aspekte der Szene einzeln in Texturen zu Rendern. Die Aufteilung erfolgt in folgende FBO und Grafiken:
\begin{itemize}
\item Hintergrund
\item Thickness
\item Partikel
	\begin{itemize}
	\item Farbwert
	\item Tiefenwert
	\item Welt - Koordinate
	\item Normale
	\item Spekulare Farbe
	\item Diffuse Farbe
	\end{itemize}
\end{itemize}  
\medskip
\noindent Das getrennte Rendern des Hintergrundes ist notwendig um den Transparenz Effekt zu erreichen da dies eine schwäche von Deferred Shading ist da man keinen nutzen aus der \texttt{GL_BLEND} Funktion ziehen kann wenn die einzelnen Geometrien in getrennten FBO gezeichnet werden. Die Thickness bestimmt im abschließenden Bild die Hauptsächliche Transparenz der Oberfläche. Der Partikel FBO schreibt die einzelnen Parameter mit denen die Lichtberechnung des Phong Modells und die Farbgebung des Objektes in die jeweiligen Texturen.
\begin{align*}
\noindent
\includegraphics[scale=0.9]{images/Dereffered}
\end{align*}
\medskip  


\begin{thebibliography}{}


\bibitem{IntroSPH}
J.\ J.\ Monaghan (1988),
{\em An Introduction to SPH}. Computer Physics Communication, Vol. 48, pp. 89-96.


\bibitem{FlowSPH}
J.\ J.\ Monaghan (1994),
{\em Simulating Free Surface Flows with SPH}. Journal of Computational Physics, Vol. 110, pp. 399-406.


\bibitem{FluidSim}
M.\ Müller, D.\ Charypar, M.\ Gross (2003),
{\em Particle-Based Fluid Simulation for Interactive Applications}. Proceedings of the 2003 ACM SIGGRAPH/Eurographics symposium on Computer animation, pp. 154-159.


\bibitem{BoundarySPH}
T.\ Harada, S.\ Koshizuka and Y.\ Kawaguchi (2007),
{\em Improvement in the Boundary Conditions of Smoothed Particle Hydrodynamics}. Computer Graphics \& Geometry, Vol. 9, No. 3, pp. 2-15.


\bibitem{MC}
W.\ E.\ Lorensen, H.\ E.\ Cline (1987),
{\em Marching cubes: A high resolution 3D surface construction algorithm}. SIGGRAPH Comput. Graph., Vol. 21, No. 4, pp. 163-169.


\bibitem{MCADD}
W.\ Heiden, T.\ Goetze, J.\ Brickmann (1993). {\em Fast generation of molecular surfaces from 3D data fields with an enhanced ``marching cube'' algorithm}. Journal of Computational Chemistry, Vol. 14, Iss. 2, pp. 246-250.

\end{thebibliography}


\end{document}